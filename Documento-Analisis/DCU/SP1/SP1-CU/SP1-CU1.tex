%REGISTRAR PROGRAMA SINTÉTICO: JORGE

\begin{UseCase}{SP1-CU1}{Registrar Programa Sintético}{El usuario podrá registrar el Programa Sintético correspondiente a un Plan de Estudios.}
    \UCitem{Versión}{\color{Gray}1.0}
    \UCitem{Autor}{\color{Gray}Maldonado Carpio Jorge Enrique.}
    \UCitem{Supervisa}{\color{Gray}Cervantes Delgadillo Mauricio.}
    \UCitem{Actor}{\hyperlink{Usuario}{Docente}}
    \UCitem{Propósito}{Servir como marco de referencia para el registro de los demás atributos de la Unidad de Aprendizaje.}
    \UCitem{Entradas}{Las entradas para el registro de tiempos de la Unidad de Aprendizaje serán:
      \begin{itemize}
          \item Unidad Académica.
          \item Programa Académico.
          \item Unidad de Aprendizaje.
          \item Semestre.
          \item Propósito de la Unidad de Aprendizaje.
          \item Contenidos.
          \item Orientación Didáctica.
          \item Evaluación y Acreditación.
      \end{itemize}
    }
    \UCitem{Origen}{Teclado y Mouse.}
    \UCitem{Salidas}{
    	\begin{itemize}
          \item \MSGref{MSG4}{Los campos marcados con (*) son obligatorios}.
          \item \MSGref{MSG5}{Registro finalizado exitosamente}.
          \item \MSGref{MSG6}{¿Está seguro que desea cancelar el registro?}.
     	\end{itemize}
    
    
    }
    \UCitem{Destino}{Pantalla.}
    \UCitem{Precondiciones}{Los siguientes catálogos no deben de estar vacios:
      \begin{itemize}
          \item Unidad Académica.
          \item Programa Académico.
          \item Unidad de Aprendizaje.
          \item Semestre
      \end{itemize}
      }
    \UCitem{Postcondiciones}{El Programa Sintético queda registrado en el sistema.}
    \UCitem{Errores}{}
    \UCitem{Estado}{Revisión.}
    \UCitem{Observaciones}{}
\end{UseCase}

%--------------------------- CU TRAYECTORIA PRINCIPAL -------------------------
\begin{UCtrayectoria}{Principal}

\UCpaso[\UCactor] Presiona el botón \IUbutton{Registrar Programa Sintético} de la interfaz de usuario \IUref{}{Página Principal}
\UCpaso Verifica que los catálogos Unidad Académica, Programa Académico, Unidad de Aprendizaje, Semestre tengan al menos un registro.
\UCpaso Carga la información de los catálogos.
\UCpaso Muestra la interfaz de usuario \IUref{IU.01}{Registrar el Programa Sintético}.
\UCpaso[\UCactor] Selecciona la Unidad Académica.
\UCpaso[\UCactor] Selecciona el Programa Académico.
\UCpaso[\UCactor] Selecciona la Unidad de Aprendizaje.
\UCpaso[\UCactor] Selecciona el Semestre.
\UCpaso[\UCactor] Ingresa el Propósito de la Unidad de Aprendizaje conforme al modelo de Negocios.
\UCpaso[\UCactor] Presiona el botón \IUbutton{Registrar Contenidos}
\UCpaso Abre un modal en el \UCref{SP1-CU2} [Trayectoria A]
\UCpaso[\UCactor] Ingresa la Orientación Didáctica conforme al modelo de Negocios.
\UCpaso[\UCactor] Presiona el botón \IUbutton{Registrar Evaluación y Acreditación}
\UCpaso Abre un modal en el \UCref{SP1-CU3}
\UCpaso[\UCactor] Termina la operación presionando el botón \IUbutton{Guardar}. [Trayectoria B]
\UCpaso Verifica que todos los campos marcados como obligatorios hayan sido llenados.[Trayectoria C]
\UCpaso Guarda la información del Programa Sintético.

\UCpaso El sistema muestra el mensaje \MSGref{MSG5}{Registro finalizado exitosamente}.
\UCpaso[\UCactor] Cierra el mensaje presionando el botón \IUbutton{Ok}. 



\UCpaso Muestra la interfaz de usuario \IUref{IU1}{Página principal}.
\end{UCtrayectoria}

%------------------------ CU TRAYECTORIA ALTERNARTIVA A -------------------------

\begin{UCtrayectoriaA}{A}{El usuario desea agregar más de un contenido.}

\UCpaso Continua en el paso 8 de la trayectoria principal del \UCref{SP1-CU1}.

\end{UCtrayectoriaA}

\begin{UCtrayectoriaA}{B}{El usuario desea cancelar el registro del Programa Sintético.}

\UCpaso[\UCactor] Presiona el botón \IUbutton{Cancelar}
\UCpaso Muestra el \MSGref{MSG6}{¿Está seguro que desea cancelar el registro?}.
\UCpaso[\UCactor] Presiona el botón \IUbutton{Si} [Trayectoria C.1]
\UCpaso Continua en el paso 20 de la trayectoria principal del \UCref{SP1-CU1}.

\end{UCtrayectoriaA}

\begin{UCtrayectoriaA}{B.1}{El usuario no desea cancelar el registro del Programa Sintético.}

\UCpaso[\UCactor] Presiona el botón \IUbutton{No}
\UCpaso Continua en el paso 15 de la trayectoria principal del \UCref{SP1-CU1}.

\end{UCtrayectoriaA}


\begin{UCtrayectoriaA}{C}{Uno o más campos obligatorios no fueron contestados.}
\UCpaso Detecta uno o más campos sin contestar.
\UCpaso Muestra el mensaje \MSGref{MSG4}{Los campos marcados con (*) son obligatorios}.
\UCpaso[\UCactor] Cierra el mensaje presionando el botón \IUbutton{Aceptar}.
\UCpaso Continua en el paso 5 de la trayectoria principal del \UCref{SP1-CU1}.
\end{UCtrayectoriaA}