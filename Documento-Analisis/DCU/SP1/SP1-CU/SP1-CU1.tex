%REGISTRAR PROGRAMA SINTÉTICO: JORGE

\begin{UseCase}{SP1-CU1}{Registrar Programa Sintético}{El usuario podrá registrar el Programa Sintético correspondiente a un Plan de Estudios.}
    \UCitem{Versión}{\color{Gray}1.0}
    \UCitem{Autor}{\color{Gray}Maldonado Carpio Jorge Enrique.}
    \UCitem{Supervisa}{\color{Gray}Cervantes Delgadillo Mauricio.}
    \UCitem{Actor}{\hyperlink{Usuario}{Docente y Jefe de Innovación Educativa.}}
    \UCitem{Propósito}{Servir como marco de referencia para el registro de los demás atributos de la Unidad de Aprendizaje.}
    \UCitem{Entradas}{Las entradas para el registro de tiempos de la Unidad de Aprendizaje serán:
      \begin{itemize}
          \item Propósito de la Unidad de Aprendizaje.
          \item Contenidos.
          \item Orientación Didáctica.
          \item Evaluación y Acreditación.
      \end{itemize}
    }
    \UCitem{Origen}{Teclado y Mouse.}
    \UCitem{Salidas}{
    	\begin{itemize}
          \item \MSGref{MSG4}{Los campos marcados con (*) son obligatorios.}
          \item \MSGref{MSG5}{Registro finalizado exitosamente.}
          \item \MSGref{MSG25}{Servicios no disponibles por el momento}.
          \item \MSGref{MSG29}{¿Está seguro que desea cancelar? Se perderán todos los avances sin guardar.}
          \item \MSGref{MSG35}{Escribe información válida}
          \item Unidad Académica.
          \item Programa Académico.
          \item Unidad de Aprendizaje.
          \item Semestre.
     	\end{itemize}
    }
    \UCitem{Destino}{Pantalla.}
    \UCitem{Precondiciones}{El usuario debe de haber ingresado al sistema como Docente o como Jefe de Innovación Educativa}
    \UCitem{Postcondiciones}{El Programa Sintético queda registrado en el sistema.}
    \UCitem{Errores}{}
    \UCitem{Estado}{Revisión.}
    \UCitem{Observaciones}{}
\end{UseCase}

%--------------------------- CU TRAYECTORIA PRINCIPAL -------------------------
\begin{UCtrayectoria}{Principal}

\UCpaso[\UCactor] Presiona el botón \IUbutton{Registrar Programa Sintético} de la interfaz de usuario \IUref{}{Principal}
\UCpaso Carga la información de los catálogos.
\UCpaso Muestra la interfaz de usuario \IUref{IU.01}{Registrar el Programa Sintético}.
\UCpaso[\UCactor] Ingresa el Propósito de la Unidad de Aprendizaje conforme al modelo de datos, la \BRref{BR38}{Verificación de formularios al momento} y la \BRref{BR39}{Todos los campos marcados con (*) son obligatorios} [Trayectoria D] [Trayectoria E]. 
\UCpaso[\UCactor] Ingresa la Orientación Didáctica conforme al modelo de datos, la \BRref{BR38}{Verificación de formularios al momento} y la \BRref{BR39}{Todos los campos marcados con (*) son obligatorios} [Trayectoria D] [Trayectoria E].
\UCpaso[\UCactor] Presiona el botón \IUbutton{Registrar Contenidos} [Trayectoria A].
\UCpaso[\UCactor] Presiona el botón \IUbutton{Registrar Evaluación y Acreditación} [Trayectoria B].
\UCpaso[\UCactor] Termina la operación presionando el botón \IUbutton{Guardar} [Trayectoria C].
\UCpaso Verifica que todos los campos marcados como obligatorios hayan sido llenados [Trayectoria D].
\UCpaso Guarda la información del Programa Sintético. [Trayectoria F]
\UCpaso El sistema muestra el mensaje \MSGref{MSG5}{Registro finalizado exitosamente.}
\UCpaso[\UCactor] Cierra el mensaje presionando el botón \IUbutton{Ok}. 
\UCpaso Muestra la interfaz de usuario \IUref{IU1}{Principal}.
\end{UCtrayectoria}


%------------------------ CU TRAYECTORIA ALTERNARTIVA A -------------------------
\begin{UCtrayectoriaA}{B}{El usuario desea agregar un Contenido.}
\UCpaso Abre un modal en el \UCref{SP1-CU2}.
\UCpaso Continua en el paso 2 de la trayectoria principal del \UCref{SP1-CU2}.
\end{UCtrayectoriaA}

%------------------------ CU TRAYECTORIA ALTERNARTIVA B -------------------------
\begin{UCtrayectoriaA}{B}{El usuario desea agregar la Evaluación y Acreditación.}
\UCpaso Abre un modal en el \UCref{SP1-CU3}.
\UCpaso Continua en el paso 2 de la trayectoria principal del \UCref{SP1-CU3}.
\end{UCtrayectoriaA}

%------------------------ CU TRAYECTORIA ALTERNARTIVA C -------------------------
\begin{UCtrayectoriaA}{C}{El usuario desea cancelar el registro del Programa Sintético.}
\UCpaso[\UCactor] Presiona el botón \IUbutton{Cancelar}
\UCpaso Muestra el \MSGref{MSG29}{¿Está seguro que desea cancelar? Se perderán todos los avances sin guardar.}
\UCpaso[\UCactor] Presiona el botón \IUbutton{Si} [Trayectoria C.1]
\UCpaso Continua en el paso 8 de la trayectoria principal del \UCref{SP1-CU1}.
\end{UCtrayectoriaA}

%------------------------ CU TRAYECTORIA ALTERNARTIVA C.1 -------------------------
\begin{UCtrayectoriaA}{C.1}{El usuario no desea cancelar el registro del Programa Sintético.}
\UCpaso[\UCactor] Presiona el botón \IUbutton{No}
\UCpaso Continua en el paso 8 de la trayectoria principal del \UCref{SP1-CU1}.
\end{UCtrayectoriaA}

%------------------------ CU TRAYECTORIA ALTERNARTIVA D -------------------------
\begin{UCtrayectoriaA}{D}{Uno o más campos obligatorios no fueron contestados.}
\UCpaso Detecta uno o más campos sin contestar.
\UCpaso Muestra el mensaje \MSGref{MSG4}{Los campos marcados con (*) son obligatorios.}
\UCpaso[\UCactor] Cierra el mensaje presionando el botón \IUbutton{Aceptar}.
\UCpaso Continua en el paso 4 de la trayectoria principal del \UCref{SP1-CU1}.
\end{UCtrayectoriaA}

%------------------------ CU TRAYECTORIA ALTERNARTIVA E -------------------------
\begin{UCtrayectoriaA}{E}{El sistema detecta caracteres no válidos conforme al Modelo de Datos.}
	\UCpaso Muestra el mensaje \MSGref{MSG35}{Escribe información válida} debajo del campo que incumplio con el diccionario de datos.
	\UCpaso Continúa en el paso 4 de la trayectoria principal del \UCref{SP1-CU1}.
\end{UCtrayectoriaA}

%------------------------ CU TRAYECTORIA ALTERNARTIVA F -------------------------
\begin{UCtrayectoriaA}{F}{Ocurre un error al momento de persistir los datos.}
	\UCpaso Muestra el mensaje \MSGref{MSG25}{Servicios no disponibles por el momento}.
	\UCpaso[\UCactor] Cierra el mensaje presionando el botón \IUbutton{Aceptar}.
	\UCpaso Continúa en el paso 13 de la trayectoria principal del \UCref{SP1-CU1}.
\end{UCtrayectoriaA}