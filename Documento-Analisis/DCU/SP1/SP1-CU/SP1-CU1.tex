%REGISTRAR PROGRAMA SINTÉTICO: JORGE
\chapter{Especificación de casos de uso.}

\begin{UseCase}{SP1-CU1}{Registrar Programa Sintético}{El actor podrá registrar el Programa Sintético correspondiente a una Unidad de Aprendizaje.}
    \UCitem{Versión}{\color{Gray}1.1}
    \UCitem{Autor}{\color{Gray}Maldonado Carpio Jorge Enrique.}
    \UCitem{Supervisa}{\color{Gray}Ramírez Martínez Janet Naibi.}
    \UCitem{Actor}{\hyperlink{Docente}{Docente y Jefe de Innovación Educativa.}}
    \UCitem{Propósito}{Servir como marco de referencia para el registro de los demás atributos de la Unidad de Aprendizaje.}
    \UCitem{Entradas}{Las entradas para el registro del Programa Sintético de la Unidad de Aprendizaje serán:
      \begin{itemize}
          \item Propósito de la Unidad de Aprendizaje.
          \item Orientación Didáctica.
          \item Contenidos.
          \item Evaluación y Acreditación.
      \end{itemize}
    }
    \UCitem{Origen}{Teclado y Mouse.}
    \UCitem{Salidas}{
    	\begin{itemize}
    		\item Unidad Académica.
          	\item Programa Académico.
	        \item Unidad de Aprendizaje.
	        \item Semestre.
        	\item \MSGref{MSG5}{Registro finalizado exitosamente.}
	        \item \MSGref{MSG25}{Servicios no disponibles}.
    		\item \MSGref{MSG29}{¿Está seguro que desea cancelar? Se perderán todos los avances sin guardar.}
          	\item \MSGref{MSG35}{Escribe información válida.}
     	\end{itemize}
    }
    \UCitem{Destino}{Pantalla.}
    \UCitem{Precondiciones}{
    	\begin{itemize}
    		\item El actor debe de haber ingresado al sistema como Docente o como Jefe de Innovación Educativa.
    		\item El docente debe tener asignada la unidad de aprendizaje.
    	\end{itemize}}
    \UCitem{Postcondiciones}{El Programa Sintético queda registrado en el sistema.}
    \UCitem{Errores}{}
    \UCitem{Estado}{Revisión.}
    \UCitem{Observaciones}{}
\end{UseCase}

%--------------------------- CU TRAYECTORIA PRINCIPAL -------------------------
\begin{UCtrayectoria}{Principal}

\UCpaso[\UCactor] Presiona el botón \IUbutton{Elaborar} de la interfaz de usuario \IUref{VTA}{Ver tareas}.

\UCpaso Carga la información de la Unidad de Aprendizaje registrada por el Jefe de Innovación.

\UCpaso Muestra la interfaz de usuario \IUref{RPS}{Registrar Programa Sintético}.

\UCpaso[\UCactor] Ingresa el Propósito de la Unidad de Aprendizaje con base en las reglas de negocio \BRref{BR38}{Verificación de formularios al momento} y \BRref{BR39}{Todos los campos marcados con (*) son obligatorios} [Trayectoria D] [Trayectoria E].

\UCpaso[\UCactor] Ingresa la Orientación Didáctica con base en las reglas de negocio \BRref{BR38}{Verificación de formularios al momento} y \BRref{BR39}{Todos los campos marcados con (*) son obligatorios} [Trayectoria D] [Trayectoria E].

\UCpaso[\UCactor] Ingresa contenidos a la Unidad de Aprendizaje. [Trayectoria A]

\UCpaso Muestra una tabla con los contenidos registrados en el paso anterior.

\UCpaso[\UCactor] Ingresa la evaluación y acreditación de la Unidad de Aprendizaje. [Trayectoria B]

\UCpaso Muestra una tabla con la evaluación y acreditación registrada en el paso anterior.

\UCpaso[\UCactor] Termina la operación presionando el botón \IUbutton{Finalizar}. [Trayectoria C]

\UCpaso Verifica que todos los campos marcados como obligatorios hayan sido llenados según la regla de negocio \BRref{BR39}{Todos los campos marcados con (*) son obligatorios}. [Trayectoria D]

\UCpaso Muestra el mensaje \MSGref{MSG30}{¿Está seguro que desea finalizar? Ya no podrá realizar modificaciones.}

\UCpaso[\UCactor] Confirma la operación presionando el botón \IUbutton{Si}. [Trayectoria G]

\UCpaso Cierra el mensaje.

\UCpaso Guarda la información del Programa Sintético. [Trayectoria F].

\UCpaso El sistema muestra el mensaje \MSGref{MSG5}{Registro finalizado exitosamente.}

\UCpaso[\UCactor] Cierra el mensaje presionando el botón \IUbutton{Aceptar}.

\UCpaso Deshabilita el formulario. 

\end{UCtrayectoria}


%------------------------ CU TRAYECTORIA ALTERNATIVA A -------------------------
\begin{UCtrayectoriaA}{A}{El actor desea agregar contenidos a la Unidad de Aprendizaje.}

\UCpaso[\UCactor] Presiona el botón \IUbutton{Registrar contenidos \BtnModal}.

\UCpaso Muestra la \IUref{RC}{Registrar contenido}.

\UCpaso Continua en el paso 1 de la trayectoria principal del \UCref{SP1-CU2}.
\end{UCtrayectoriaA}

%------------------------ CU TRAYECTORIA ALTERNARTIVA B -------------------------
\begin{UCtrayectoriaA}{B}{El actor desea agregar la Evaluación y Acreditación de la Unidad de Aprendizaje.}

\UCpaso[\UCactor] Presiona el botón \IUbutton{Registrar evaluación y acreditación \BtnModal}.

\UCpaso Extrae la información del catálogo ``Acreditación'' de la base de datos. [Trayectoria B.1]

\UCpaso Muestra el modal \IUref{REA}{Registrar Evaluación y Acreditación}.

\UCpaso Continua en el paso 1 de la trayectoria principal del \UCref{SP1-CU3}.
\end{UCtrayectoriaA}
%------------------------ CU TRAYECTORIA ALTERNARTIVA B.1 -------------------------

\begin{UCtrayectoriaA}{B.1}{El catálogo está vacío}
    \UCpaso No encuentra información en el catálogo ``Acreditación''.
    \UCpaso El sistema muestra el mensaje \MSGref{MSG25}{Servicios no disponibles.}
    \UCpaso[\UCactor] Cierra el mensaje presionando el botón \IUbutton{Aceptar}.
    \UCpaso Muestra la \IUref{VTA}{Ver tareas}.
\end{UCtrayectoriaA}

%------------------------ CU TRAYECTORIA ALTERNATIVA C -------------------------
\begin{UCtrayectoriaA}{C}{El actor desea guardar el progreso de su registro.}
\UCpaso[\UCactor] Presiona el botón \IUbutton{Guardar}
\UCpaso Almacena la información.
\UCpaso Muestra el \MSGref{MSG58}{Avances guardados exitosamente.}
\UCpaso[\UCactor] Presiona el botón \IUbutton{Aceptar} 
\UCpaso Muestra la interfaz de usuario \IUref{RPS}{Registrar Programa Sintético}.
\end{UCtrayectoriaA}

%------------------------ CU TRAYECTORIA ALTERNARTIVA D -------------------------
\begin{UCtrayectoriaA}{D}{Uno o más campos obligatorios no fueron contestados.}
\UCpaso Detecta uno o más campos sin contestar.

\UCpaso Muestra el mensaje \MSGref{MSG44}{Este campo es requerido} debajo de los campos que no fueron contestados.

\UCpaso Continua en el paso 4 de la trayectoria principal del \UCref{SP1-CU1}
\end{UCtrayectoriaA}

%------------------------ CU TRAYECTORIA ALTERNARTIVA E -------------------------
\begin{UCtrayectoriaA}{E}{El sistema detecta caracteres no válidos conforme al diccionario de datos.}
	\UCpaso Muestra el mensaje \MSGref{MSG35}{Escribe información válida} debajo del campo que incumplió según el diccionario de datos.
	\UCpaso Continúa en el paso 4 de la trayectoria principal del \UCref{SP1-CU1}.
\end{UCtrayectoriaA}

%------------------------ CU TRAYECTORIA ALTERNARTIVA F -------------------------
\begin{UCtrayectoriaA}{F}{Ocurre un error al momento de persistir los datos.}
	\UCpaso Muestra el mensaje \MSGref{MSG25}{Servicios no disponibles}.
	
	\UCpaso[\UCactor] Cierra el mensaje presionando el botón \IUbutton{Aceptar}.
	
	\UCpaso Muestra la interfaz de usuario \IUref{RPS}{Registrar Programa Sintético}.
\end{UCtrayectoriaA}

%------------------------ CU TRAYECTORIA ALTERNARTIVA G -------------------------
\begin{UCtrayectoriaA}{G}{El actor aún no desea finalizar el registro}
	\UCpaso[\UCactor] Presiona el botón \IUbutton{No}.
	
	\UCpaso Cierra el mensaje.
	
	\UCpaso Continúa en el paso 10 de la trayectoria principal del \UCref{SP1-CU1}.
\end{UCtrayectoriaA}