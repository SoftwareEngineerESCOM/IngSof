%REGISTRAR RELACIÓN DE PRÁCTICAS: AIKO.
\begin{UseCase}{SP1-CU10}{Registrar Relación de Prácticas}{El usuario podrá registrar la relación de prácticas correspondientes a la Unidad de Aprendizaje.}
    \UCitem{Versión}{\color{Gray}1.0}
    \UCitem{Autor}{\color{Gray}López Rivera Aiko Dallane}
    \UCitem{Supervisa}{\color{Gray}Ramírez Martínez Janet Naibi}
    \UCitem{Actor}{\hyperlink{Usuario}{Usuario}}
    \UCitem{Propósito}{Servir como marco de referencia para el registro de la relación de prácticas de la Unidad de Aprendizaje.}
    \UCitem{Entradas}{Las entradas para el registro del Perfil Docente serán:
          \begin{itemize}
            \item Unidad de Aprendizaje. 
            \item No. de Pŕactica.
            \item Nombre de la Práctica.
            \item Unidad Temática.
            \item Duración.
            \item Lugar de realización.
            \item Total de Horas.
            \item Evaluación y acreditación
          \end{itemize}
        }
    \UCitem{Origen}{Teclado y Mouse.}
    \UCitem{Salidas}{
    }
    \UCitem{Destino}{Pantalla.}
    \UCitem{Precondiciones}{
        \begin{itemize}
            \item Se debe haber elaborado previamente el Programa Sintético.
            \item Los catalogos que deben estar registrados preaviamente son: Unidad de Aprendizaje, Unidad Temática y Lugar de realización.
        \end{itemize}}
    \UCitem{Postcondiciones}{La relación de Prácticas queda registrada en el Sistema.}
    \UCitem{Errores}{}
    \UCitem{Estado}{Revisión.}
    \UCitem{Observaciones}{}
\end{UseCase}

%--------------------------- CU TRAYECTORIA PRINCIPAL -------------------------
\begin{UCtrayectoria}{Principal}

    \UCpaso[\UCactor] Presiona el botón \IUbutton{Registrar Relación de Prácticas} de la interfaz de usuario. \IUref{SP1-UI}{Página Principal}
    \UCpaso El sistema verifica que el programa sintectico haya sido registrado. [Trayectoria A]
    \UCpaso Extrae la información del catalogo de Unidades de Aprendizaje perteneciente al \UCref{SP1-CU1}. 
    \UCpaso Verifica que el catálogo contenga la información de las Unidades de Aprendizaje. [Trayectoria B]
    \UCpaso Muestra la interfaz de usuario \IUref{UI.04}{Registro de Relación de Prácticas}.
    \UCpaso[\UCactor] Selecciona la Unidad de Aprendizaje.
    \UCpaso[\UCactor] Ingresa el no. de práctica.
    \UCpaso[\UCactor] Ingresa el nombre de la práctica.
    \UCpaso[\UCactor] Ingresa la duración.
    \UCpaso[\UCactor] Presiona el botón \IUbutton{Agregar Práctica} [Trayectoria C]
    \UCpaso[\UCactor] Selecciona el lugar de realización.
    \UCpaso EL sistema calcula el Total de horas %Se hace sumando la duración de todas las praćticas registradad%
    \UCpaso[\UCactor] Presiona el botón \IUbutton{Evaluación y Acreditación}
    \UCpaso Abre un modal en el \UCref{MIU4.01} [Trayectoria D]
    \UCpaso[\UCactor] Termina la operación presionando el botón \IUbutton{Guardar}. 
    \UCpaso Verifica que todos los campos marcados como obligatorios hayan sido llenados.[Trayectoria E]
    \UCpaso Guarda la información de la Relación de Prácticas.
    \UCpaso El sistema muestra el mensaje \MSGref{MSG5}{Registro finalizado exitosamente}.
    \UCpaso[\UCactor] Cierra el mensaje presionando el botón \IUbutton{Ok}.
    \UCpaso Muestra la interfaz de usuario \IUref{SP1-UI}{Página principal}.
\end{UCtrayectoria}

%------------------------ CU TRAYECTORIA ALTERNATIVA A -------------------------

\begin{UCtrayectoriaA}{A}{No se ha realizado la elaboración del Programa Sintético.}
	\UCpaso No muestra los datos provenientes del \UCref{SP1-CU1}.
	\UCpaso Muestra el mensaje \MSGref{MSG41}{Debe llenar el Programa Sintético para realizar este registro}.
	\UCpaso[\UCactor] Cierra el mensaje presionando el botón \IUbutton{Aceptar}.
	\UCpaso Muestra la interfaz de usuario \IUref{SP1-IU}{Principal}.
\end{UCtrayectoriaA}

%------------------------ CU TRAYECTORIA ALTERNATIVA B -------------------------

\begin{UCtrayectoriaA}{B}{Uno o más catálogos están vacíos.}
	\UCpaso No encuentra información en los catálogos.
    \UCpaso Muestra el mensaje \MSGref{MSG9.}{Por el momento no se puede realizar el registro.}
    \UCpaso[\UCactor] Cierra el mensaje presionando el botón \IUbutton{Aceptar}.
	\UCpaso Muestra la interfaz de usuario \IUref{SP1-IU}{Principal}.
\end{UCtrayectoriaA}

%------------------------ CU TRAYECTORIA ALTERNATIVA C -------------------------

\begin{UCtrayectoriaA}{B}{El usuario requiere registrar otra práctica.}
    \UCpaso El sistema genera los pasos 6-9 nuevamente de la trayectoria principal del \UCref{SP1-CU9}..
    \UCpaso[\UCactor] El usuario ingresa los datos correspondientes a la práctica.
    \UCpaso{}[\UCactor] EL usuario continua con el paso 10 de la trayectoria principal del \UCref{SP1-CU9}.
    \IUref{SP1-IU}{Principal}.
\end{UCtrayectoriaA}

%------------------------ CU TRAYECTORIA ALTERNATIVA D -------------------------

\begin{UCtrayectoriaA}{D}{El docente quiere registrar la Evaluación y Acreditación.}
	\UCpaso[\UCactor] Presiona el botón \BtnModal que se encuentra a un lado del campo ``Evaluación y Acreditación'' de la \IUref{IU.04}{Registrar el Programa en Extenso}.
	\UCpaso Muestra el modal \IUref{MIU4.01}{Registrar Evaluación y Acreditación}.
	\UCpaso Continua en el paso X de la trayectoria principal de \UCref{SP1-CU10}
\end{UCtrayectoriaA}

%------------------------ CU TRAYECTORIA ALTERNATIVA E -------------------------

\begin{UCtrayectoriaA}{E}{Uno o más campos obligatorios no fueron contestados.}
	\UCpaso Detecta uno o más campos sin contestar.
    \UCpaso Muestra el mensaje \MSGref{MSG4.}{Los campos marcados con (*) son obligatorios.}
    \UCpaso[\UCactor] Cierra el mensaje presionando el botón \IUbutton{Aceptar}.
    \UCpaso Continua en el paso 17 de la trayectoria principal del \UCref{SP1-CU4}.
\end{UCtrayectoriaA}