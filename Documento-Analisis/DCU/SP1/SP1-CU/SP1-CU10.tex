%REGISTRAR RELACIÓN DE PRÁCTICAS: AIKO.
\begin{UseCase}{SP1-CU10}{Registrar Relación de Prácticas}{El usuario podrá registrar la relación de prácticas correspondientes a la Unidad de Aprendizaje.}
    \UCitem{Versión}{\color{Gray}1.0}
    \UCitem{Autor}{\color{Gray}López Rivera Aiko Dallane}
    \UCitem{Supervisa}{\color{Gray}Ramírez Martínez Janet Naibi}
    \UCitem{Actor}{\hyperlink{Usuario}{Usuario}}
    \UCitem{Propósito}{Servir como marco de referencia para el registro de la relación de prácticas de la Unidad de Aprendizaje.}
    \UCitem{Entradas}{Las entradas para el registro del Perfil Docente serán:
          \begin{itemize}
            \item Unidad de Aprendizaje. 
            \item No. de Pŕactica.
            \item Nombre de la Práctica.
            \item Unidad Temática.
            \item Duración.
            \item Lugar de realización.
            \item Total de Horas.
            \item Evaluación y acreditación
          \end{itemize}
        }
    \UCitem{Origen}{Teclado y Mouse.}
    \UCitem{Salidas}{
		\begin{itemize}

			\item \MSGref{MSG5}{Registro finalizado exitosamente.}
			\item \MSGref{MSG25}{Servicios no disponibles por el momento.}
			\item \MSGref{MSG29}{ ¿Está seguro que desea cancelar? Se perderán todos los avances sin guardar.}
			\item \MSGref{MSG35}{Escribe información válida}
			\item \MSGref{MSG41}{Debe llenar el Programa Sintético para realizar este registro}.			
			\item \MSGref{MSG44}{Este campo es requerido}
		\end{itemize}
    }
    \UCitem{Destino}{Pantalla.}
    \UCitem{Precondiciones}{
        \begin{itemize}
            \item Se debe haber elaborado previamente el Programa Sintético.
            \item Los catalogos que deben estar registrados preaviamente son: Unidad de Aprendizaje, Unidad Temática y Lugar de realización.
        \end{itemize}}
    \UCitem{Postcondiciones}{La relación de Prácticas queda registrada en el Sistema.}
    \UCitem{Errores}{
		\begin{itemize}
		\item E1. Los catálogos no se cargaron correctamente.
		\item E2. Hubo un problema al conectarse con la base de datos.
		\end{itemize}
    }
    \UCitem{Estado}{Revisión.}
    \UCitem{Observaciones}{}
\end{UseCase}

%--------------------------- CU TRAYECTORIA PRINCIPAL -------------------------
\begin{UCtrayectoria}{Principal}

    \UCpaso[\UCactor] Presiona el botón \IUbutton{Registrar Relación de Prácticas} de la \IUref{SP1-IU}{Principal}.
    \UCpaso Verifica que el programa sintectico haya sido registrado. [Trayectoria A]
    \UCpaso Carga los catálogos de ``Unidad de Aprendizaje´´, ´´Unidad Temática´´ y ´´Lugar de realización´´. descritos en la \BRref{BR14}{Existen los catálogos}.[Trayectoria B]
    \UCpaso Muestra la interfaz de usuario \IUref{UI.04}{Registro de Relación de Prácticas}.
    \UCpaso[\UCactor] Selecciona la Unidad de Aprendizaje.
    \UCpaso[\UCactor] Ingresa el no. de práctica, el sistema verifica conforme al modelo de datos, la \BRref{BR38}{Verificación de formularios al momento} y la \BRref{BR39}{Todos los campos marcados con (*) son obligatorios}.[Trayectoria C]
    \UCpaso[\UCactor] Ingresa el nombre de la práctica, el sistema verifica conforme al modelo de datos, la \BRref{BR38}{Verificación de formularios al momento} y la \BRref{BR39}{Todos los campos marcados con (*) son obligatorios}.[Trayectoria C]
    \UCpaso[\UCactor] Ingresa la duración, el sistema verifica conforme al modelo de datos, la \BRref{BR38}{Verificación de formularios al momento} y la \BRref{BR39}{Todos los campos marcados con (*) son obligatorios}.[Trayectoria C]
    \UCpaso[\UCactor] Presiona el botón \IUbutton{Agregar Práctica} [Trayectoria D]
    \UCpaso[\UCactor] Selecciona el lugar de realización.
    \UCpaso EL sistema calcula el Total de horas %Se hace sumando la duración de todas las praćticas registradad%
    \UCpaso[\UCactor] Presiona el botón \BtnModal{Evaluación y Acreditación}
    \UCpaso Abre un modal en el \UCref{SPI-CU9}

\UCpaso[\UCactor] Termina la operación presionando el botón \IUbutton{Finalizar}. [Trayectoria E] [Trayectoria E.1]
	\UCpaso Verifica que se cumpla con el modelos de datos. [Trayectoria C]
	\UCpaso Persiste los datos ingresados. [Trayectoria F]
	\UCpaso El sistema muestra el mensaje \MSGref{MSG5}{Registro finalizado exitosamente}.
	\UCpaso[\UCactor] Cierra el mensaje presionando el botón \IUbutton{Aceptar}.
	\UCpaso Muestra la interfaz de usuario {SP1-IU}{Principal}.
\end{UCtrayectoria}

%------------------------ CU TRAYECTORIA ALTERNATIVA A -------------------------

\begin{UCtrayectoriaA}{A}{No se ha realizado la elaboración del Programa Sintético.}
	\UCpaso No muestra los datos provenientes del \UCref{SP1-CU1}.
	\UCpaso Muestra el mensaje \MSGref{MSG41}{Debe llenar el Programa Sintético para realizar este registro}.
	\UCpaso[\UCactor] Cierra el mensaje presionando el botón \IUbutton{Aceptar}.
	\UCpaso Muestra la interfaz de usuario \IUref{SP1-IU}{Principal}.
\end{UCtrayectoriaA}

%------------------------ CU TRAYECTORIA ALTERNARIVA B -------------------------
\begin{UCtrayectoriaA}{B}{ Los catálogos de la \BRref{BR14}{Existen los catálogos} necesarios no se pudieron cargar.}
	\UCpaso Muestra el mensaje \MSGref{MSG25}{ Servicios no disponibles por el momento. }
	\UCpaso[\UCactor] Cierra el mensaje presionando el botón \IUbutton{Aceptar}.
	\UCpaso Muestra la interfaz de usuario \IUref{SP1-IU}{Principal}.
\end{UCtrayectoriaA}

%------------------------ CU TRAYECTORIA ALTERNARIVA C -------------------------
\begin{UCtrayectoriaA}{C}{El sistema detecta uno o más campos sin contestar.}
	\UCpaso Muestra el mensaje \MSGref{MSG44}{Este campo es requerido} debajo de cada campo obligatorio sin contestar.
	\UCpaso Continúa en el paso 5 de la trayectoria principal del \UCref{SP1-CU10}.
\end{UCtrayectoriaA}

%------------------------ CU TRAYECTORIA ALTERNATIVA D -------------------------

\begin{UCtrayectoriaA}{B}{El usuario requiere registrar otra práctica.}
    \UCpaso El sistema genera los pasos 5-11 nuevamente de la trayectoria principal del \UCref{SP1-CU10}..
    \UCpaso[\UCactor] El usuario ingresa los datos correspondientes a la práctica.
    \UCpaso{}[\UCactor] EL usuario continua con el paso 12 de la trayectoria principal del \UCref{SP1-CU10}.
\end{UCtrayectoriaA}

%------------------------ CU TRAYECTORIA ALTERNARIVA E -------------------------
\begin{UCtrayectoriaA}{E}{El usuario presiona el botón \IUbutton{Cancelar}.}
	\UCpaso Muestra el mensaje \MSGref{MSG29}{¿Está seguro que desea cancelar? Se perderán todos los avances sin guardar}.
	\UCpaso[\UCactor] Confirma la operación presionando el botón \IUbutton{Si}.
	\UCpaso Muestra la interfaz de usuario {SP1-IU}{Principal}.
\end{UCtrayectoriaA}

%------------------------ TRAYECTORIA ALTERNARIVA E.1 -------------------------
\begin{UCtrayectoriaA}{E.1}{El usuario no desea cancelar el Registro de COntedidos \IUbutton{Cancelar}.}
	\UCpaso[\UCactor] Presiona el botón \IUbutton{Cancelar}
	\UCpaso Muestra el mensaje \MSGref{MSG29}{¿Está seguro que desea cancelar? Se perderán todos los avances sin guardar}.
	\UCpaso[\UCactor] Presiona el botón \IUbutton{No}.
	\UCpaso Cierra el mensaje.
	\UCpaso Continúa en el paso 12 de la trayectoria principal del \UCref{SP1-CU10}.
\end{UCtrayectoriaA}


%------------------------ CU TRAYECTORIA ALTERNARIVA F -------------------------
\begin{UCtrayectoriaA}{E}{Ocurre un error al momento de persistir los datos.}
	\UCpaso Muestra el mensaje \MSGref{MSG25}{Servicios no disponibles por el momento}.
	\UCpaso[\UCactor] Cierra el mensaje presionando el botón \IUbutton{Aceptar}.
	\UCpaso Muestra la interfaz de usuario \IUref{SP1-IU}{Principal}.
\end{UCtrayectoriaA}
