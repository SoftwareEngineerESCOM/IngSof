%REGISTRAR RELACIÓN DE PRÁCTICAS: AIKO.
\begin{UseCase}{SP1-CU10}{Registrar Relación de Prácticas}{El usuario podrá registrar la relación de prácticas correspondientes a una Unidad de Aprendizaje.}
    \UCitem{Versión}{\color{Gray}1.1}
    \UCitem{Autor}{\color{Gray}López Rivera Aiko Dallane}
    \UCitem{Supervisa}{\color{Gray}Ramírez Martínez Janet Naibi}
    \UCitem{Actor}{\hyperlink{Docente}{Docente}}
    \UCitem{Propósito}{Servir como marco de referencia para el registro de la relación de prácticas de una Unidad de Aprendizaje y poder asociar ciertas prácticas a una unidad temática.}
    \UCitem{Entradas}{Las entradas para el registro de prácticas serán:
          \begin{itemize}
            \item Lugar de realización.
            \item Número de pŕactica.
            \item Nombre de la Práctica.
            \item Unidad Temática a la que pertenece la práctica.
            \item Duración de la práctica.
            \item Acreditación de la práctica.
            \item Nombre de la evaluación de la práctica. 
            \item Porcentaje de la evaluación.
          \end{itemize}
        }
    \UCitem{Origen}{Teclado y Mouse.}
    \UCitem{Salidas}{
        \begin{itemize}
            \item MSG4. Los campos marcados con (*) son obligatorios.
            \item MSG5. Registro finalizado exitosamente.
            \item MSG23. Los porcentajes de evaluación no cumplen con el porcentaje total obligatorio.
            \item MSG25. Servicios no disponibles.
            \item MSG58. Avances guardados exitosamente.
            \item MSG30. ¿Está seguro que desea finalizar? Ya no podrá realizar modificaciones.
            \item MSG41. Debe llenar el Programa Sintético para realizar este registro.
            \item MSG61. Debe llenar las Unidades Temáticas para realizar este registro.
        \end{itemize}
    }
    \UCitem{Destino}{Pantalla.}
    \UCitem{Precondiciones}{
        \begin{itemize}
            \item Se debe haber elaborado previamente el Programa Sintético
            \item Se debe haber realizado el registro de Unidad Temática.
        \end{itemize}
    }
    \UCitem{Postcondiciones}{La relación de Prácticas queda registrada en el Sistema.}
    \UCitem{Errores}{}
    \UCitem{Estado}{Revisión.}
    \UCitem{Observaciones}{}
\end{UseCase}

%--------------------------- CU TRAYECTORIA PRINCIPAL -------------------------
\begin{UCtrayectoria}{Principal}

    \UCpaso[\UCactor] Selecciona la opción \IUbutton{Relación de Prácticas} del menú superior.
    
    \UCpaso Verifica que el programa sintético haya sido registrado. [Trayectoria A]
    
    \UCpaso Verifica que el registro de las unidades temáticas ya haya sido elaborado. [Trayectoria B]
    
    \UCpaso Muestra la interfaz de usuario \IUref{RRP}{Registrar Relación de Prácticas}.
    
    \UCpaso[\UCactor] Ingresa el lugar de realización de la práctica.
    
    \UCpaso[\UCactor] Ingresa el número de la práctica. 
    
    \UCpaso[\UCactor] Ingresa el nombre de la práctica.
    
    \UCpaso[\UCactor] Selecciona la unidad temática a la que pertenece la práctica que está seleccionando.
    
    \UCpaso[\UCactor] Ingresa la duración de la práctica.
    
    \UCpaso[\UCactor] Presiona el botón \IUbutton{Agregar práctica}. [Trayectoria C] [Trayectoria D]
    
    \UCpaso Muestra la práctica registrada anteriormente en una tabla. [Trayectoria E]
    
    \UCpaso Limpia el formulario del número, nombre, unidad temática y duración de la práctica.
    
    \UCpaso[\UCactor] Ingresa la forma de acreditación de la práctica.
    
    \UCpaso[\UCactor] Ingresa el nombre de la evaluación de la práctica.
    
    \UCpaso[\UCactor] Ingresa el porcentaje de la evaluación registrada anteriormente con base en la regla \BRref{BR25}{La suma de los porcentajes de cada evaluación debe ser igual a 100\%}.
    
    \UCpaso[\UCactor] Termina la operación presionando el botón \IUbutton{Finalizar}. [Trayectoria F] [Trayectoria G] [Trayectoria L]
    
    \UCpaso Verifica que todos los campos marcados como obligatorios hayan sido completamente llenados. [Trayectoria H]
    
    \UCpaso Verifica que la suma de los porcentajes cumpla con la regla de negocio \BRref{BR25}{La suma de los porcentajes de cada evaluación debe ser igual a 100\%}. [Trayectoria I]
    
    \UCpaso Muestra el mensaje \MSGref{MSG30}{¿Está seguro que desea finalizar? Ya no podrá realizar modificaciones.}

    \UCpaso[\UCactor] Confirma la operación presionando el botón \IUbutton{Si}. [Trayectoria J]
    
    \UCpaso Guarda la información del Programa en Extenso en la base de datos. [Trayectoria K]
    
    \UCpaso Muestra el mensaje \MSGref{MSG5}{Registro finalizado exitosamente}.
    
    \UCpaso[\UCactor] Cierra el mensaje presionando el botón \IUbutton{Aceptar}.
    
    \UCpaso Muestra la interfaz de usuario \IUref{RRP}{Registrar Relación de Prácticas}.
\end{UCtrayectoria}

%------------------------ CU TRAYECTORIA ALTERNATIVA A -------------------------

\begin{UCtrayectoriaA}{A}{No se ha realizado el registro del Programa Sintético.}
    \UCpaso[\UCactor] Selecciona la opción \IUbutton{Relación de prácticas} del menú superior.
    \UCpaso Detecta que no se ha realizado el registro del Programa Sintético.
        \UCpaso Muestra el mensaje \MSGref{MSG41}{Debe llenar el Programa Sintético para realizar este registro}.
    \UCpaso[\UCactor] Cierra el mensaje presionando el botón \IUbutton{Aceptar}.
    \UCpaso Muestra la \IUref{RRP}{Registrar Relación de Prácticas.}
\end{UCtrayectoriaA}

%------------------------ CU TRAYECTORIA ALTERNATIVA B -------------------------

\begin{UCtrayectoriaA}{B}{No se ha realizado el registro de las Unidades Temáticas.}
    \UCpaso[\UCactor] Selecciona la opción \IUbutton{Relación de prácticas} del menú superior.
    \UCpaso Detecta que no se ha realizado el registro de las Unidades Temáticas.
        \UCpaso Muestra el mensaje \MSGref{MSG61}{Debe llenar las Unidades Temáticas para realizar este registro}.
    \UCpaso[\UCactor] Cierra el mensaje presionando el botón \IUbutton{Aceptar}.
    \UCpaso Muestra la \IUref{RRP}{Registrar Relación de Prácticas.}
\end{UCtrayectoriaA}

%------------------------ CU TRAYECTORIA ALTERNATIVA C -------------------------

\begin{UCtrayectoriaA}{C}{El actor ingresa un número de práctica que ya ha sido registrado.}
    \UCpaso[\UCactor] Ingresa un número de práctica que ya ha sido registrado anteriormente.
    \UCpaso Muestra el mensaje \MSGref{MSG62}{Ya existe una práctica con el número indicado}
    \UCpaso[\UCactor] Cierra el mensaje presionando el botón \IUbutton{Aceptar}.
    \UCpaso Cierra el mensaje. 
    \UCpaso Continua en el paso 6 de la trayectoria principal del \UCref{SP1-CU10}.
\end{UCtrayectoriaA}

%------------------------ CU TRAYECTORIA ALTERNATIVA D -------------------------

\begin{UCtrayectoriaA}{D}{El actor desea registrar otra práctica.}
    \UCpaso[\UCactor] Presiona el botón \IUbutton{Agregar práctica}.
    \UCpaso Muestra la práctica registrada anteriormente en una tabla.
    \UCpaso Limpia el formulario del número, nombre, unidad temática y duración de la práctica.
    \UCpaso Continua en el paso 6 de la trayectoria pricipal del \UCref{SP1-CU10}.
\end{UCtrayectoriaA}

%------------------------ CU TRAYECTORIA ALTERNATIVA E -------------------------

\begin{UCtrayectoriaA}{E}{El actor desea eliminar la práctica que agregó.}
    \UCpaso[\UCactor] Presiona el botón \IUbutton{Eliminar}.
    \UCpaso Borra el registro de la práctica de la tabla.
\end{UCtrayectoriaA}

%------------------------ CU TRAYECTORIA ALTERNATIVA F -------------------------

\begin{UCtrayectoriaA}{F}{El actor desea guardar el progreso de su registro.}
\UCpaso[\UCactor] Presiona el botón \IUbutton{Guardar}
\UCpaso Almacena la información.
\UCpaso Muestra el \MSGref{MSG58}{Avances guardados exitosamente.}
\UCpaso[\UCactor] Presiona el botón \IUbutton{Aceptar} 
\UCpaso Muestra la interfaz de usuario \IUref{RRP}{Registrar Relación de Prácticas}.
\end{UCtrayectoriaA}

%------------------------ CU TRAYECTORIA ALTERNATIVA G -------------------------

\begin{UCtrayectoriaA}{G}{El actor quiere agregar más criterios de evaluación.}
    \UCpaso[\UCactor] Quiere agregar más criterios de evaluación para las prácticas.
    \UCpaso[\UCactor] Presiona el botón \IUbutton{Agregar evaluación}.
    \UCpaso Despliega otro campo ``Evaluación'' con su respectivo campo ``Porcentaje''.
    \UCpaso Continua en el paso 14 de la trayectoria principal del \UCref{SP1-CU10}.
\end{UCtrayectoriaA}

%------------------------ CU TRAYECTORIA ALTERNATIVA H -------------------------

\begin{UCtrayectoriaA}{H}{Uno o más campos obligatorios no fueron contestados.}
	\UCpaso Detecta uno o más campos sin contestar.
    \UCpaso Muestra el mensaje \MSGref{MSG4}{Los campos marcados con (*) son obligatorios.}
    \UCpaso[\UCactor] Cierra el mensaje presionando el botón \IUbutton{Aceptar}.
    \UCpaso Continua en el paso 5 de la trayectoria principal del \UCref{SP1-CU10}.
\end{UCtrayectoriaA}

%------------------------ CU TRAYECTORIA ALTERNATIVA I -------------------------

\begin{UCtrayectoriaA}{I}{La suma de los porcentajes es distinta a 100\%}
    \UCpaso Detecta que la suma de los porcentajes dados por el actor no cumplen con el 100\% requerido.
    \UCpaso Muestra el mensaje \MSGref{MSG23}{Los porcentajes de evaluación no cumplen con el porcentaje total obligatorio.}
    \UCpaso[\UCactor] Cierra el mensaje presionando el botón \IUbutton{Aceptar}.
    \UCpaso Continua en el paso 15 de la trayectoria principal del \UCref{SP1-CU10}.
\end{UCtrayectoriaA}

%------------------------ CU TRAYECTORIA ALTERNATIVA J -------------------------

\begin{UCtrayectoriaA}{J}{El actor aún no desea finalizar el registro}
	\UCpaso[\UCactor] Presiona el botón \IUbutton{No}.
	\UCpaso Cierra el mensaje.
	\UCpaso Continúa en el paso 16 de la trayectoria principal del \UCref{SP1-CU10}.
\end{UCtrayectoriaA}

%------------------------ CU TRAYECTORIA ALTERNATIVA K -------------------------

\begin{UCtrayectoriaA}{K}{Ocurre un error al momento de persistir los datos.}
	\UCpaso Muestra el mensaje \MSGref{MSG25}{Servicios no disponibles}.
	\UCpaso[\UCactor] Cierra el mensaje presionando el botón \IUbutton{Aceptar}.
	\UCpaso Muestra la interfaz de usuario \IUref{RRP}{Registrar Relación de Prácticas}.
\end{UCtrayectoriaA}

%------------------------ CU TRAYECTORIA ALTERNATIVA L -------------------------

\begin{UCtrayectoriaA}{L}{El sistema detecta que el número de prácticas no es consecutivo.}
	\UCpaso Muestra el mensaje \MSGref{MSG64}{El número de prácticas tiene que ser consecutivo.}
	\UCpaso[\UCactor] Cierra el mensaje presionando el botón \IUbutton{Aceptar}.
	\UCpaso Muestra la interfaz de usuario \IUref{RRP}{Registrar Relación de Prácticas}.
	\UCpaso Continua en el paso 6 de la trayectoria principal del \UCref{SP1-CU10}
\end{UCtrayectoriaA}