%REGISTRAR SISTEMA DE EVALUACIÓN: NAIBI
\begin{UseCase}{SP1-CU12}{Registrar el sistema de evaluación}{El usuario podrá especificar el sistema de evaluación por periodo, así como asignar las unidades temáticas asociadas al mismo.}
        \UCitem{Versión}{\color{Gray}1.1}
        \UCitem{Autor}{\color{Gray}Ramírez Martínez Janet Naibi.}
        \UCitem{Supervisa}{\color{Gray}Cervantes Delgadillo Mauricio.}
        \UCitem{Actor}{\hyperlink{Docente}{Docente}}
        \UCitem{Propósito}{Poder identificar qué porcentaje tiene cada unidad temática en relación con la evaluación final, y además saber qué unidades temáticas pertenecen a qué periodo.}
        \UCitem{Entradas}{Las entradas para el registro del sistema de evaluación serán:
          \begin{itemize}
            \item Periodo.
            \item Porcentaje.
          \end{itemize}
        }
        \UCitem{Origen}{Teclado y Mouse.}
        \UCitem{Salidas}{
            \begin{itemize}
                \item MSG4. Los campos marcados con (*) son obligatorios.
                \item MSG5. Registro finalizado exitosamente.
                \item MSG23. Los porcentajes de evaluación no cumplen con el porcentaje total obligatorio.
                \item MSG25. Servicios no disponibles.
                \item MSG30. ¿Está seguro que desea finalizar? Ya no podrá realizar modificaciones.
                \item MSG41. Debe llenar el Programa Sintético para realizar este registro.
                \item MSG58. Avances guardados exitosamente.
                \item MSG61. Debe llenar las Unidades Temáticas para realizar este registro.
            \end{itemize}
        }
        \UCitem{Destino}{Pantalla.}
        \UCitem{Precondiciones}{
            \begin{itemize}
                \item Se debe haber realizado previamente el registro del Programa Sintético.
                \item Se debe haber realizado previamente el registro de las Unidades Temáticas.
            \end{itemize}
        }
        \UCitem{Postcondiciones}{El sistema de evaluación queda registrado en el sistema.}
        \UCitem{Errores}{}
        \UCitem{Estado}{Revisión.}
        \UCitem{Observaciones}{}.
\end{UseCase}

%--------------------------- CU TRAYECTORIA PRINCIPAL -------------------------
\begin{UCtrayectoria}{Principal}
    \UCpaso[\UCactor] Selecciona la opción \IUbutton{Sistema de evaluación} del menú superior.
    
    \UCpaso Verifica que el programa sintético haya sido registrado. [Trayectoria A]
    
    \UCpaso Verifica que el registro de las unidades temáticas ya haya sido elaborado. [Trayectoria B]
    
    \UCpaso Muestra la interfaz de usuario \IUref{RSE}{Registrar Sistema de Evaluación}.
    
	\UCpaso[\UCactor] Selecciona el número del periodo al que pertenece la Unidad Temática.
	
	\UCpaso[\UCactor] Ingresa el porcentaje que se le atribuye a cada unidad temática ingresada con base en la regla de negocios \BRref{BR25}{La suma de los porcentajes de cada evaluación debe ser igual a 100\%}.
	
	\UCpaso[\UCactor] Termina la operación presionando el botón \IUbutton{Finalizar}. [Trayectoria C]
        
    \UCpaso Verifica que todos los campos marcados como obligatorios hayan sido completamente contestados. [Trayectoria D]
    
    \UCpaso Verifica que los porcentajes ingresados cumplan la regla de negocio \BRref{BR25}{La suma de los porcentajes de cada evaluación debe ser igual a 100\%.} [Trayectoria E]
    
    \UCpaso Muestra el mensaje \MSGref{MSG30}{¿Está seguro que desea finalizar? Ya no podrá realizar modificaciones.}

    \UCpaso[\UCactor] Confirma la operación presionando el botón \IUbutton{Si}. [Trayectoria F]
    
    \UCpaso Guarda la información del Sistema de Evaluación. [Trayectoria G]
    
    \UCpaso Muestra el mensaje \MSGref{MSG5}{Registro finalizado exitosamente}.
    
    \UCpaso[\UCactor] Cierra el mensaje presionando el botón \IUbutton{Aceptar}.
    
    \UCpaso Muestra la interfaz de usuario \IUref{RSE}{Registrar Sistema de Evaluación}.
\end{UCtrayectoria}

%------------------------ CU TRAYECTORIA ALTERNATIVA A -------------------------

\begin{UCtrayectoriaA}{A}{No se ha realizado el registro del Programa Sintético.}
    \UCpaso[\UCactor] Selecciona la opción \IUbutton{Sistema de Evaluación} del menú superior.
    \UCpaso Detecta que no se ha realizado el registro del Programa Sintético.
        \UCpaso Muestra el mensaje \MSGref{MSG41}{Debe llenar el Programa Sintético para realizar este registro}.
    \UCpaso[\UCactor] Cierra el mensaje presionando el botón \IUbutton{Aceptar}.
    \UCpaso Muestra la interfaz de usuario \IUref{RSE}{Registrar Sistema de Evaluación.}
\end{UCtrayectoriaA}

%------------------------ CU TRAYECTORIA ALTERNATIVA B -------------------------

\begin{UCtrayectoriaA}{B}{No se ha realizado el registro de las Unidades Temáticas.}
    \UCpaso[\UCactor] Selecciona la opción \IUbutton{Sistema de Evaluación} del menú superior.
    \UCpaso Detecta que no se ha realizado el registro de las Unidades Temáticas.
        \UCpaso Muestra el mensaje \MSGref{MSG61}{Debe llenar las Unidades Temáticas para realizar este registro}.
    \UCpaso[\UCactor] Cierra el mensaje presionando el botón \IUbutton{Aceptar}.
    \UCpaso Muestra la interfaz de usuario \IUref{RSE}{Registrar Sistema de Evalación.}
\end{UCtrayectoriaA}

%------------------------ CU TRAYECTORIA ALTERNATIVA C -------------------------

\begin{UCtrayectoriaA}{C}{El actor desea guardar el progreso de su registro.}
\UCpaso[\UCactor] Presiona el botón \IUbutton{Guardar}
\UCpaso Almacena la información.
\UCpaso Muestra el \MSGref{MSG58}{Avances guardados exitosamente.}
\UCpaso[\UCactor] Presiona el botón \IUbutton{Aceptar} 
\UCpaso Muestra la interfaz de usuario \IUref{RSE}{Registrar Sistema de Evaluación}.
\end{UCtrayectoriaA}

%------------------------ CU TRAYECTORIA ALTERNATIVA D -------------------------

\begin{UCtrayectoriaA}{D}{Uno o más campos obligatorios no fueron contestados.}
	\UCpaso Detecta uno o más campos sin contestar.
    \UCpaso Muestra el mensaje \MSGref{MSG4}{Los campos marcados con (*) son obligatorios.}
    \UCpaso[\UCactor] Cierra el mensaje presionando el botón \IUbutton{Aceptar}.
    \UCpaso Continua en el paso 5 de la trayectoria principal del \UCref{SP1-CU12}.
\end{UCtrayectoriaA}


%------------------------ CU TRAYECTORIA ALTERNATIVA E -------------------------

\begin{UCtrayectoriaA}{E}{La suma de los porcentajes es distinta a 100\%}
    \UCpaso Detecta que la suma de los porcentajes dados por el usuario no cumplen con el 100\% requerido.
    \UCpaso Muestra el mensaje \MSGref{MSG23}{Los porcentajes de evaluación no cumplen con el porcentaje total obligatorio.}
    \UCpaso[\UCactor] Cierra el mensaje presionando el botón \IUbutton{Aceptar}.
    \UCpaso Continua en el paso 6 de la trayectoria principal del \UCref{SP1-CU12}.
\end{UCtrayectoriaA}

%------------------------ CU TRAYECTORIA ALTERNATIVA F -------------------------

\begin{UCtrayectoriaA}{F}{El actor aún no desea finalizar el registro}
	\UCpaso[\UCactor] Presiona el botón \IUbutton{No}.
	\UCpaso Cierra el mensaje.
	\UCpaso Continúa en el paso 7 de la trayectoria principal del \UCref{SP1-CU12}.
\end{UCtrayectoriaA}

%------------------------ CU TRAYECTORIA ALTERNATIVA G -------------------------

\begin{UCtrayectoriaA}{G}{Ocurre un error al momento de persistir los datos.}
	\UCpaso Muestra el mensaje \MSGref{MSG25}{Servicios no disponibles}.
	\UCpaso[\UCactor] Cierra el mensaje presionando el botón \IUbutton{Aceptar}.
	\UCpaso Muestra la interfaz de usuario \IUref{RSE}{Registrar Sistema de Evaluación}.
\end{UCtrayectoriaA}