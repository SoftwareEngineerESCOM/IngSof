%REGISTRAR SISTEMA DE EVALUACIÓN: NAIBI
\begin{UseCase}{SP1-CU12}{Registrar el sistema de evaluación}{El usuario podrá especificar el sistema de evaluación por periodo, así como asignar las unidades temáticas asocidas al mismo.}
        \UCitem{Versión}{\color{Gray}1.0}
        \UCitem{Autor}{\color{Gray}Ramírez Martínez Janet Naibi.}
        \UCitem{Supervisa}{\color{Gray}Cervantes Delgadillo Mauricio.}
        \UCitem{Actor}{\hyperlink{Docente}{Docente}}
        \UCitem{Propósito}{Poder identificar qué porcentaje tiene cada unidad temática en relación con la evaluación final, y además saber qué unidades temáticas pertecen a qué periodo.}
        \UCitem{Entradas}{Las entradas para el registro del sistema de evaluación serán:
          \begin{itemize}
            \item Periodo.
            \item Unidad temática.
            \item Porcentaje.
          \end{itemize}
        }
        \UCitem{Origen}{Teclado y Mouse.}
        \UCitem{Salidas}{
            \begin{itemize}
                \item MSG4. Los campos marcados con (*) son obligatorios.
                \item MSG5. Registro finalizado exitosamente.
                \item MSG23. Los porcentajes de evaluación no cumplen con el porcentaje total obligatorio.
                \item MSG9. Por el momento no se puede realizar el registro.
            \end{itemize}
        }
        \UCitem{Destino}{Pantalla.}
        \UCitem{Precondiciones}{
            \begin{itemize}
                \item Se debe haber realizado previamente el registro del Programa Sintético.
                \item Se debe haber realizado previamente el registro de las Unidades Temáticas. 
                \item El catálogo Unidad\_Tematica debe tener al menos un registro.
            \end{itemize}
        }
        \UCitem{Postcondiciones}{El sistema de evaluación queda registrado en el sistema.}
        \UCitem{Errores}{}
        \UCitem{Estado}{Revisión.}
        \UCitem{Observaciones}{}.
\end{UseCase}

%--------------------------- CU TRAYECTORIA PRINCIPAL -------------------------
\begin{UCtrayectoria}{Principal}

    \UCpaso[\UCactor] Presiona el botón \IUbutton{Registrar SE} de la \IUref{SP1-IU}{Principal}.
    
    \UCpaso Extrae la información de la Unidad Aprendizaje. [Trayectoria A]
	
	\UCpaso Verifica que el catálogo ``Unidad\_Tematica'' tenga al menos un registro. [Trayectoria B]
	
	\UCpaso Muestra la \IUref{IU.05}{Registrar Sistema de Evaluación}.
	
	\UCpaso[\UCactor] Ingresa el número del periodo. [Trayectoria C]
	
	\UCpaso[\UCactor] Selecciona la Unidad Temática que se evaluará en el periodo ingresado en el paso anterior.
	
	\UCpaso[\UCactor] Selecciona el porcentaje que se le atribuye a cada unidad temática ingresada.
	
	\UCpaso[\UCactor] Termina la operación presionando el botón \IUbutton{Guardar}.
        
    \UCpaso Verifica que todos los campos marcados como obligatorios hayan sido completamente contestados. [Trayectoria D]
    
    \UCpaso Verifica que los porcentajes ingresados cumplan la regla de negocio \BRref{BR25}{La suma de los porcentajes de cada evaluación debe ser igual a 100\%.} [Trayectoria E]
    
    \UCpaso Guarda la información del Sistema de Evaluación en la base de datos.
    
    \UCpaso Muestra el mensaje \MSGref{MSG5.}{Registro finalizado exitosamente}.
    
    \UCpaso[\UCactor] Cierra el mensaje presionando el botón \IUbutton{Aceptar}.
    
    \UCpaso Muestra la interfaz de usuario \IUref{SP1-IU}{Principal}.
\end{UCtrayectoria}

%------------------------ CU TRAYECTORIA ALTERNATIVA A -------------------------

\begin{UCtrayectoriaA}{A}{No se ha realizado la elaboración del Programa Sintético.}
	\UCpaso No muestra los datos provenientes del \UCref{SP1-CU1}.
	\UCpaso Muestra el mensaje \MSGref{MSG41}{Debe llenar el Programa Sintético para realizar este registro}.
	\UCpaso[\UCactor] Cierra el mensaje presionando el botón \IUbutton{Aceptar}.
	\UCpaso Muestra la interfaz de usuario \IUref{SP1-IU}{Principal}.
\end{UCtrayectoriaA}

%------------------------ CU TRAYECTORIA ALTERNATIVA B -------------------------

\begin{UCtrayectoriaA}{B}{El catálogo Unidad\_Tematica está vacio.}
	\UCpaso No encuentra información en el catálogo Unidad\_Tematica.
    \UCpaso Muestra el mensaje \MSGref{MSG9.}{Por el momento no se puede realizar el registro.}
    \UCpaso[\UCactor] Cierra el mensaje presionando el botón \IUbutton{Aceptar}.
	\UCpaso Muestra la interfaz de usuario \IUref{SP1-IU}{Principal}.
\end{UCtrayectoriaA}

%------------------------ CU TRAYECTORIA ALTERNATIVA C -------------------------

\begin{UCtrayectoriaA}{C}{Desea registrar más de un periodo para el Sistema de Evaluación.}
	\UCpaso[\UCactor] Presiona el botón \IUbutton{Agregar Periodo}.
	\UCpaso Despliega los campos ``Periodo'', y ``Unidad Temática''.
	\UCpaso Continua en el paso 5 de la trayectoria principal de \UCref{SP1-CU12}
\end{UCtrayectoriaA}

%------------------------ CU TRAYECTORIA ALTERNATIVA D -------------------------

\begin{UCtrayectoriaA}{D}{Uno o más campos obligatorios no fueron contestados.}
	\UCpaso Detecta uno o más campos sin contestar.
    \UCpaso Muestra el mensaje \MSGref{MSG4.}{Los campos marcados con (*) son obligatorios.}
    \UCpaso[\UCactor] Cierra el mensaje presionando el botón \IUbutton{Aceptar}.
    \UCpaso Continua en el paso 5 de la trayectoria principal del \UCref{SP1-CU4}.
\end{UCtrayectoriaA}


%------------------------ CU TRAYECTORIA ALTERNATIVA E -------------------------

\begin{UCtrayectoriaA}{E}{La suma de los porcentajes es distinta a 100\%}
    \UCpaso Detecta que la suma de los porcentajes dados por el usuario no cumplen con el 100\% requerido.
    \UCpaso Muestra el mensaje \MSGref{MSG23.}{Los porcentajes de evaluación no cumplen con el porcentaje total obligatorio.}
    \UCpaso[\UCactor] Cierra el mensaje presionando el botón \IUbutton{Aceptar}.
    \UCpaso Continua en el paso 7 de la trayectoria principal del \UCref{SP1-CU12}.
\end{UCtrayectoriaA}


