%REGISTRAR BIBLIOGRAFÍA: NAIBI
\begin{UseCase}{SP1-CU13}{Registrar bibliografía de la UA}{El usuario podrá registrar una o más bibliografías en el sistema.}
        \UCitem{Versión}{\color{Gray}1.2}
        \UCitem{Autor}{\color{Gray}Ramírez Martínez Janet Naibi}
        \UCitem{Supervisa}{\color{Gray}Cervantes Delgadillo Mauricio}
        \UCitem{Actor}{\hyperlink{Docente}{Docente}}
        \UCitem{Propósito}{Registrar la bibliografía sugerida a distintas unidades de aprendizaje a través de un identificador único.}
        \UCitem{Entradas}{Las entradas para el registro de una bibliografía serán:
          \begin{itemize}
            \item Autor.
            \item ISBN.
            \item Título del libro.
            \item Año de publicación.
            \item Edición.
            \item Ciudad de publicación.
            \item Editorial.
          \end{itemize}
        }
        \UCitem{Origen}{Teclado y Mouse}
        \UCitem{Salidas}{
            \begin{itemize}
                \item MSG4. Los campos marcados con (*) son obligatorios.
                \item MSG5. Registro finalizado exitosamente.
                \item MSG6. ¿Seguro que desea cancelar el registro?
                \item MSG8. El libro con el ISBN "Número de ISBN" ya existe.
                \item MSG9. Por el momento no se puede realizar el registro.
            \end{itemize}
        }
        \UCitem{Destino}{Pantalla.}
        \UCitem{Precondiciones}{
            \begin{itemize}
                \item La bibliografía que se desea registrar, no debe existir previamente en el sistema.
                \item Los catálogos ``Tipo\_Bibliografía'' y ``Lugar\_Publicacion'' no deben estar vacios.
            \end{itemize}}
        \UCitem{Postcondiciones}{La bibliografía queda registrada en el sistema.}
        \UCitem{Errores}{}
        \UCitem{Estado}{Revisión.}
        \UCitem{Observaciones}{Si el autor no se encuentra en la lista de autores mostrados por esta interfaz, se tendrá que presionar el \IUbutton{Registrar autor} para ingresar el nuevo registro a la base de datos.}
\end{UseCase}

%--------------------------- CU TRAYECTORIA PRINCIPAL -------------------------
\begin{UCtrayectoria}{Principal}

    \UCpaso[\UCactor] Presiona el botón \IUbutton{Registrar bibliografía}.

    \UCpaso Busca en la base de datos la información de los catálogos ``Tipo\_Bibliografía'' y ``Lugar\_Publicacion''. [Trayectoria A]

    \UCpaso Muestra la interfaz de usuario \IUref{IU6}{Registrar bibliografía}.

    \UCpaso[\UCactor] Selecciona el(los) autor(es) de la lista de autores disponibles con base en la regla de negocio \BRref{BR23}{Un libro con más de 3 autores es sólo citado por los primeros 3.}. [Trayectoria B]

    \UCpaso[\UCactor] Selecciona el año de publicación del libro que se está registrando.

    \UCpaso[\UCactor] Ingresa el título del libro que desea registrar.

    \UCpaso[\UCactor] Selecciona el lugar de publicación del libro que está siendo registrado.

    \UCpaso[\UCactor] Selecciona la editorial del libro que está siendo registrado.

    \UCpaso[\UCactor] Ingresa el ISBN asociado al libro que está registrando con base en las reglas de negocio \BRref{BR21}{Un libro tiene un ISBN único.} y \BRref{BR22}{Longitud del ISBN}.

    \UCpaso[\UCactor] Selecciona el tipo de la bibliografía.

    \UCpaso[\UCactor] Termina la operación presionando el botón \IUbutton{Continuar}.

    \UCpaso Verifica que todos los campos marcados como obligatorios hayan sido completamente contestados con base en la regla de negocio \BRref{BR13}{Todos los datos solicitados son obligatorios}. [Trayectoria C]

    \UCpaso Valida que no exista un libro con el mismo ISBN. [Trayectoria D]

    \UCpaso Guarda la información de la bibliografía en la base de datos.

    \UCpaso El sistema muestra el mensaje \MSGref{MSG2.}{Registro finalizado exitosamente.}

    \UCpaso[\UCactor] Cierra el mensaje presionando el botón \IUbutton{Aceptar}.

    \UCpaso Muestra la interfaz de usuario \IUref{IU8}{Registrar Perfil Docente}.
\end{UCtrayectoria}

%------------------------ CU TRAYECTORIA ALTERNATIVA A -------------------------

\begin{UCtrayectoriaA}{A}{Los catálogos están vacios.}
    \UCpaso No encuentra información en los catálogos ``Tipo\_Bibliografía'' y ``Lugar\_Publicacion''.
    \UCpaso El sistema muestra el mensaje \MSGref{MSG9.}{Por el momento no se puede realizar el registro.}
    \UCpaso[\UCactor] Cierra el mensaje presionando el botón \IUbutton{Aceptar}.
    %\UCpaso Continua en el paso 1 de la trayectoria principal del \UCref{CU1}.
\end{UCtrayectoriaA}

%------------------------ CU TRAYECTORIA ALTERNATIVA B -------------------------

\begin{UCtrayectoriaA}{B}{No existe el(los) autor(es) requerido(s) en la lista de autores disponibles.}
    \UCpaso[\UCactor] No encuentra al autor requerido.
    \UCpaso[\UCactor] Presiona el botón \IUbutton{Agregar Autor} de la \IUref{IU6}{Registrar bibliografía.}
    \UCpaso Muestra el modal \IUref{IU6.1}{Registrar Autor}.
    \UCpaso Continua en el paso 1 de la trayectoria principal del \UCref{SP1-CU14}.
\end{UCtrayectoriaA}

%------------------------ CU TRAYECTORIA ALTERNATIVA C -------------------------

\begin{UCtrayectoriaA}{C}{Uno o más campos obligatorios no fueron contestados.}
    \UCpaso Detecta uno o más campos sin contestar.
    \UCpaso Muestra el mensaje \MSGref{MSG4.}{Los campos marcados con (*) son obligatorios.}
    \UCpaso[\UCactor] Cierra el mensaje presionando el botón \IUbutton{Aceptar}.
    \UCpaso Continua en el paso 4 de la trayectoria principal del \UCref{SP1-CU1}.
\end{UCtrayectoriaA}

%------------------------ CU TRAYECTORIA ALTERNARIVA D -------------------------

\begin{UCtrayectoriaA}{D}{El ISBN que se intenta registrar ya existe en el sistema.}
    \UCpaso Encuentra un libro registrado con el mismo ISBN que el libro que se está registrando.
    \UCpaso Muestra el mensaje \MSGref{MSG8.}{El libro con el ISBN "Número de ISBN" ya existe.}
    \UCpaso[\UCactor] Cierra el mensaje presionando el botón \IUbutton{Aceptar}.
    \UCpaso Continua en el paso 4 de la trayectoria principal del \UCref{SP1-CU1}.
\end{UCtrayectoriaA}