%REGISTRAR BIBLIOGRAFÍA: NAIBI
\begin{UseCase}{SP1-CU13}{Registrar bibliografía de la UA}{El usuario podrá registrar una o más bibliografías en el sistema.}
        \UCitem{Versión}{\color{Gray}1.2}
        \UCitem{Autor}{\color{Gray}Ramírez Martínez Janet Naibi}
        \UCitem{Supervisa}{\color{Gray}Cervantes Delgadillo Mauricio}
        \UCitem{Actor}{\hyperlink{Docente}{Docente}}
        \UCitem{Propósito}{Registrar la bibliografía sugerida a distintas unidades de aprendizaje a través de un identificador único.}
        \UCitem{Entradas}{Las entradas para el registro de una bibliografía serán:
          \begin{itemize}
            \item Autor.
            \item Año de publicación.
            \item Título del libro.
            \item Lugar de publicación.
            \item Editorial.
            \item ISBN.
            \item Tipo de bibliografía.
            \item Unidad temática en la que se puede consultar esa bibliografía.
          \end{itemize}
        }
        \UCitem{Origen}{Teclado y Mouse}
        \UCitem{Salidas}{
            \begin{itemize}
                \item MSG4. Los campos marcados con (*) son obligatorios.
                \item MSG5. Registro finalizado exitosamente.
                \item MSG8. Ya existe un libro con ese ISBN.
                \item MSG25. Servicios no disponibles.
                \item MSG58. Avances guardados exitosamente.
            \end{itemize}
        }
        \UCitem{Destino}{Pantalla.}
        \UCitem{Precondiciones}{
            \begin{itemize}
                \item La bibliografía que se desea registrar, no debe existir previamente en el sistema.
                \item Los catálogos ``Tipo\_Bibliografía'' y ``Lugar\_Publicacion'' no deben estar vacíos.
            \end{itemize}}
        \UCitem{Postcondiciones}{La bibliografía queda registrada en el sistema.}
        \UCitem{Errores}{}
        \UCitem{Estado}{Revisión.}
        \UCitem{Observaciones}{Si el autor no se encuentra en la lista de autores mostrados por esta interfaz, se tendrá que presionar el \IUbutton{Agregar autor} para ingresar el nuevo registro al sistema.}
\end{UseCase}

%--------------------------- CU TRAYECTORIA PRINCIPAL -------------------------
\begin{UCtrayectoria}{Principal}

    \UCpaso[\UCactor] Selecciona la opción \IUbutton{Registrar bibliografía}.

    \UCpaso Busca en la base de datos la información de los catálogos ``Tipo\_Bibliografía'' y ``Lugar\_Publicacion''. [Trayectoria A]

    \UCpaso Muestra la interfaz de usuario \IUref{RB}{Registrar Bibliografía}.

    \UCpaso[\UCactor] Selecciona el(los) autor(es) de la lista de autores disponibles con base en la regla de negocio \BRref{BR23}{Un libro con más de 3 autores es sólo citado por los primeros 3.}. [Trayectoria B]
    
    \UCpaso[\UCactor] Selecciona el año de publicación del libro que se está registrando.
    
    \UCpaso[\UCactor] Ingresa el título del libro que desea registrar.
    
    \UCpaso[\UCactor] Selecciona el lugar de publicación del libro que está siendo registrado.
    
    \UCpaso[\UCactor] Ingresa la editorial del libro que está siendo registrado.

    \UCpaso[\UCactor] Ingresa el ISBN asociado al libro que está registrando con base en las reglas de negocio \BRref{BR21}{Un libro tiene un ISBN único.} y \BRref{BR22}{Longitud del ISBN}.

    \UCpaso[\UCactor] Selecciona el tipo de la bibliografía.
    
    \UCpaso[\UCactor] Selecciona la unidad temática a la que pertenece la bibliografía.

    \UCpaso[\UCactor] Selecciona si la bibliografía es clásica o no.
    
    \UCpaso[\UCactor] Termina la operación presionando el botón \IUbutton{Finalizar}. [Trayectoria C] [Trayectoria D]

    \UCpaso Verifica que todos los campos marcados como obligatorios hayan sido completamente contestados con base en la regla de negocio \BRref{BR13}{Todos los datos solicitados son obligatorios}. [Trayectoria E]

    \UCpaso Valida que no exista un libro con el mismo ISBN. [Trayectoria F]

    \UCpaso Guarda la información de la bibliografía. [Trayectoria G]

    \UCpaso El sistema muestra el mensaje \MSGref{MSG5}{Registro finalizado exitosamente.}

    \UCpaso[\UCactor] Cierra el mensaje presionando el botón \IUbutton{Aceptar}.

    \UCpaso Muestra la interfaz de usuario \IUref{RB}{Registrar bibliografía}.
\end{UCtrayectoria}

%------------------------ CU TRAYECTORIA ALTERNATIVA A -------------------------

\begin{UCtrayectoriaA}{A}{Los catálogos están vacíos.}
    \UCpaso No encuentra información en los catálogos ``Tipo\_Bibliografía'' y ``Lugar\_Publicacion''.
    \UCpaso El sistema muestra el mensaje \MSGref{MSG25}{Servicios no disponibles}.
    \UCpaso[\UCactor] Cierra el mensaje presionando el botón \IUbutton{Aceptar}.
    \UCpaso Muestra la interfaz de usuario \IUref{RB}{Registrar Bibliografía}.
\end{UCtrayectoriaA}

%------------------------ CU TRAYECTORIA ALTERNATIVA B -------------------------

\begin{UCtrayectoriaA}{B}{No existe el(los) autor(es) requerido(s) en la lista de autores disponibles.}
    \UCpaso[\UCactor] No encuentra al autor requerido.
    \UCpaso[\UCactor] Presiona el botón \IUbutton{Agregar Autor} de la interfaz de usuario \IUref{RB}{Registrar Bibliografía.}
    \UCpaso Muestra el modal \IUref{ATHR}{Registrar Autor}.
    \UCpaso Continua en el paso 1 de la trayectoria principal del \UCref{SP1-CU14}.
\end{UCtrayectoriaA}

%------------------------ CU TRAYECTORIA ALTERNATIVA C -------------------------

\begin{UCtrayectoriaA}{C}{El actor desea guardar el progreso de su registro.}
    \UCpaso[\UCactor] Presiona el botón \IUbutton{Guardar}
    \UCpaso Almacena la información.
    \UCpaso Muestra el \MSGref{MSG58}{Avances guardados exitosamente.}
    \UCpaso[\UCactor] Presiona el botón \IUbutton{Aceptar} 
    \UCpaso Muestra la interfaz de usuario \IUref{RB}{Registrar Bibliografía}.
\end{UCtrayectoriaA}

%------------------------ CU TRAYECTORIA ALTERNATIVA D -------------------------

\begin{UCtrayectoriaA}{D}{El actor quiere agregar otra bibliografía a la Unidad de Aprendizaje.}
    \UCpaso[\UCactor] Presiona el botón \IUbutton{Limpiar}.
    \UCpaso Limpia todo el formulario.
    \UCpaso Continua en el paso 4 de la trayectoria principal del \UCref{SP1-CU13}
\end{UCtrayectoriaA}

%------------------------ CU TRAYECTORIA ALTERNATIVA E -------------------------

\begin{UCtrayectoriaA}{E}{Uno o más campos obligatorios no fueron contestados.}
    \UCpaso Detecta uno o más campos sin contestar.
    \UCpaso Muestra el mensaje \MSGref{MSG4.}{Los campos marcados con (*) son obligatorios.}
    \UCpaso[\UCactor] Cierra el mensaje presionando el botón \IUbutton{Aceptar}.
    \UCpaso Continua en el paso 4 de la trayectoria principal del \UCref{SP1-CU13}.
\end{UCtrayectoriaA}

%------------------------ CU TRAYECTORIA ALTERNARIVA F -------------------------

\begin{UCtrayectoriaA}{F}{El ISBN que se intenta registrar ya existe en el sistema.}
    \UCpaso Encuentra un libro registrado con el mismo ISBN que el libro que se está registrando.
    \UCpaso Muestra el mensaje \MSGref{MSG8.}{Ya existe un libro con ese ISBN.}
    \UCpaso[\UCactor] Cierra el mensaje presionando el botón \IUbutton{Aceptar}.
    \UCpaso Continua en el paso 9 de la trayectoria principal del \UCref{SP1-CU13}.
\end{UCtrayectoriaA}

%------------------------ CU TRAYECTORIA ALTERNATIVA G -------------------------

\begin{UCtrayectoriaA}{G}{Ocurre un error al momento de persistir los datos.}
	\UCpaso Muestra el mensaje \MSGref{MSG25}{Servicios no disponibles}.
	\UCpaso[\UCactor] Cierra el mensaje presionando el botón \IUbutton{Aceptar}.
	\UCpaso Muestra la interfaz de usuario \IUref{RB}{Registrar Bibliografía}.
\end{UCtrayectoriaA}