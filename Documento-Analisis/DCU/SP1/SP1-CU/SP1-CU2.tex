%REGISTRAR CONTENIDOS DE LA UA: AIKO
\begin{UseCase}{SP1-CU2}{Registrar los contenidos de la Unidad de Aprendizaje}{El actor podrá registrar contenidos correspondientes a una Unidad de Aprendizaje.}
    \UCitem{Versión}{\color{Gray}1.1}
    \UCitem{Autor}{\color{Gray}López Rivera Aiko Dallane}
    \UCitem{Supervisa}{\color{Gray}Ramírez Martínez Janet Naibi}
    \UCitem{Actor}{\hyperlink{Docente}{Docente}}
    \UCitem{Propósito}{Servir como marco de referencia para el registro de diversos contenidos a una Unidad de Aprendizaje.}
    \UCitem{Entradas}{Las entradas para el registro de contenidos de la Unidad de Aprendizaje serán:
          \begin{itemize}
            \item Número de Contenidos.
            \item Nombre del Contenido.
          \end{itemize}
        }
    \UCitem{Origen}{Teclado y Mouse.}
    \UCitem{Salidas}{
        \begin{itemize}
            \item \MSGref{MSG4}{Los campos marcados con (*) son obligatorios}.
            \item \MSGref{MSG5}{Registro finalizado exitosamente}.
            \item \MSGref{MSG25}{Servicios no disponibles}.
            \item \MSGref{MSG29}{¿Está seguro que desea cancelar? Se perderán todos los avances sin guardar.}
            \item \MSGref{MSG59}{Número de contenidos excedido, máximo 30}.
        \end{itemize}}
    \UCitem{Destino}{Pantalla.}
    \UCitem{Precondiciones}{El actor debe haber presionado el botón \IUbutton{Registrar contenidos \BtnModal} de la \IUref{RPS}{Registrar Programa Sintético.}}
    \UCitem{Postcondiciones}{Los contenidos quedan registrados en el sistema.}
    \UCitem{Errores}{}
    \UCitem{Estado}{Revisión.}
    \UCitem{Observaciones}{}
\end{UseCase}

%--------------------------- CU TRAYECTORIA PRINCIPAL -------------------------
\begin{UCtrayectoria}{Principal}

    \UCpaso[\UCactor] Selecciona el número de contenidos a registrar con base en la regla de negocio \BRref{BR42}{El número máximo de contenidos que puede tener una Unidad de Aprendizaje es de 30}. [Trayectoria A]
    \UCpaso Despliega los campos para ingresar el nombre del  contenido de acuerdo al número previamente seleccionado.
    \UCpaso[\UCactor] Ingresa el nombre de cada contenido. [Trayectoria B]
    \UCpaso[\UCactor] Termina la operación presionando el botón \IUbutton{Guardar}. [Trayectoria C]
    \UCpaso Verifica que todos los campos marcados como obligatorios hayan sido llenados con base en la regla de negocio \BRref{BR39}{Todos los campos marcados con (*) son obligatorios}.[Trayectoria D]
    \UCpaso Guarda la información de los contenidos. [Trayectoria E]
    \UCpaso El sistema muestra el mensaje \MSGref{MSG5}{Registro finalizado exitosamente}.
    \UCpaso[\UCactor] Cierra el mensaje presionando el botón \IUbutton{Aceptar}.
    \UCpaso Muestra la interfaz de usuario \IUref{RPS}{Registrar Programa Sintético}.
    \UCpaso Continua en el paso 7 de la trayectoria principal del \UCref{SP1-CU1}.
\end{UCtrayectoria}

%------------------------ CU TRAYECTORIA ALTERNATIVA A -------------------------

\begin{UCtrayectoriaA}{A}{El actor selecciona más contenidos de los permitidos.}
    \UCpaso[\UCactor] Ingresa un número de contenidos mayor a los 30 registros permitidos según la regla de negocio \BRref{BR42}{El número máximo de contenidos que puede tener una Unidad de Aprendizaje es de 30}. 
    \UCpaso Muestra el mensaje \MSGref{MSG59}{Número de contenidos excedido, máximo 30}
    \UCpaso Continua en el paso 1 de la trayectoria principal del \UCref{SP1-CU2}.
\end{UCtrayectoriaA}

%------------------------ CU TRAYECTORIA ALTERNATIVA B -------------------------

\begin{UCtrayectoriaA}{B}{El actor desea modificar el número de contenidos.}
    \UCpaso[\UCactor] Modifica el número de contenidos que desea registrar. 
    \UCpaso Despliega los campos para agregar el nombre del contenido de acuerdo al número ingresado en el paso anterior.
    \UCpaso Continua en el paso 3 de la trayectoria principal del \UCref{SP1-CU2}.
\end{UCtrayectoriaA}

%------------------------ CU TRAYECTORIA ALTERNARTIVA C -------------------------

\begin{UCtrayectoriaA}{C}{El actor desea cancelar el registro de contenidos.}
    \UCpaso[\UCactor] Presiona el botón \IUbutton{Cancelar}
    \UCpaso Muestra el \MSGref{MSG29}{¿Está seguro que desea cancelar? Se perderán todos los avances sin guardar.}
    \UCpaso[\UCactor] Confirma la operación presionando el botón \IUbutton{Si} [Trayectoria C.1]
    \UCpaso Cierra el modal.
    \UCpaso Muestra la interfaz de usuario \IUref{RPS}{Registrar Programa Sintético}
\end{UCtrayectoriaA}

%------------------------ CU TRAYECTORIA ALTERNARTIVA C.1 -------------------------

\begin{UCtrayectoriaA}{C.1}{El actor no desea cancelar el registro de Contenidos.}

    \UCpaso[\UCactor] Presiona el botón \IUbutton{No}
    \UCpaso Continua en el paso 4 de la trayectoria principal del \UCref{SP1-CU2}.

\end{UCtrayectoriaA}

%------------------------ CU TRAYECTORIA ALTERNARTIVA D -------------------------

\begin{UCtrayectoriaA}{D}{Uno o más campos obligatorios no fueron contestados.}
    \UCpaso Detecta uno o más campos sin contestar.
    \UCpaso Muestra el mensaje \MSGref{MSG4}{Los campos marcados con (*) son obligatorios}.
    \UCpaso[\UCactor] Cierra el mensaje presionando el botón \IUbutton{Aceptar}.
    \UCpaso Continua en el paso 3 de la trayectoria principal del \UCref{SP1-CU2}.
\end{UCtrayectoriaA}

%------------------------ CU TRAYECTORIA ALTERNARTIVA E -------------------------

\begin{UCtrayectoriaA}{E}{Ocurre un error al momento de persistir los datos.}
    \item Muestra el mensaje \MSGref{MSG25}{Servicios no disponibles}.  
    \item Cierra el mensaje presionando el botón \IUbutton{Aceptar}.
    \item Muestra la \IUref{RC}{Registrar contenido}
\end{UCtrayectoriaA}