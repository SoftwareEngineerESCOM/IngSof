%REGISTRAR CONTENIDOS DE LA UA: AIKO
\begin{UseCase}{SP1-CU2}{Registrar los contenidos de la Unidad de Aprendizaje}{El usuario podrá registrar contenidos correspondientes a una Unidad de Aprendizaje.}
    \UCitem{Versión}{\color{Gray}1.0}
    \UCitem{Autor}{\color{Gray}López Rivera Aiko Dallane}
    \UCitem{Supervisa}{\color{Gray}Ramírez Martínez Janet Naibi}
    \UCitem{Actor}{\hyperlink{Usuario}{Docente}}
    \UCitem{Propósito}{Servir como marco de referencia para el registro de diversos contenidos a una Unidad de Aprendizaje.}
    \UCitem{Entradas}{Las entradas para el registro de contenidos de la Unidad de Aprendizaje serán:
          \begin{itemize}
            \item Número de Contenidos.
            \item Nombre del Contenido.
          \end{itemize}
        }
    \UCitem{Origen}{Teclado y Mouse.}
    \UCitem{Salidas}{
     \begin{itemize}
               \item \MSGref{MSG5}{Registro finalizado exitosamente.}
	       \item \MSGref{MSG25}{Servicios no disponibles por el momento.}
	       \item \MSGref{MSG29}{ ¿Está seguro que desea cancelar? Se perderán todos los avances sin guardar.}
	       \item \MSGref{MSG35}{Escribe información válida}
	       \item \MSGref{MSG44}{Este campo es requerido}
     \end{itemize}

    }
    \UCitem{Destino}{Pantalla.}
    \UCitem{Precondiciones}{}
    \UCitem{Postcondiciones}{Los contenidos quedan registrados en el sistema.}
    \UCitem{Errores}{
    		\begin{itemize}
			\item E1. Hubo un problema al conectarse con la base de datos.
		\end{itemize}
    }
    \UCitem{Estado}{Revisión.}
    \UCitem{Observaciones}{}
\end{UseCase}

%--------------------------- CU TRAYECTORIA PRINCIPAL -------------------------
\begin{UCtrayectoria}{Principal}
    \UCpaso[\UCactor] Presiona el botón \BtnModal{Registrar Contenidos} de la interfaz de Usuario \IUref{IU.01}{Registrar el Programa Sintético}.
    \UCpaso Muestra la interfaz de usuario \IUref{MIU1.01}{Registrar Contenidos de la Unidad de Aprendizaje}.
    \UCpaso[\UCactor] Selecciona el número de contenidos a registrar.
    \UCpaso Despliega un listado para agregar nombre al contenido de acuerdo al número previamente seleccionado. [Trayectoria A]
    \UCpaso[\UCactor] Ingresa el nombre de cada contenido, el sistema verifica conforme al modelo de datos, la \BRref{BR38}{Verificación de formularios al momento} y la \BRref{BR39}{Todos los campos marcados con (*) son obligatorios}.[Trayectoria C]
	\UCpaso[\UCactor] Termina la operación presionando el botón \IUbutton{Guardar}. [Trayectoria B] [Trayectoria B.1]
	\UCpaso Verifica que se cumpla con el modelos de datos. [Trayectoria C]
	\UCpaso Persiste los datos ingresados. [Trayectoria D]
	\UCpaso El sistema muestra el mensaje \MSGref{MSG5}{Registro finalizado exitosamente}.
	\UCpaso[\UCactor] Cierra el mensaje presionando el botón \IUbutton{Aceptar}.
	\UCpaso Muestra la interfaz de usuario \IUref{UI.01}{Registrar el Programa Sintético}.
\end{UCtrayectoria}

%------------------------ CU TRAYECTORIA ALTERNARTIVA A -------------------------

\begin{UCtrayectoriaA}{A}{El usuario desea agregar más contenidos.}
    \UCpaso[\UCactor] Selecciona el número de contenidos a registrar
    \UCpaso Despliega un listado para agregar nombre al contenido de acuerdo al número previamente seleccionado.
    \UCpaso Continua en el paso 4 de la trayectoria principal del \UCref{SP1-CU2}.

\end{UCtrayectoriaA}

%------------------------ CU TRAYECTORIA ALTERNARIVA B -------------------------
\begin{UCtrayectoriaA}{B}{El usuario presiona el botón \IUbutton{Cancelar}.}
	\UCpaso Muestra el mensaje \MSGref{MSG29}{¿Está seguro que desea cancelar? Se perderán todos los avances sin guardar}.
	\UCpaso[\UCactor] Confirma la operación presionando el botón \IUbutton{Si}.
	\UCpaso Muestra la interfaz de usuario \IUref{UI.01}{Registrar el programa Sintético}
\end{UCtrayectoriaA}

%------------------------ TRAYECTORIA ALTERNARIVA B.1 -------------------------
\begin{UCtrayectoriaA}{B.1}{El usuario no desea cancelar el Registro de COntedidos \IUbutton{Cancelar}.}
	\UCpaso[\UCactor] Presiona el botón \IUbutton{Cancelar}
	\UCpaso Muestra el mensaje \MSGref{MSG29}{¿Está seguro que desea cancelar? Se perderán todos los avances sin guardar}.
	\UCpaso[\UCactor] Presiona el botón \IUbutton{No}.
	\UCpaso Cierra el mensaje.
	\UCpaso Continúa en el paso 4 de la trayectoria principal del \UCref{SP1-CU2}.
\end{UCtrayectoriaA}

%------------------------ TRAYECTORIA ALTERNARIVA C -------------------------
\begin{UCtrayectoriaA}{C}{El sistema detecta uno o más campos sin contestar.}
	\UCpaso Muestra el mensaje \MSGref{MSG44}{Este campo es requerido} debajo de cada campo obligatorio sin contestar.
	\UCpaso Continúa en el paso 4 de la trayectoria principal del \UCref{SP1-CU2}.
\end{UCtrayectoriaA}

%------------------------ TRAYECTORIA ALTERNARIVA D -------------------------
\begin{UCtrayectoriaA}{E}{Ocurre un error al momento de persistir los datos.}
	\UCpaso Muestra el mensaje \MSGref{MSG25}{Servicios no disponibles por el momento}.
	\UCpaso[\UCactor] Cierra el mensaje presionando el botón \IUbutton{Aceptar}.
	\UCpaso Muestra la interfaz de usuario \IUref{UI.01}{Registrar el programa Sintético}.
\end{UCtrayectoriaA}

