%REGISTRAR EVALUACIÓN Y ACREDITACIÓN DE LA UA: NAIBI
\begin{UseCase}{SP1-CU3}{Registrar la evaluación y acreditación de la UA}{El usuario podrá registrar la forma de evaluación con sus respectivos porcentajes y además podrá seleccionar la forma de acreditación de la UA.}
        \UCitem{Versión}{\color{Gray}1.2}
        \UCitem{Autor}{\color{Gray}Ramírez Martínez Janet Naibi.}
        \UCitem{Supervisa}{\color{Gray}Cervantes Delgadillo Mauricio.}
        \UCitem{Actor}{\hyperlink{Docente}{Docente}}
        \UCitem{Propósito}{Registrar las evaluaciones que deberá tener la Unidad de Aprendizaje para poder tener un marco de referencia al evaluar al alumno, y además poder conocer el medio para poder acreditar dicha Unidad de Aprendizaje.}
        \UCitem{Entradas}{Las entradas para el registro de las evaluaciones serán:
          \begin{itemize}
            \item Nombre de la forma de acreditación (opcional).
            \item Nombre del criterio de evaluación.
            \item Porcentaje del criterio ingresado en el punto anterior.
          \end{itemize}
        }
        \UCitem{Origen}{Teclado y Mouse.}
        \UCitem{Salidas}{
            \begin{itemize}
                \item MSG4. Los campos marcados con (*) son obligatorios.
                \item MSG5. Registro finalizado exitosamente.
                \item MSG23. Los porcentajes de evaluación no cumplen con el porcentaje total obligatorio.
                \item MSG25. Servicios no disponibles.
                \item MSG29. ¿Está seguro que desea cancelar? Se perderán todos los avances sin guardar.
            \end{itemize}
        }
        \UCitem{Destino}{Pantalla.}
        \UCitem{Precondiciones}{
            \begin{itemize}
                \item Debe existir al menos un criterio de evaluación para poder asignarle un porcentaje.
                \item El catálogo ``Acreditacion'' debe tener al menos un registro.
            \end{itemize}
        }
        \UCitem{Postcondiciones}{La evaluación y acreditación de la Unidad de Aprendizaje quedan registradas en el sistema.}
        \UCitem{Errores}{}
        \UCitem{Estado}{Revisión.}
        \UCitem{Observaciones}{
            \begin{itemize}
                \item Se pueden agregar tantos criterios como el actor desee siempre y cuando el porcentaje final total de la Unidad de Aprendizaje sea siempre de 100\%.
                \item Si no existe la forma de acreditación que el actor necesita, podrá registrarla presionando el botón \IUbutton{Registrar Acreditación}.
            \end{itemize}}
\end{UseCase}

%--------------------------- CU TRAYECTORIA PRINCIPAL -------------------------
\begin{UCtrayectoria}{Principal}

    \UCpaso[\UCactor] Selecciona la(s) forma(s) de acreditación de la lista de acreditaciones disponibles. [Trayectoria A]

    \UCpaso[\UCactor] Ingresa el criterio de evaluación con base en la regla \BRref{BR24}{Debe existir al menos un criterio de evaluación para una Unidad de Aprendizaje}. [Trayectoria B]

    \UCpaso[\UCactor] Ingresa el porcentaje de la evaluación registrada anteriormente con base en la regla \BRref{BR25}{La suma de los porcentajes de cada evaluación debe ser igual a 100\%}.

    \UCpaso[\UCactor] Termina la operación presionando el botón \IUbutton{Guardar}. [Trayectoria C]

    \UCpaso Verifica que todos los campos marcados como obligatorios hayan sido completamente llenados. [Trayectoria D]

    \UCpaso Valida que la suma de los porcentajes sea igual a 100\% según la regla de negocio \BRref{BR25}{La suma de los porcentajes de cada evaluación debe ser igual a 100\%}. [Trayectoria E]

    \UCpaso Guarda la información de la evaluación y acreditación. [Trayectoria F]

    \UCpaso El sistema muestra el mensaje \MSGref{MSG5.}{Registro finalizado exitosamente}.

    \UCpaso[\UCactor] Cierra el mensaje presionando el botón \IUbutton{Aceptar}.

    \UCpaso Muestra la interfaz de usuario \IUref{RPS}{Registrar Programa Sintético}.
    
    \UCpaso Continua en el paso 9 de la trayectoria principal del \UCref{SP1-CU1}.
\end{UCtrayectoria}

%------------------------ CU TRAYECTORIA ALTERNATIVA A -------------------------

\begin{UCtrayectoriaA}{A}{No existe la forma de acreditación requerida en la lista de acreditaciones disponibles.}
    \UCpaso[\UCactor] No encuentra la forma de acreditación requerida.
    \UCpaso[\UCactor] Presiona el botón \IUbutton{Registrar Acreditación}.
    \UCpaso Muestra el modal \IUref{AAT}{Registrar Tipo de Acreditación}.
    \UCpaso[\UCactor] Ingresa la forma de acreditación requerida. [Trayectoria A.1]
    \UCpaso[\UCactor] Termina la operación presionando el botón \IUbutton{Aceptar}. [Trayectoria A.2] 
    \UCpaso Guarda la nueva forma de acreditación.
    \UCpaso Cierra el modal \IUref{AAT}{Registrar Tipo de Acreditación}.
    \UCpaso Continua en el paso 1 de la trayectoria principal del \UCref{SP1-CU3}.
\end{UCtrayectoriaA}

%------------------------ CU TRAYECTORIA ALTERNATIVA A.1 -------------------------

\begin{UCtrayectoriaA}{A.1}{El usuario no ingresa ninguna acreditación nueva.}
    \UCpaso[\UCactor] No ha ingresado un nuevo tipo de acreditación.
    \UCpaso Bloquea el botón \IUbutton{Aceptar}
    \UCpaso Marca con color rojo el campo para que el usuario ingrese un dato.
    \UCpaso Continua en el paso 3 de la trayectoria alternativa B del \UCref{SP1-CU3}.
\end{UCtrayectoriaA}

%------------------------ CU TRAYECTORIA ALTERNATIVA A.2 -------------------------

\begin{UCtrayectoriaA}{A.2}{El usuario quiere cancelar el registro de la nueva acreditación.}
    \UCpaso[\UCactor] Presiona el botón \IUbutton{Cancelar}.
    \UCpaso Muestra el mensaje \MSGref{MSG29}{¿Está seguro que desea cancelar? Se perderán todos los avances sin guardar}.
    \UCpaso[\UCactor] Confirma la operación presionando el botón \IUbutton{Si}. [Trayectoria A.2.1]
    \UCpaso Cierra el modal \IUref{AAT}{Registrar Tipo de Acreditación}.
    \UCpaso Continua en el paso 1 de la trayectoria principal del \UCref{SP1-CU3}
\end{UCtrayectoriaA}

%------------------------ CU TRAYECTORIA ALTERNATIVA A.2.2 -------------------------

\begin{UCtrayectoriaA}{A.2.1}{El usuario no quiere cancelar el registro de la nueva acreditación.}
    \UCpaso[\UCactor] Presiona el botón \IUbutton{No}.
    \UCpaso Cierra el mensaje.
    \UCpaso Continua en el paso 4 de la trayectoria alternativa A del \UCref{SP1-CU3}
\end{UCtrayectoriaA}

%------------------------ CU TRAYECTORIA ALTERNATIVA B -------------------------

\begin{UCtrayectoriaA}{B}{El docente quiere agregar más criterios de evaluación.}
    \UCpaso[\UCactor] Quiere agregar más criterios de evaluación para la Unidad de Aprendizaje.
    \UCpaso[\UCactor] Presiona el botón \IUbutton{Agregar evaluación}.
    \UCpaso Despliega otro campo ``Evaluación'' con su respectivo campo ``Porcentaje''.
    \UCpaso Continua en el paso 2 de la trayectoria principal del \UCref{SP1-CU3}.
\end{UCtrayectoriaA}

%------------------------ CU TRAYECTORIA ALTERNATIVA C -------------------------

\begin{UCtrayectoriaA}{C}{El docente quiere quiere cancelar el registro.}
    \UCpaso[\UCactor] Quiere cancelar el registro de evaluación y acreditación de una Unidad de Aprendizaje.
    \UCpaso[\UCactor] Presiona el botón \IUbutton{Cancelar}.
    \UCpaso Muestra el mensaje \MSGref{MSG29}{¿Está seguro que desea cancelar? Se perderán todos los avances sin guardar.}
    \UCpaso[\UCactor] Confirma la operación presionando el botón \IUbutton{Si}. [Trayectoria C.1]
    \UCpaso Cierra el mensaje.
    \UCpaso Muestra la \IUref{RPS}{Registrar Programa Sintético.}
    \UCpaso Continua en el paso 8 de la trayectoria principal del \UCref{SP1-CU1}.
\end{UCtrayectoriaA}

%------------------------ CU TRAYECTORIA ALTERNATIVA C.1 -------------------------

\begin{UCtrayectoriaA}{C.1}{El docente no quiere quiere cancelar el registro.}

    \UCpaso[\UCactor] Presiona el botón \IUbutton{No}.
    \UCpaso Cierra el mensaje.
    \UCpaso Continua en el paso 4 de la trayectoria principal del \UCref{SP1-CU3}.
\end{UCtrayectoriaA}

%------------------------ CU TRAYECTORIA ALTERNATIVA D -------------------------

\begin{UCtrayectoriaA}{D}{Uno o más campos obligatorios no fueron contestados.}
    \UCpaso Detecta uno o más campos sin contestar.
    \UCpaso Bloquea el botón \IUbutton{Guardar}.
    \UCpaso Marca con color rojo los campos que no han sido contestados.
    \UCpaso Continua en el paso 1 de la trayectoria principal del \UCref{SP1-CU3}.
\end{UCtrayectoriaA}


%------------------------ CU TRAYECTORIA ALTERNATIVA E -------------------------

\begin{UCtrayectoriaA}{E}{La suma de los porcentajes es distinta a 100\%}
    \UCpaso Detecta que la suma de los porcentajes dados por el usuario no cumplen con el 100\% requerido.
    \UCpaso Muestra el mensaje \MSGref{MSG23.}{Los porcentajes de evaluación no cumplen con el porcentaje total obligatorio.}
    \UCpaso[\UCactor] Cierra el mensaje presionando el botón \IUbutton{Aceptar}.
    \UCpaso Continua en el paso 3 de la trayectoria principal del \UCref{SP1-CU3}.
\end{UCtrayectoriaA}

%------------------------ CU TRAYECTORIA ALTERNATIVA F -------------------------

\begin{UCtrayectoriaA}{F}{Ocurre un error al momento de persistir los datos.}
	\UCpaso Muestra el mensaje \MSGref{MSG25}{Servicios no disponibles}.
	\UCpaso[\UCactor] Cierra el mensaje presionando el botón \IUbutton{Aceptar}.
	\UCpaso Muestra la interfaz de usuario \IUref{REA}{Registrar Evaluación y Acreditación}.
\end{UCtrayectoriaA}