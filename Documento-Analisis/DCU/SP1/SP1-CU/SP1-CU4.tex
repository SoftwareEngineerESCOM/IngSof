% REGISTRAR EL PROGRAMA EN EXTENSO: NAIBI.
\begin{UseCase}{SP1-CU4}{Registrar el Programa en Extenso}{El usuario podrá registrar de manera más detallada las características fundamentales de la Unidad de Aprendizaje.}
		\UCitem{Versión}{\color{Gray}1.1}
		\UCitem{Autor}{\color{Gray}Ramírez Martínez Janet Naibi.}
		\UCitem{Supervisa}{\color{Gray}Cervantes Delgadillo Mauricio.}
		\UCitem{Actor}{\hyperlink{Docente}{Docente}}
		\UCitem{Propósito}{Registrar aspectos importantes de la Unidad de Aprendizaje.}
		\UCitem{Entradas}{Las entradas serán:
          \begin{itemize}
            \item Modalidad.
            \item Tipo de Unidad de Aprendizaje.
            \item Enseñanza.
            \item Vigencia.
            \item Intención educativa.
            \item Grado de estudios.
            \item Conocimientos.
            \item Experiencia profesional.
            \item Habilidad.
            \item Actitud.
          \end{itemize}
        }
		\UCitem{Origen}{Teclado y Mouse.}
		\UCitem{Salidas}{
        	\begin{itemize}
        		\item MSG4. Los campos marcados con (*) son obligatorios.
                \item MSG5. Registro finalizado exitosamente.
                \item MSG25. Servicios no disponibles.
                \item MSG58. Avances guardados exitosamente.
        	\end{itemize}
        }
		\UCitem{Destino}{Pantalla.}
		\UCitem{Precondiciones}{
			\begin{itemize}
				\item Se debe haber elaborado previamente el Programa Sintético.
				\item Los catálogos deben contener información.
			\end{itemize}
		}
		\UCitem{Postcondiciones}{El Programa en Extenso queda registrado en la base de datos.}
		\UCitem{Errores}{}
		\UCitem{Estado}{Revisión.}
		\UCitem{Observaciones}{}
\end{UseCase}

%--------------------------- CU TRAYECTORIA PRINCIPAL -------------------------
\begin{UCtrayectoria}{Principal}

	\UCpaso[\UCactor] Selecciona la opción \IUbutton{Programa en Extenso} del menú superior.
	
	\UCpaso Extrae la información: ``Unidad Académica'', ``Programa Académico'', ``Unidad de Aprendizaje'', ``Área de Formación'', ``Semestre'' y ``Créditos TEPIC/SATCA'' perteneciente al identificador de la Unidad de Aprendizaje seleccionada en la interfaz de usuario \IUref{VTA}{Ver tareas}.
	
	\UCpaso Calcula los tiempos asignados de la Unidad de Aprendizaje con base en la regla de negocios \BRref{BR44}{Calculo de los tiempos asignados.}
	
	\UCpaso Verifica que los catálogos ``Modalidad'', ``Tipo'' y ``Enseñanza'' contengan información. [Trayectoria A]
	
	\UCpaso Muestra la \IUref{RPS}{Registrar Programa en Extenso}. 
	
	\UCpaso[\UCactor] Selecciona la Modalidad de la Unidad de Aprendizaje. %presencial, en linea, etc.
	
	\UCpaso[\UCactor] Selecciona el/los tipo(s) de la Unidad de Aprendizaje. %teorica/teórica-práctica/práctica.
	
	\UCpaso[\UCactor] Selecciona la Enseñanza de la Unidad de Aprendizaje. %obligatoria, optativa, electiva, etc.
	
	\UCpaso[\UCactor] Ingresa el año de vigencia de la Unidad de Aprendizaje con base en la regla \BRref{BR26}{Vigencia de una Unidad de Aprendizaje}. [Trayectoria B]
	
	\UCpaso[\UCactor] Ingresa la Intención Educativa de la Unidad de Aprendizaje.
	
	\UCpaso[\UCactor] Ingresa el grado de estudios del docente que va a impartir la unidad de aprendizaje.
	
	\UCpaso[\UCactor] Ingresa los conocimientos con los que debe contar el docente que va a impartir la unidad de aprendizaje.
	
	\UCpaso[\UCactor] Ingresa la experiencia profesional del docente que va a impartir la unidad de aprendizaje.
	
	\UCpaso[\UCactor] Ingresa las habilidades con las que debe contar el docente que va a impartir la unidad de aprendizaje.
	
	\UCpaso[\UCactor] Ingresa las actitudes con las que debe contar el docente que va a impartir la unidad de aprendizaje.
	
	\UCpaso[\UCactor] Termina la operación presionando el botón \IUbutton{Finalizar}. [Trayectoria C]
        
    \UCpaso Verifica que todos los campos marcados como obligatorios hayan sido completamente llenados. [Trayectoria D]
    
    \UCpaso Guarda la información del Programa en Extenso en la base de datos. [Trayectoria F]
    
    \UCpaso Muestra el mensaje \MSGref{MSG5.}{Registro finalizado exitosamente}.
    
    \UCpaso[\UCactor] Cierra el mensaje presionando el botón \IUbutton{Aceptar}.
    
    \UCpaso Muestra la interfaz de usuario \IUref{RPE}{Registrar Programa en Extenso}.
\end{UCtrayectoria}

%------------------------ CU TRAYECTORIA ALTERNATIVA A -------------------------

\begin{UCtrayectoriaA}{A}{Uno o más catálogos están vacíos.}
	\UCpaso No encuentra información en los catálogos.
    \UCpaso Muestra el mensaje \MSGref{MSG25}{Servicios no disponibles}.
    \UCpaso[\UCactor] Cierra el mensaje presionando el botón \IUbutton{Aceptar}.
	\UCpaso Muestra la interfaz de usuario \IUref{RPS}{Registrar programa en extenso}.
\end{UCtrayectoriaA}

%------------------------ CU TRAYECTORIA ALTERNATIVA B -------------------------

\begin{UCtrayectoriaA}{B}{La vigencia que seleccionó el docente es invalida.}
	\UCpaso[\UCactor] Selecciona una fecha distinta a la fecha en la que está siendo registrada la Unidad de Aprendizaje.
	\UCpaso Marca con color rojo el campo vigencia.
	\UCpaso Continua en el paso 9 de la trayectoria principal del \UCref{SP1-CU4}
\end{UCtrayectoriaA}

%------------------------ CU TRAYECTORIA ALTERNATIVA C -------------------------

\begin{UCtrayectoriaA}{C}{El actor desea guardar el progreso de su registro.}
\UCpaso[\UCactor] Presiona el botón \IUbutton{Guardar}
\UCpaso Almacena la información.
\UCpaso Muestra el \MSGref{MSG58}{Avances guardados exitosamente.}
\UCpaso[\UCactor] Presiona el botón \IUbutton{Aceptar} 
\UCpaso Muestra la \IUref{RPE}{Registrar Programa en Extenso}.
\end{UCtrayectoriaA}

%------------------------ CU TRAYECTORIA ALTERNATIVA D -------------------------

\begin{UCtrayectoriaA}{D}{Uno o más campos obligatorios no fueron contestados.}
	\UCpaso Detecta uno o más campos sin contestar.
    \UCpaso Muestra el mensaje \MSGref{MSG4.}{Los campos marcados con (*) son obligatorios.}
    \UCpaso[\UCactor] Cierra el mensaje presionando el botón \IUbutton{Aceptar}.
    \UCpaso Continua en el paso 5 de la trayectoria principal del \UCref{SP1-CU4}.
\end{UCtrayectoriaA}

%------------------------ CU TRAYECTORIA ALTERNATIVA D -------------------------

\begin{UCtrayectoriaA}{F}{Ocurre un error al momento de persistir los datos.}
	\UCpaso Muestra el mensaje \MSGref{MSG25}{Servicios no disponibles}.
	\UCpaso[\UCactor] Cierra el mensaje presionando el botón \IUbutton{Aceptar}.
	\UCpaso Muestra la interfaz de usuario \IUref{RPE}{Registrar Programa en Extenso}.
\end{UCtrayectoriaA}