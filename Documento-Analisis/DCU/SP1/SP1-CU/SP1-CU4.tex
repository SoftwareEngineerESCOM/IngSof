% REGISTRAR EL PROGRAMA EN EXTENSO: NAIBI.
\begin{UseCase}{SP1-CU4}{Registrar el Programa en Extenso}{El usuario podrá registrar de manera más detallada las características fundamentales de la Unidad de Aprendizaje.}
		\UCitem{Versión}{\color{Gray}1.0}
		\UCitem{Autor}{\color{Gray}Ramírez Martínez Janet Naibi.}
		\UCitem{Supervisa}{\color{Gray}Cervantes Delgadillo Mauricio.}
		\UCitem{Actor}{\hyperlink{Docente}{Docente}}
		\UCitem{Propósito}{Registrar aspectos importantes de la Unidad de Aprendizaje.}
		\UCitem{Entradas}{Las entradas serán:
          \begin{itemize}
            \item Modalidad.
            \item Tipo de Unidad de Aprendizaje.
            \item Enseñanza.
            \item Vigencia.
            \item Intención educativa.
            \item Tiempos asignados. 
          \end{itemize}
        }
		\UCitem{Origen}{Teclado y Mouse.}
		\UCitem{Salidas}{
        	\begin{itemize}
        		\item MSG4. Los campos marcados con (*) son obligatorios.
                \item MSG5. Registro finalizado exitosamente.
                \item MSG9. Por el momento no se puede realizar el registro.
        	\end{itemize}
        }
		\UCitem{Destino}{Pantalla.}
		\UCitem{Precondiciones}{
			\begin{itemize}
				\item Se debe haber elaborado previamente el Programa Sintético.
				\item Los catálogos deben contener información.
			\end{itemize}
		}
		\UCitem{Postcondiciones}{El Programa en Extenso queda registrado en la base de datos.}
		\UCitem{Errores}{}
		\UCitem{Estado}{Revisión.}
		\UCitem{Observaciones}{}
\end{UseCase}

%--------------------------- CU TRAYECTORIA PRINCIPAL -------------------------
\begin{UCtrayectoria}{Principal}

	\UCpaso[\UCactor] Presiona el botón \IUbutton{Registrar PE} de la \IUref{SP1-IU}{Principal}.
	
	\UCpaso Extrae la información: ``Unidad Académica'', ``Programa Académico'', ``Unidad de Aprendizaje'', ``Área de Formación'', ``Semestre'', ``Créditos TEPIC/SATCA'', ``Propósito de la Unidad de Aprendizaje'' perteneciente al \UCref{SP1-CU1}. [Trayectoria A]
	
	\UCpaso Verifica que los catálogos contengan información. [Trayectoria B]
	
	\UCpaso Muestra la \IUref{IU.02}{Registrar el Programa en Extenso}. 
	
	\UCpaso[\UCactor] Selecciona la Modalidad de la Unidad de Aprendizaje que está siendo registrada. %presencial, en linea, etc.
	
	\UCpaso[\UCactor] Selecciona el Tipo de Unidad de Aprendizaje. %teorica/teórica-práctica/práctica.
	
	\UCpaso[\UCactor] Selecciona la Enseñanza de la Unidad de Aprendizaje. %obligatoria, optativa, electiva, etc.
	\UCpaso[\UCactor] Ingresa la Vigencia de la Unidad de Aprendizaje con base en la regla \BRref{BR26}{Vigencia de una Unidad de Aprendizaje}. [Trayectoria C]
	
	\UCpaso[\UCactor] Ingresa la Intención Educativa de la Unidad de Aprendizaje.
	
	\UCpaso[\UCactor] Ingresa los Tiempos Asignados de la Unidad de Aprendizaje. [Trayectoria D]
	
	\UCpaso[\UCactor] Termina la operación presionando el botón \IUbutton{Guardar}.
        
    \UCpaso Verifica que todos los campos marcados como obligatorios hayan sido completamente contestados. [Trayectoria E]
    
    \UCpaso Guarda la información del Programa en Extenso en la base de datos.
    
    \UCpaso Muestra el mensaje \MSGref{MSG5.}{Registro finalizado exitosamente}.
    
    \UCpaso[\UCactor] Cierra el mensaje presionando el botón \IUbutton{Aceptar}.
    
    \UCpaso Muestra la interfaz de usuario \IUref{SP1-IU}{Principal}.
\end{UCtrayectoria}

%------------------------ CU TRAYECTORIA ALTERNATIVA A -------------------------

\begin{UCtrayectoriaA}{A}{No se ha realizado la elaboración del Programa Sintético.}
	\UCpaso No muestra los datos provenientes del \UCref{SP1-CU1}.
	\UCpaso Muestra el mensaje \MSGref{MSG41}{Debe llenar el Programa Sintético para realizar este registro}.
	\UCpaso[\UCactor] Cierra el mensaje presionando el botón \IUbutton{Aceptar}.
	\UCpaso Muestra la interfaz de usuario \IUref{SP1-IU}{Principal}.
\end{UCtrayectoriaA}

%------------------------ CU TRAYECTORIA ALTERNATIVA B -------------------------



\begin{UCtrayectoriaA}{B}{Uno o más catálogos están vacíos.}
	\UCpaso No encuentra información en los catálogos.
    \UCpaso Muestra el mensaje \MSGref{MSG9.}{Por el momento no se puede realizar el registro.}
    \UCpaso[\UCactor] Cierra el mensaje presionando el botón \IUbutton{Aceptar}.
	\UCpaso Muestra la interfaz de usuario \IUref{SP1-IU}{Principal}.
\end{UCtrayectoriaA}

%------------------------ CU TRAYECTORIA ALTERNATIVA C -------------------------

\begin{UCtrayectoriaA}{C}{La vigencia que seleccionó el docente es invalida.}
	\UCpaso[\UCactor] Selecciona una fecha distinta a la fecha en la que está siendo registrada la Unidad de Aprendizaje.
	\UCpaso Muestra el mensaje \MSGref{MSG42}{Fecha invalida}.
	\UCpaso[\UCactor] Cierra el mensaje presionando el botón \IUbutton{Aceptar}.
	\UCpaso Continua en el paso 8 de la trayectoria principal del \UCref{SP1-CU4}
\end{UCtrayectoriaA}

%------------------------ CU TRAYECTORIA ALTERNATIVA D -------------------------

\begin{UCtrayectoriaA}{D}{El docente quiere registrar los tiempos asignados de la Unidad de Aprendizaje.}
	\UCpaso[\UCactor] Presiona el botón \BtnModal que se encuentra a un lado del campo ``Tiempos Asignados'' de la \IUref{IU.02}{Registrar el Programa en Extenso}.
	\UCpaso Muestra el modal \IUref{MIU2.01}{Registrar Tiempos Asignados de la Unidad de Aprendizaje}.
	\UCpaso Continua en el paso X de la trayectoria principal de \UCref{SP1-CU5}
\end{UCtrayectoriaA}

%------------------------ CU TRAYECTORIA ALTERNATIVA E -------------------------

\begin{UCtrayectoriaA}{E}{Uno o más campos obligatorios no fueron contestados.}
	\UCpaso Detecta uno o más campos sin contestar.
    \UCpaso Muestra el mensaje \MSGref{MSG4.}{Los campos marcados con (*) son obligatorios.}
    \UCpaso[\UCactor] Cierra el mensaje presionando el botón \IUbutton{Aceptar}.
    \UCpaso Continua en el paso 5 de la trayectoria principal del \UCref{SP1-CU4}.
\end{UCtrayectoriaA}