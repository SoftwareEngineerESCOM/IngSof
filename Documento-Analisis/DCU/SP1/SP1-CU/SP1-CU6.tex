%REGISTRAR UNIDAD TEMÁTICA: JORGE

\begin{UseCase}{SP1-CU6}{Registrar Unidad Temática}{El usuario podrá registrar una Unidad Temática correspondiente a una Unidad de Aprendizaje}
    \UCitem{Versión}{\color{Gray}1.0}
    \UCitem{Autor}{\color{Gray}Maldonado Carpio Jorge Enrique.}
    \UCitem{Supervisa}{\color{Gray}Cervantes Delgadillo Mauricio.}
    \UCitem{Actor}{\hyperlink{Usuario}{Docente y Jefe de Innovación Educativa.}}
    \UCitem{Propósito}{Poblar de Unidades Temáticas  a una Unidad de Aprendizaje.}
    \UCitem{Entradas}{Las entradas para el registro de tiempos de la Unidad de Aprendizaje serán:
      \begin{itemize}
          \item Unidad de Aprendizaje.
          \item Número de Unidad Tématica.
          \item Nombre.
          \item Unidad de Competencia.
          \item Temas.
          \item Estrategias de Aprendizaje.
          \item Evaluación de los Aprendizajes.
      \end{itemize}
    }
    \UCitem{Origen}{Teclado y Mouse.}
    \UCitem{Salidas}{
    \begin{itemize}
          \item \MSGref{MSG4}{Los campos marcados con (*) son obligatorios}.
          \item \MSGref{MSG5}{Registro finalizado exitosamente}.
          \item \MSGref{MSG25}{Servicios no disponibles por el momento}.
          \item \MSGref{MSG29}{¿Está seguro que desea cancelar? Se perderán todos los avances sin guardar.}
          \item \MSGref{MSG35}{Escribe información válida.}

      \end{itemize}
    }
    \UCitem{Destino}{Pantalla.}
    \UCitem{Precondiciones}{Los siguientes catálogos no deben de estar vacios:
      \begin{itemize}
          \item Unidad de Aprendizaje.
      \end{itemize}
      }
    \UCitem{Postcondiciones}{La Unidad Temáticas queda registrado en el sistema y asociada a una Unidad de Aprendizaje.}
    \UCitem{Errores}{}
    \UCitem{Estado}{Revisión.}
    \UCitem{Observaciones}{}
\end{UseCase}

%--------------------------- CU TRAYECTORIA PRINCIPAL -------------------------
\begin{UCtrayectoria}{Principal}

\UCpaso[\UCactor] Presiona la opción \IUbutton{Registrar Unidad Temática} de la interfaz de usuario \IUref{IU1}{Principal}.
\UCpaso Verifica que los catálogos de Unidad de Aprendizaje contengan información.
\UCpaso Carga la información de los catálogos.
\UCpaso Muestra la interfaz de usuario \IUref{SP1-U3}{Registro de Unidad Temática}.
\UCpaso[\UCactor] Selecciona la Unidad de Aprendizaje conforme al modelo de datos, la \BRref{BR38}{Verificación de formularios al momento} y la \BRref{BR39}{Todos los campos marcados con (*) son obligatorios} [Trayectoria D] [Trayectoria E].
\UCpaso[\UCactor] Ingresa el número de Unidad Tématica conforme al modelo de datos, la \BRref{BR38}{Verificación de formularios al momento} y la \BRref{BR39}{Todos los campos marcados con (*) son obligatorios} [Trayectoria D] [Trayectoria E].
\UCpaso[\UCactor] Ingresa el nombre de la Unidad Temática conforme al modelo de datos, la \BRref{BR38}{Verificación de formularios al momento} y la \BRref{BR39}{Todos los campos marcados con (*) son obligatorios} [Trayectoria D] [Trayectoria E].
\UCpaso[\UCactor] Ingresa la Unidad de Competencia de la Unidad Temática conforme al modelo de datos, la \BRref{BR38}{Verificación de formularios al momento} y la \BRref{BR39}{Todos los campos marcados con (*) son obligatorios} [Trayectoria D] [Trayectoria E].
\UCpaso[\UCactor] Presiona el botón \BtnModal que se encuentra a un lado del campo ``Temas''  [Trayectoria A] 
\UCpaso[\UCactor] Ingresa las Estrategias de Aprendizaje conforme al modelo de datos, la \BRref{BR38}{Verificación de formularios al momento} y la \BRref{BR39}{Todos los campos marcados con (*) son obligatorios} [Trayectoria D] [Trayectoria E].
\UCpaso[\UCactor] Presiona el botón  \BtnModal que se encuentra a un lado del campo ``Evaluación de los Aprendizajes''[Trayectoria B] 
\UCpaso[\UCactor] Termina la operación presionando el botón \IUbutton{Guardar}. [Trayectoria C]
\UCpaso Verifica que todos los campos marcados como obligatorios hayan sido llenados.[Trayectoria D]
\UCpaso Guarda la información de la Unidad Temática [Trayectoria F].
\UCpaso El sistema muestra el mensaje \MSGref{MSG5}{Registro finalizado exitosamente}.
\UCpaso[\UCactor] Cierra el mensaje presionando el botón \IUbutton{OK}. 
\UCpaso Muestra la interfaz de usuario \IUref{IU1}{Principal}.
\end{UCtrayectoria}


\begin{UCtrayectoriaA}{A}{El usuario desea agregar un tema.}
\UCpaso Abre un modal en el \UCref{SP1-CU7}
\UCpaso Continua en el paso 2 de la trayectoria principal del \UCref{SP1-CU7}.
\end{UCtrayectoriaA}

\begin{UCtrayectoriaA}{B}{El usuario desea agregar la Evaluación de los Aprendizajes.}
\UCpaso Abre un modal en el \UCref{SP1-CU8}
\UCpaso Continua en el paso 2 de la trayectoria principal del \UCref{SP1-CU8}.
\end{UCtrayectoriaA}

\begin{UCtrayectoriaA}{C}{El usuario desea cancelar el registro del Programa Sintético.}
\UCpaso[\UCactor] Presiona el botón \IUbutton{Cancelar}
\UCpaso Muestra el \MSGref{MSG29}{¿Está seguro que desea cancelar? Se perderán todos los avances sin guardar.}
\UCpaso[\UCactor] Presiona el botón \IUbutton{Si} [Trayectoria C.1]
\UCpaso Continua en el paso 17 de la trayectoria principal del \UCref{SP1-CU6}.
\end{UCtrayectoriaA}

\begin{UCtrayectoriaA}{C.1}{El usuario no desea cancelar el registro del Programa Sintético.}
\UCpaso[\UCactor] Presiona el botón \IUbutton{No}
\UCpaso Continua en el paso 12 de la trayectoria principal del \UCref{SP1-CU6}.
\end{UCtrayectoriaA}

\begin{UCtrayectoriaA}{D}{Uno o más campos obligatorios no fueron contestados.}
\UCpaso Detecta uno o más campos sin contestar.
\UCpaso Muestra el mensaje \MSGref{MSG4}{Los campos marcados con (*) son obligatorios}.
\UCpaso[\UCactor] Cierra el mensaje presionando el botón \IUbutton{Aceptar}.
\UCpaso Continua en el paso 6 de la trayectoria principal del \UCref{SP1-CU1}.
\end{UCtrayectoriaA}

\begin{UCtrayectoriaA}{E}{El sistema detecta caracteres no válidos conforme al Modelo de Datos.}
\UCpaso Muestra el mensaje \MSGref{MSG35}{Escribe información válida} debajo del campo que incumplio con el diccionario de datos.
\UCpaso Continúa en el paso 6 de la trayectoria principal del \UCref{SP1-CU1}.
\end{UCtrayectoriaA}

\begin{UCtrayectoriaA}{F}{Ocurre un error al momento de persistir los datos.}
\UCpaso Muestra el mensaje \MSGref{MSG25}{Servicios no disponibles por el momento}.
\UCpaso[\UCactor] Cierra el mensaje presionando el botón \IUbutton{Aceptar}.
\UCpaso Continúa en el paso 17 de la trayectoria principal del \UCref{SP1-CU1}.
\end{UCtrayectoriaA}