%REGISTRAR TEMAS DE LA UA: AIKO.
\begin{UseCase}{SP1-CU7}{Registrar Temas de la Unidad de Aprendizaje}{El usuario podrá registrar temas a la Unidad de Aprendizaje.}
    \UCitem{Versión}{\color{Gray}1.0}
    \UCitem{Autor}{\color{Gray}López Rivera Aiko Dallane}
    \UCitem{Supervisa}{\color{Gray}Ramírez Martínez Janet Naibi}
    \UCitem{Actor}{\hyperlink{Usuario}{Docente}}
    \UCitem{Propósito}{Servir como marco de referencia para el registro de Temas de la Unidad de Aprendizaje.}
    \UCitem{Entradas}{Las entradas para el registro del Perfil Docente serán:
          \begin{itemize}
            \item Número de tema. 
            \item Nombre del Tema.
            \item Horas con docente teóricas.
            \item Horas con docente prácticas.
            \item Horas de Aprendizaje Autónomo.
            \item Subtemas.
          \end{itemize}
        }
    \UCitem{Origen}{Teclado y Mouse.}
    \UCitem{Salidas}{
	\begin{itemize}
               \item \MSGref{MSG5}{Registro finalizado exitosamente.}
	       \item \MSGref{MSG25}{Servicios no disponibles por el 	momento.}
	       \item \MSGref{MSG29}{ ¿Está seguro que desea cancelar? Se perderán todos los avances sin guardar.}
	       \item \MSGref{MSG35}{Escribe información válida}
	       \item \MSGref{MSG41}{Debe llenar el Programa Sintético para realizar este registro}.
	       \item \MSGref{MSG44}{Este campo es requerido}
     \end{itemize}

    }
    \UCitem{Destino}{Pantalla.}
    \UCitem{Precondiciones}{Se debe haber registrado previamente el Programa Sintético.}
    \UCitem{Postcondiciones}{Los temas de la Unidad de Aprendizaje quedan registrados.}
    \UCitem{Errores}{
	\begin{itemize}
		\item E1. Hubo un problema al conectarse con la base de datos.
	\end{itemize}
}
    \UCitem{Estado}{Revisión.}
    \UCitem{Observaciones}{}
\end{UseCase}

%--------------------------- CU TRAYECTORIA PRINCIPAL -------------------------
\begin{UCtrayectoria}{Principal}

     \UCpaso[\UCactor] Presiona el botón \BtnModal que se encuentra a un lado del campo ``Temas'' de la \IUref{IU.03}{Registrar Unidad Tématica}.
    \UCpaso Muestra el modal \IUref{MIU3.01}{Registrar Temas de la Unidad de Aprendizaje}.
    \UCpaso[\UCactor] Ingresa el no. de tema, el sistema verifica conforme al modelo de datos, la \BRref{BR38}{Verificación de formularios al momento} y la \BRref{BR39}{Todos los campos marcados con (*) son obligatorios}.[Trayectoria C]
    \UCpaso[\UCactor] Ingresa el nombre del tema, el sistema verifica conforme al modelo de datos, la \BRref{BR38}{Verificación de formularios al momento} y la \BRref{BR39}{Todos los campos marcados con (*) son obligatorios}.[Trayectoria C]
    \UCpaso[\UCactor] Selecciona las horas con docente teóricas, el sistema verifica conforme al modelo de datos, la \BRref{BR38}{Verificación de formularios al momento} y la \BRref{BR39}{Todos los campos marcados con (*) son obligatorios}.[Trayectoria C]
    \UCpaso[\UCactor] Selecciona las horas con docente prácticas, el sistema verifica conforme al modelo de datos, la \BRref{BR38}{Verificación de formularios al momento} y la \BRref{BR39}{Todos los campos marcados con (*) son obligatorios}.[Trayectoria C]
    \UCpaso[\UCactor] Selecciona las horas con docente autónomas, el sistema verifica conforme al modelo de datos, la \BRref{BR38}{Verificación de formularios al momento} y la \BRref{BR39}{Todos los campos marcados con (*) son obligatorios}.[Trayectoria C]
    \UCpaso[\UCactor] Presiona el botón \BtnModal{Agregar Subtemas}
    \UCpaso Abre un modal en el \UCref{SP1-CU8}
    \UCpaso[\UCactor] Presiona el botón \IUbutton{Agregar Tema}. [Trayectoria A]
    \UCpaso[\UCactor] Termina la operación presionando el botón \IUbutton{Guardar}. [Trayectoria B] [Trayectoria B.1]
	\UCpaso Verifica que se cumpla con el modelos de datos. [Trayectoria C]
	\UCpaso Persiste los datos ingresados. [Trayectoria D]
	\UCpaso El sistema muestra el mensaje \MSGref{MSG5}{Registro finalizado exitosamente}.
	\UCpaso[\UCactor] Cierra el mensaje presionando el botón \IUbutton{Aceptar}.
	\UCpaso Muestra la interfaz de usuario \IUref{UI.03}{Registrar Unidad Tématica}.
\end{UCtrayectoria}

%------------------------ CU TRAYECTORIA ALTERNARTIVA A -------------------------

\begin{UCtrayectoriaA}{A}{El usuario desea agregar otro tema.}
    \UCpaso El sistema genera nuevamente el formulario completo.
    \UCpaso[\UCactor] Registra los datos correspondientes al nuevo tema.
    \UCpaso Continua en el paso 9 de la trayectoria principal del \UCref{SP1-CU7}.

\end{UCtrayectoriaA}

%------------------------ CU TRAYECTORIA ALTERNARIVA B -------------------------
\begin{UCtrayectoriaA}{B}{El usuario presiona el botón \IUbutton{Cancelar}.}
	\UCpaso Muestra el mensaje \MSGref{MSG29}{¿Está seguro que desea cancelar? Se perderán todos los avances sin guardar}.
	\UCpaso[\UCactor] Confirma la operación presionando el botón \IUbutton{Si}.
	\UCpaso Muestra la interfaz de usuario \IUref{UI.03}{Registrar Unidad Tématica}.
\end{UCtrayectoriaA}

%------------------------ TRAYECTORIA ALTERNARIVA B.1 -------------------------
\begin{UCtrayectoriaA}{B.1}{El usuario no desea cancelar el Registro de COntedidos \IUbutton{Cancelar}.}
	\UCpaso[\UCactor] Presiona el botón \IUbutton{Cancelar}
	\UCpaso Muestra el mensaje \MSGref{MSG29}{¿Está seguro que desea cancelar? Se perderán todos los avances sin guardar}.
	\UCpaso[\UCactor] Presiona el botón \IUbutton{No}.
	\UCpaso Cierra el mensaje.
	\UCpaso Continúa en el paso 3 de la trayectoria principal del \UCref{SP1-CU7}.
\end{UCtrayectoriaA}

%------------------------ TRAYECTORIA ALTERNARIVA C -------------------------
\begin{UCtrayectoriaA}{C}{El sistema detecta uno o más campos sin contestar.}
	\UCpaso Muestra el mensaje \MSGref{MSG44}{Este campo es requerido} debajo de cada campo obligatorio sin contestar.
	\UCpaso Continúa en el paso 3 de la trayectoria principal del \UCref{SP1-CU7}.
\end{UCtrayectoriaA}

%------------------------ TRAYECTORIA ALTERNARIVA D -------------------------
\begin{UCtrayectoriaA}{D}{Ocurre un error al momento de persistir los datos.}
	\UCpaso Muestra el mensaje \MSGref{MSG25}{Servicios no disponibles por el momento}.
	\UCpaso[\UCactor] Cierra el mensaje presionando el botón \IUbutton{Aceptar}.
	\UCpaso Muestra la interfaz de usuario \IUref{UI.01}{Registrar el programa Sintético}.
\end{UCtrayectoriaA}

