%REGISTRAR TEMAS DE LA UA: AIKO.
\begin{UseCase}{SP1-CU7}{Registrar Temas de la Unidad de Aprendizaje}{El usuario podrá registrar temas a la Unidad de Aprendizaje.}
    \UCitem{Versión}{\color{Gray}1.1}
    \UCitem{Autor}{\color{Gray}López Rivera Aiko Dallane}
    \UCitem{Supervisa}{\color{Gray}Ramírez Martínez Janet Naibi}
    \UCitem{Actor}{\hyperlink{Docente}{Docente}}
    \UCitem{Propósito}{Servir como marco de referencia para el registro de Temas de cada Unidad de Temática.}
    \UCitem{Entradas}{Las entradas para el registro del Perfil Docente serán:
          \begin{itemize}
            \item Nombre del Tema.
            \item Horas con docente teóricas.
            \item Horas con docente prácticas.
            \item Horas de Aprendizaje Autónomo.
            \item Subtemas.
          \end{itemize}
        }
    \UCitem{Origen}{Teclado y Mouse.}
    \UCitem{Salidas}{
        \begin{itemize}
            \item MSG4. Los campos marcados con (*) son obligatorios.
            \item MSG5. Registro finalizado exitosamente.
            \item MSG25. Servicios no disponibles.
            \item MSG29. ¿Está seguro que desea cancelar? Se perderán todos los avances sin guardar.
        \end{itemize}
    }
    \UCitem{Destino}{Pantalla.}
    \UCitem{Precondiciones}{Se debe haber elaborado previamente el Programa Sintético.}
    \UCitem{Postcondiciones}{Los temas de la Unidad de Aprendizaje quedan registrados.}
    \UCitem{Errores}{}
    \UCitem{Estado}{Revisión.}
    \UCitem{Observaciones}{}
\end{UseCase}

%--------------------------- CU TRAYECTORIA PRINCIPAL -------------------------
\begin{UCtrayectoria}{Principal}
    \UCpaso[\UCactor] Ingresa el nombre del tema.
    
    \UCpaso[\UCactor] Selecciona las horas con docente teóricas.
    
    \UCpaso[\UCactor] Selecciona las horas con docente prácticas.
    
    \UCpaso[\UCactor] Selecciona las horas de aprendizaje autónomo.
    
    \UCpaso[\UCactor] Termina la operación presionando el botón \IUbutton{Guardar}.[Trayectoria A] [Trayectoria B] [Trayectoria C]
    
    \UCpaso Verifica que todos los campos marcados como obligatorios hayan sido llenados completamente. [Trayectoria D]
    
    \UCpaso Guarda la información de los temas correspondientes a la Unidad Temática. [Trayectoria E]
    
    \UCpaso Muestra el mensaje \MSGref{MSG5}{Registro finalizado exitosamente}.
    
    \UCpaso[\UCactor] Cierra el mensaje presionando el botón \IUbutton{Aceptar}.
    
    \UCpaso Muestra la interfaz de usuario \IUref{RUT}{Registrar Unidad Temática}.
    
    \UCpaso Continua en el paso 8 de la trayectoria principal del \UCref{SP1-CU6}.

\end{UCtrayectoria}

%------------------------ CU TRAYECTORIA ALTERNATIVA A -------------------------

\begin{UCtrayectoriaA}{A}{El actor quiere agregar más temas a la Unidad Temática.}
    \UCpaso[\UCactor] Quiere agregar otro tema a la Unidad Temática que está registrando.
    \UCpaso[\UCactor] Presiona el botón \IUbutton{Agregar tema}. [Trayectoria A.1]
    \UCpaso Muestra el número de tema consecutivo con sus respectivos campos: ``Nombre del tema'', ``Horas con docente teóricas'', ``Horas con docente prácticas'', ``Horas de aprendizaje autónomo'' y el botón \IUbutton{Registrar subtema} para el nuevo tema agregado.
    \UCpaso Continua en el paso 1 de la trayectoria principal de \UCref{SP1-CU7}
\end{UCtrayectoriaA}

%------------------------ CU TRAYECTORIA ALTERNATIVA A.1 -------------------------

\begin{UCtrayectoriaA}{A.1}{El actor quiere eliminar el tema que acaba de agregar.}
    \UCpaso[\UCactor] Quiere eliminar el último tema agregado.
    \UCpaso[\UCactor] Presiona el botón \IUbutton{Borrar último tema}.
    \UCpaso Elimina el último tema añadido. 
\end{UCtrayectoriaA}

%------------------------ CU TRAYECTORIA ALTERNATIVA B -------------------------

\begin{UCtrayectoriaA}{B}{El actor quiere agregar subtemas al tema que está registrando.}
    \UCpaso[\UCactor] Quiere agregar subtemas al tema que está registrando.
    \UCpaso[\UCactor] Presiona el botón \IUbutton{Registrar subtemas \BtnModal}
    \UCpaso Muestra la interfaz de usuario \IUref{RSUA}{Registrar subtemas}
    \UCpaso Continua en el paso 1 de la trayectoria principal del \UCref{SP1-CU8}
\end{UCtrayectoriaA}

%------------------------ CU TRAYECTORIA ALTERNATIVA C -------------------------

\begin{UCtrayectoriaA}{C}{El actor quiere cancelar el registro de temas para la Unidad Temática.}
    \UCpaso[\UCactor] Quiere cancelar el registro de temas para la Unidad Temática.
    \UCpaso[\UCactor] Presiona el botón \IUbutton{Cancelar}.
    \UCpaso Muestra el mensaje \MSGref{MSG29}{¿Está seguro que desea cancelar? Se perderán todos los avances sin guardar}.
    \UCpaso[\UCactor] Confirma la operación presionando el botón \IUbutton{Si}. [Trayectoria C.1]
    \UCpaso Muestra la interfaz de usuario \IUref{RUT}{Registrar Unidades Temáticas}.
    \UCpaso Continua en el paso 7 de la trayectoria principal del \UCref{SP1-CU6}
\end{UCtrayectoriaA}

%------------------------ CU TRAYECTORIA ALTERNATIVA C.1 -------------------------

\begin{UCtrayectoriaA}{C.1}{El actor no quiere cancelar el registro de temas.}
    \UCpaso[\UCactor] No quiere cancelar el registro de temas.
    \UCpaso[\UCactor] Presiona el botón \IUbutton{No}.
    \UCpaso Cierra el mensaje. 
    \UCpaso Muestra la interfaz de usuario \IUref{RTUA}{Registrar temas de la Unidad Temática}.
\end{UCtrayectoriaA}

%------------------------ CU TRAYECTORIA ALTERNATIVA D -------------------------

\begin{UCtrayectoriaA}{D}{Uno o más campos obligatorios no fueron contestados.}
	\UCpaso Detecta uno o más campos sin contestar.
    \UCpaso Muestra el mensaje \MSGref{MSG4}{Los campos marcados con (*) son obligatorios.}
    \UCpaso[\UCactor] Cierra el mensaje presionando el botón \IUbutton{Aceptar}.
    \UCpaso Continua en el paso 1 de la trayectoria principal del \UCref{SP1-CU7}.
\end{UCtrayectoriaA}

%------------------------ CU TRAYECTORIA ALTERNATIVA E -------------------------

\begin{UCtrayectoriaA}{E}{Ocurre un error al momento de persistir los datos.}
	\UCpaso Muestra el mensaje \MSGref{MSG25}{Servicios no disponibles}.
	\UCpaso[\UCactor] Cierra el mensaje presionando el botón \IUbutton{Aceptar}.
	\UCpaso Muestra la interfaz de usuario \IUref{RTUA}{Registrar temas de la Unidad Temática}
\end{UCtrayectoriaA}
