%REGISTRAR SUBTEMAS DE LA UA: AIKO
\begin{UseCase}{SP1-CU8}{Registrar los subtemas de la Unidad de Aprendizaje}{El usuario podrá registrar los subtemas correspondientes a una Unidad de Aprendizaje.}
    \UCitem{Versión}{\color{Gray}1.0}
    \UCitem{Autor}{\color{Gray}López Rivera Aiko Dallane}
    \UCitem{Supervisa}{\color{Gray}Ramírez Martínez Janet Naibi}
    \UCitem{Actor}{\hyperlink{Usuario}{Usuario}}
    \UCitem{Propósito}{Servir como marco de referencia para el registro de los subtemas de una  Unidad de Aprendizaje.}
    \UCitem{Entradas}{Las entradas para el registro de contenidos de la Unidad de Aprendizaje serán:
          \begin{itemize}
            \item Número de Subtemas.
            \item Nombre del Subtema.
          \end{itemize}
        }
    \UCitem{Origen}{Teclado y Mouse.}
    \UCitem{Salidas}{}
    \UCitem{Destino}{Pantalla.}
    \UCitem{Precondiciones}{}
    \UCitem{Postcondiciones}{Los subtemas quedan registrados en el sistema.}
    \UCitem{Errores}{}
    \UCitem{Estado}{Revisión.}
    \UCitem{Observaciones}{}
\end{UseCase}

%--------------------------- CU TRAYECTORIA PRINCIPAL -------------------------
\begin{UCtrayectoria}{Principal}

    \UCpaso[\UCactor] Selecciona el número de subtemas a registrar
    \UCpaso Despliega un listado para agregar nombre al contenido de acuerdo al número previamente seleccionado.[Trayectoria A]
    \UCpaso[\UCactor] Ingresa el nombre de cada subtema.
    \UCpaso[\UCactor] Termina la operación presionando el botón \IUbutton{Guardar}. [Trayectoria B]
    \UCpaso Verifica que todos los campos marcados como obligatorios hayan sido llenados.[Trayectoria C]
    \UCpaso Guarda la información de los contenidos.
    \UCpaso El sistema muestra el mensaje \MSGref{MSG5}{Registro finalizado exitosamente}.
    \UCpaso[\UCactor] Cierra el mensaje presionando el botón \IUbutton{Ok}.
    \UCpaso Muestra la interfaz de usuario \IUref{MIU3.01}{Página principal}.
\end{UCtrayectoria}

%------------------------ CU TRAYECTORIA ALTERNARTIVA A -------------------------

\begin{UCtrayectoriaA}{A}{El usuario desea agregar más subtemas.}
    \UCpaso[\UCactor] Selecciona el número de contenidos a registrar
    \UCpaso Despliega un listado para agregar nombre al subtema de acuerdo al número previamente seleccionado.
    \UCpaso Continua en el paso 3 de la trayectoria principal del \UCref{SP1-CUX}.

\end{UCtrayectoriaA}

\begin{UCtrayectoriaA}{B}{El usuario desea cancelar el registro de Contenidos.}
    \UCpaso[\UCactor] Presiona el botón \IUbutton{Cancelar}
    \UCpaso Muestra el \MSGref{MSG6}{¿Seguro que desea cancelar el registro?}.
    \UCpaso[\UCactor] Presiona el botón \IUbutton{Si} [Trayectoria B.1]
    \UCpaso Continua en el paso X de la trayectoria principal del \UCref{SP1-C7}.

\end{UCtrayectoriaA}

\begin{UCtrayectoriaA}{B.1}{El usuario no desea cancelar el registro de Contenidos.}

    \UCpaso[\UCactor] Presiona el botón \IUbutton{No}
    \UCpaso Continua en el paso 4 de la trayectoria principal del \UCref{SP1-CUX}.

\end{UCtrayectoriaA}


\begin{UCtrayectoriaA}{C}{Uno o más campos obligatorios no fueron contestados.}
  \UCpaso Detecta uno o más campos sin contestar.
    \UCpaso Muestra el mensaje \MSGref{MSG4}{Los campos marcados con (*) son obligatorios}.
    \UCpaso[\UCactor] Cierra el mensaje presionando el botón \IUbutton{Aceptar}.
    \UCpaso Continua en el paso 3 de la trayectoria principal del \UCref{SP1-CUX}.
\end{UCtrayectoriaA}
