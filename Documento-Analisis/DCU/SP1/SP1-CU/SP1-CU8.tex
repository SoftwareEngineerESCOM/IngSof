%REGISTRAR SUBTEMAS DE LA UA: AIKO
\begin{UseCase}{SP1-CU8}{Registrar los subtemas de la Unidad de Aprendizaje}{El usuario podrá registrar los subtemas correspondientes a un tema de una Unidad Temática.}
    \UCitem{Versión}{\color{Gray}1.1}
    \UCitem{Autor}{\color{Gray}López Rivera Aiko Dallane}
    \UCitem{Supervisa}{\color{Gray}Ramírez Martínez Janet Naibi}
    \UCitem{Actor}{\hyperlink{Docente}{Docente}}
    \UCitem{Propósito}{Servir como marco de referencia para el registro de los subtemas de una  Unidad Temática.}
    \UCitem{Entradas}{Nombre del Subtema.}
    \UCitem{Origen}{Teclado y Mouse.}
    \UCitem{Salidas}{
        \begin{itemize}
            \item MSG4. Los campos marcados con (*) son obligatorios.
            \item MSG5. Registro finalizado exitosamente.
            \item MSG25. Servicios no disponibles.
            \item MSG29. ¿Está seguro que desea cancelar? Se perderán todos los avances sin guardar.
        \end{itemize}
    }
    \UCitem{Destino}{Pantalla.}
    \UCitem{Precondiciones}{Debe de existir un tema al cual asignarle subtemas.}
    \UCitem{Postcondiciones}{Los subtemas quedan registrados en el sistema.}
    \UCitem{Errores}{}
    \UCitem{Estado}{Revisión.}
    \UCitem{Observaciones}{}
\end{UseCase}

%--------------------------- CU TRAYECTORIA PRINCIPAL -------------------------
\begin{UCtrayectoria}{Principal}

    \UCpaso[\UCactor] Ingresa el nombre del subtema.
    
    \UCpaso[\UCactor] Termina la operación presionando el botón \IUbutton{Guardar}. [Trayectoria A] [Trayectoria B]
    
    \UCpaso Verifica que todos los campos marcados como obligatorios hayan sido llenados.[Trayectoria C]
    
    \UCpaso Guarda la información de los subtemas. [Trayectoria D]
    
    \UCpaso El sistema muestra el mensaje \MSGref{MSG5}{Registro finalizado exitosamente}.
    
    \UCpaso[\UCactor] Cierra el mensaje presionando el botón \IUbutton{Aceptar}.
    
    \UCpaso Muestra la interfaz de usuario \IUref{RTUA}{Registrar temas de la Unidad Temática}.
\end{UCtrayectoria}

%------------------------ CU TRAYECTORIA ALTERNARTIVA A -------------------------

\begin{UCtrayectoriaA}{A}{El usuario desea agregar más subtemas.}
    \UCpaso[\UCactor] Quiere agregar otro subtema.
    \UCpaso[\UCactor] Presiona el botón \IUbutton{Agregar subtema}. [Trayectoria A.1]
    \UCpaso Muestra el número de subtema consecutivo con su respectivo campo: ``Nombre del subtema''.
    \UCpaso Continua en el paso 1 de la trayectoria principal de \UCref{SP1-CU8}
\end{UCtrayectoriaA}

%------------------------ CU TRAYECTORIA ALTERNATIVA A.1 -------------------------

\begin{UCtrayectoriaA}{A.1}{El actor quiere eliminar el subtema que acaba de agregar.}
    \UCpaso[\UCactor] Quiere eliminar el último subtema agregado.
    \UCpaso[\UCactor] Presiona el botón \IUbutton{Eliminar último subtema}.
    \UCpaso Elimina el último subtema añadido. 
\end{UCtrayectoriaA}

%------------------------ CU TRAYECTORIA ALTERNATIVA B -------------------------

\begin{UCtrayectoriaA}{B}{El actor quiere cancelar el registro de subtemas.}
    \UCpaso[\UCactor] Quiere cancelar el registro de subtemas.
    \UCpaso[\UCactor] Presiona el botón \IUbutton{Cancelar}.
    \UCpaso Muestra el mensaje \MSGref{MSG29}{¿Está seguro que desea cancelar? Se perderán todos los avances sin guardar}.
    \UCpaso[\UCactor] Confirma la operación presionando el botón \IUbutton{Si}. [Trayectoria B.1] 
    \UCpaso Cierra el mensaje. 
    \UCpaso Muestra la interfaz de usuario \IUref{RTUA}{Registrar temas de la Unidad Temática}.
\end{UCtrayectoriaA}

%------------------------ CU TRAYECTORIA ALTERNATIVA B.1 -------------------------

\begin{UCtrayectoriaA}{B.1}{El actor no quiere cancelar el registro de subtemas.}
    \UCpaso[\UCactor] No quiere cancelar el registro de subtemas.
    \UCpaso[\UCactor] Presiona el botón \IUbutton{No}.
    \UCpaso Cierra el mensaje. 
    \UCpaso Muestra la interfaz de usuario \IUref{RSUA}{Registrar subtemas de la Unidad Temática}.
\end{UCtrayectoriaA}

%------------------------ CU TRAYECTORIA ALTERNATIVA C -------------------------

\begin{UCtrayectoriaA}{C}{Uno o más campos obligatorios no fueron contestados.}
  \UCpaso Detecta uno o más campos sin contestar.
    \UCpaso Muestra el mensaje \MSGref{MSG4}{Los campos marcados con (*) son obligatorios}.
    \UCpaso[\UCactor] Cierra el mensaje presionando el botón \IUbutton{Aceptar}.
    \UCpaso Continua en el paso 1 de la trayectoria principal del \UCref{SP1-CU8}.
\end{UCtrayectoriaA}

%------------------------ CU TRAYECTORIA ALTERNATIVA D -------------------------

\begin{UCtrayectoriaA}{D}{Ocurre un error al momento de persistir los datos.}
	\UCpaso Muestra el mensaje \MSGref{MSG25}{Servicios no disponibles}.
	\UCpaso[\UCactor] Cierra el mensaje presionando el botón \IUbutton{Aceptar}.
	\UCpaso Muestra la interfaz de usuario \IUref{RSUA}{Registrar subtemas de la Unidad Temática}
\end{UCtrayectoriaA}