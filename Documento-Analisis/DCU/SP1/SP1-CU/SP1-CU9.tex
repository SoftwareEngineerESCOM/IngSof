%REGISTRAR EVALUACIÓN DE LOS APRENDIZAJES: NAIBI
\begin{UseCase}{SP1-CU9}{Registrar la Evaluación de los Aprendizajes}{El usuario podrá registrar la evaluación de los aprendizajes esperados para cada unidad temática con su respectivo porcentaje y tipo de evaluación.}
		\UCitem{Versión}{\color{Gray}1.0}
		\UCitem{Autor}{\color{Gray}Ramírez Martínez Janet Naibi.}
		\UCitem{Supervisa}{\color{Gray}Cervantes Delgadillo Mauricio.}
		\UCitem{Actor}{\hyperlink{Docente}{Docente}}
		\UCitem{Propósito}{Que el alumno sepa de dónde proviene su calificación y además que el docente pueda tener un marco de referencia para evaluar al alumno.}
		\UCitem{Entradas}{Las entradas serán:
          \begin{itemize}
            \item Nombre del criterio de evaluación.
            \item Porcentaje del criterio ingresado en el punto anterior.
            \item Evaluación. %evaluación continua, evaluación escrita.
          \end{itemize}
        }
		\UCitem{Origen}{Teclado y Mouse.}
		\UCitem{Salidas}{
        	\begin{itemize}
        		\item MSG4. Los campos marcados con (*) son obligatorios.
                \item MSG5. Registro finalizado exitosamente.
                \item MSG9. Por el momento no se puede realizar el registro.
                \item MSG23. Los porcentajes de evaluación no cumplen con el porcentaje total obligatorio.
        	\end{itemize}
        }
		\UCitem{Destino}{Pantalla.}
		\UCitem{Precondiciones}{
			\begin{itemize}
				\item Se debe haber registrado previamente los Contenidos de la Unidad de Aprendizaje.
				\item Los catálogos deben contener información. 
				\item Debe existir un criterio de evaluación para poder asignarle un porcentaje.
				\item Se debe haber elaborado la Unidad Temática con sus respectivos temas y subtemas.  
			\end{itemize}
		}
		\UCitem{Postcondiciones}{La Evaluación de los Aprendizajes queda registrada en la base de datos.}
		\UCitem{Errores}{}
		\UCitem{Estado}{Revisión.}
		\UCitem{Observaciones}{La Evaluación de los Aprendizajes se debe realizar para cada Unidad Temática.}
\end{UseCase}

%--------------------------- CU TRAYECTORIA PRINCIPAL -------------------------
\begin{UCtrayectoria}{Principal}

	\UCpaso[\UCactor] Presiona el botón \BtnModal que se encuentra a un lado del campo ``Evaluación de los Aprendizajes'' de la \IUref{IU.03}{Registrar Unidad Temática}.
	
	\UCpaso Extrae la información de los catálogos de la base de datos. [Trayectoria A]

    \UCpaso Muestra el modal \IUref{MIU3.02}{Registrar Evaluación de los Aprendizajes}.
    
    \UCpaso[\UCactor] Ingresa el criterio de evaluación con base en la regla \BRref{BR24}{Debe existir al menos un criterio de evaluación para una Unidad de Aprendizaje}. [Trayectoria B]

    \UCpaso[\UCactor] Ingresa el porcentaje de la evaluación registrada anteriormente con base en la regla \BRref{BR25}{La suma de los porcentajes de cada evaluación debe ser igual a 100\%}.
    
    \UCpaso[\UCactor] Selecciona la evaluación del criterio que ingresó en el paso 4.
    
    \UCpaso[\UCactor] Termina la operación presionando el botón \IUbutton{Guardar}.

    \UCpaso Verifica que todos los campos marcados como obligatorios hayan sido completamente contestados. [Trayectoria C]

    \UCpaso Valida que la suma de los porcentajes sea igual a 100\%. [Trayectoria D]

    \UCpaso Guarda la información de la Evaluación de los Aprendizajes en la base de datos.

    \UCpaso El sistema muestra el mensaje \MSGref{MSG5.}{Registro finalizado exitosamente}.

    \UCpaso[\UCactor] Cierra el mensaje presionando el botón \IUbutton{Aceptar}.

    \UCpaso Muestra la interfaz de usuario \IUref{SP1-IU}{Principal}.
    
\end{UCtrayectoria}

%------------------------ CU TRAYECTORIA ALTERNATIVA A -------------------------

\begin{UCtrayectoriaA}{A}{Uno o más catálogos están vacíos.}
	\UCpaso No encuentra información en los catálogos.
    \UCpaso Muestra el mensaje \MSGref{MSG9.}{Por el momento no se puede realizar el registro.}
    \UCpaso[\UCactor] Cierra el mensaje presionando el botón \IUbutton{Aceptar}.
	\UCpaso Muestra la interfaz de usuario \IUref{SP1-IU}{Principal}.
\end{UCtrayectoriaA}

%------------------------ CU TRAYECTORIA ALTERNATIVA B -------------------------

\begin{UCtrayectoriaA}{B}{El docente quiere agregar más criterios de evaluación.}
    \UCpaso[\UCactor] Quiere agregar más criterios de evaluación para la Unidad de Aprendizaje.
    \UCpaso[\UCactor] Presiona el botón \IUbutton{Agregar Evaluación} que se encuentra debajo del campo ``Porcentaje''.
    \UCpaso Despliega otro campo ``Evaluación'' con su respectivo campo ``Porcentaje''.
    \UCpaso Continua en el paso 5 de la trayectoria principal del \UCref{SP1-CU3}.
\end{UCtrayectoriaA}

%------------------------ CU TRAYECTORIA ALTERNATIVA C -------------------------

\begin{UCtrayectoriaA}{C}{Uno o más campos obligatorios no fueron contestados.}
	\UCpaso Detecta uno o más campos sin contestar.
    \UCpaso Muestra el mensaje \MSGref{MSG4.}{Los campos marcados con (*) son obligatorios.}
    \UCpaso[\UCactor] Cierra el mensaje presionando el botón \IUbutton{Aceptar}.
    \UCpaso Continua en el paso 4 de la trayectoria principal del \UCref{SP1-CU8}.
\end{UCtrayectoriaA}

%------------------------ CU TRAYECTORIA ALTERNATIVA D -------------------------

\begin{UCtrayectoriaA}{D}{La suma de los porcentajes es distinta a 100\%}
    \UCpaso Detecta que la suma de los porcentajes dados por el usuario no cumplen con el 100\% requerido.
    \UCpaso Muestra el mensaje \MSGref{MSG23.}{Los porcentajes de evaluación no cumplen con el porcentaje total obligatorio.}
    \UCpaso[\UCactor] Cierra el mensaje presionando el botón \IUbutton{Aceptar}.
    \UCpaso Continua en el paso 6 de la trayectoria principal del \UCref{SP1-CU8}.
\end{UCtrayectoriaA}