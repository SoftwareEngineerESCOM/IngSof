\chapter{sprint 3 Herramientas para verificar las secciones de propuesta de unidad de Aprendizaje}
\begin{UseCase}{SP2-CU11}{ Agregar comentarios a propuesta de unidad de aprendizaje }{El usuario podrá agregar uno o más comentarios en la sección de propuesta de unidad de aprendizaje que está revisando.}
		\UCitem{Versión}{\color{Gray}1.1}
		\UCitem{Autor}{\color{Gray}Romero Ponce Mauricio Isaac}
		\UCitem{Supervisa}{\color{Gray}Parra Garcilazo Cinthya Dolores}
		\UCitem{Actor}{Analista}
		\UCitem{Propósito}{Asignar cuales son y donde están las correcciones que se deben realizar a la sección de la unidad de aprendizaje que se está revisando.}
		\UCitem{Entradas}{
          \begin{itemize}
          	\item Descripcion del comentario.
          \end{itemize}
        }
		\UCitem{Origen}{teclado.}
		\UCitem{Salidas}{
        	\begin{itemize}
        		\item \MSGref{MSG5}{Registro finalizado exitosamente.}
                \item \MSGref{MSG11}{Error: Debe agregar un texto al nuevo comentario.}
                \item \MSGref{MSG12}{Ingrese su descripción mensaje a modificar en la bitácora.}
                \item \MSGref{MSG25}{Servicios no disponibles.}
        	\end{itemize}
        }
		\UCitem{Destino}{Pantalla.}
		\UCitem{Precondiciones}{ Se llamó el caso de uso SP2-CU5 o SP2-CU6 o SP2-CU7 o SP2-CU8 o  SP2-CU9 o o SP2-CU10}
		\UCitem{Postcondiciones}{
            \begin{itemize}
                \item Se agregará al sistema en la sección de comentario la fecha, la descipción y la sección.
                \item Se mostrará en la bitácora el nuevo comentario.
                \item Se habilita la llamada a los casos de uso SP2-CU12 y SP2-CU13.  
             \end{itemize}  
        }
		\UCitem{Errores}{}
    \UCitem{Puntos de Extensión}
    {
      \begin{itemize}
            \item \UCref{SP2-CU5}: Autorizar Sección Programa Sintético de Unidad de Aprendizaje.
            \item \UCref{SP4-CU6}: Autorizar Sección Programa en Extenso de Unidad de Aprendizaje.
            \item \UCref{SP4-CU7}: Autorizar Sección Relación de Prácticas de Unidad de Aprendizaje.
            \item \UCref{SP4-CU8}: Autorizar Sección Bibliografía de Unidad de Aprendizaje.
            \item \UCref{SP4-CU9}: Autorizar Sección Sistema de Evaluación de Unidad de Aprendizaje.
            \item \UCref{SP4-CU10}: Autorizar Unidad Temática de Unidad de Aprendizaje

        \end{itemize}
    }
		\UCitem{Estado}{Revisión.}
		\UCitem{Observaciones}{}
\end{UseCase}

%--------------------------- CU TRAYECTORIA PRINCIPAL -------------------------
\begin{UCtrayectoria}{Principal}

    \UCpaso[\UCactor] Presiona el botón \IUbutton{Nuevo comentario}. 
    
    \UCpaso Obtiene la fecha de la fecha actual. 
    
    \UCpaso Obtiene el nombre de la sección en la que se está trabajando actualmente.
    
    \UCpaso Muestra el nuevo comentario en la bitácora de comentarios debajo del ultimo comentario generado.
    
    \UCpaso Muestra el mensaje \MSGref{MSG12}.

    \UCpaso[\UCactor] Cierra el mensaje presionando \IUbutton{Aceptar}.
    
    \UCpaso[\UCactor] Ingresa el nuevo comentario en el input text "comentario".
    
    \UCpaso[\UCactor] Presiona el botón \IUbutton{Aceptar}. \hyperlink{SP2-CU11-A}{Trayectoria A}.
    
    \UCpaso Verifica que el campo comentario haya sido contestado. \hyperlink{SP2-CU11-B}{Trayectoria B}.

    \UCpaso Guarda la información del nuevo comentario en el sistema. \hyperlink{SP2-CU11-C}{Trayectoria C}.

    \UCpaso Muestra el  \MSGref{MSG5}.

    \UCpaso[\UCactor] Cierra el mensaje presionando \IUbutton{Aceptar}.

    \UCpaso Desaparecen los botones  \IUbutton{Aceptar} y  \IUbutton{Cancelar} sobre el nuevo comentario generado.

    \UCpaso Aparecen los botones  \IUbutton{Editar} y  \IUbutton{Eliminar} sobre el nuevo comentario generado.

    \UCpaso Se muestra la pantalla que invocó al SP2-CU11 con el nuevo comentario en la bitácora de comentarios. 

\end{UCtrayectoria}


%------------------------ CU TRAYECTORIA ALTERNARIVA A -----------------------
\begin{UCtrayectoriaA}{A}{El usuario presionó \IUbutton{Cancelar}}
  \hypertarget{SP2-CU11-A}{}
	\UCpaso Elimina el comentario generado previamente.
  \UCpaso Se muestra la pantalla actual sin modificaciones en la bitacora de comentarios.
\end{UCtrayectoriaA}

%------------------------ CU TRAYECTORIA ALTERNARIVA B -----------------------
\begin{UCtrayectoriaA}{B}{El sistema detecta que el campo “comentario” se encuentra vacío.} 
  \hypertarget{SP2-CU11-B}{}
	\UCpaso Muestra el \MSGref{MSG11}.
    \UCpaso[\UCactor] Cierra el mensaje presionando \IUbutton{Aceptar}.
    \UCpaso Continúa en el paso 7 de la trayectoria principal del CU-V11.
\end{UCtrayectoriaA}

%------------------------ CU TRAYECTORIA ALTERNARIVA C -----------------------
\begin{UCtrayectoriaA}{C}{No se puede subir al sistema el nuevo comentario} 
  \hypertarget{SP2-CU11-C}{}
  \UCpaso Muestra el \MSGref{MSG25}.
  \UCpaso[\UCactor] Cierra el mensaje presionando \IUbutton{Aceptar}.
\end{UCtrayectoriaA}