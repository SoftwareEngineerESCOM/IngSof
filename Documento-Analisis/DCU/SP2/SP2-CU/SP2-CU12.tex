\begin{UseCase}{SP2-CU12}{ Modificar comentarios a propuesta de Unidad de Aprendizaje }{El usuario podrá modificar los comentarios previamente realizados en la sección de propuesta de Unidad de Aprendizaje que se está revisando.}
		\UCitem{Versión}{\color{Gray}1.1}
		\UCitem{Autor}{\color{Gray}Romero Ponce Mauricio Isaac}
		\UCitem{Supervisa}{\color{Gray}Parra Garcilazo Cinthya Dolores}
		\UCitem{Actor}{Analista}
		\UCitem{Propósito}{Tener oportunidad de modificar el contenido de los comentarios generados previamente en la propuesta de Unidad de Aprendizaje.}
		\UCitem{Entradas}{
          \begin{itemize}
          	\item Descripción del comentario.
           % \item fecha en que se genera el nuevo comentario.
            %\item Identificador unico del analista.
          \end{itemize}
        }
		\UCitem{Origen}{Mouse y teclado.}
		\UCitem{Salidas}{
        	\begin{itemize}
        		\item \MSGref{MSG5}{Registro finalizado exitosamente.}
                \item \MSGref{MSG11}{Error: Debe agregar un texto al nuevo comentario.}
                \item \MSGref{MSG12}{Ingrese su descripción mensaje a modificar en la bitácora.}
                \item \MSGref{MSG25}{Servicios no disponibles.}
        	\end{itemize}
        }
		\UCitem{Destino}{Pantalla.}
		\UCitem{Precondiciones}{ Se llamó al caso de uso SP2-CU11}
		\UCitem{Postcondiciones}{
            \begin{itemize}
                \item Se modificará en el sistema la descripción del comentario.
                \item Se modificará en el sistema la fecha del comentario.
                \item Se mostrará en la bitácora la modificación del comentario.  
             \end{itemize}  
        }
		\UCitem{Errores}{}
        \UCitem{Puntos de Extensión}{

        \begin{itemize}

            \item \UCref{SP2-CU5}: Autorizar Sección Programa Sintético de Unidad de Aprendizaje.

            \item \UCref{SP4-CU6}: Autorizar Sección Programa en Extenso de Unidad de Aprendizaje.

            \item \UCref{SP4-CU7}: Autorizar Sección Relación de Prácticas de Unidad de Aprendizaje.
            \item \UCref{SP4-CU8}: Autorizar Sección Bibliografía de Unidad de Aprendizaje.
            \item \UCref{SP4-CU9}: Autorizar Sección Sistema de Evaluación de Unidad de Aprendizaje.
            \item \UCref{SP4-CU10}: Autorizar Unidad Temática de Unidad de Aprendizaje

        \end{itemize}
        }
		\UCitem{Estado}{Revisión.}
		\UCitem{Observaciones}{}
\end{UseCase}

%--------------------------- CU TRAYECTORIA PRINCIPAL -------------------------
\begin{UCtrayectoria}{Principal}


    \UCpaso[\UCactor] Presiona el botón \IUbutton{Modificar comentario}. 
    
    \UCpaso Obtiene la fecha actual. 
    
    \UCpaso Muestra el mensaje \MSGref{MSG12}{Ingrese su descripción mensaje a modificar en la bitácora.}

    \UCpaso[\UCactor] Cierra el mensaje presionando \IUbutton{Aceptar}.
    
    \UCpaso[\UCactor] Ingresa la nueva descripción en el input text "descripción".
    
    \UCpaso[\UCactor] Presiona el botón \IUbutton{Aceptar}. \hyperlink{SP2-CU12-A}{Trayectoria A}.
    
    \UCpaso Verifica que el campo comentario haya sido contestado. \hyperlink{SP2-CU12-B}{Trayectoria B}.

    \UCpaso Actualiza la fecha del comentario en el sistema al que se presionó el botón \IUbutton{Modificar comentario}.

    \UCpaso Guarda la modificación del comentario y la nueva fecha en el sistema. \hyperlink{SP2-CU12-C}{Trayectoria C}.

    \UCpaso Muestra en la bitácora la nueva descripción y la nueva fecha del comentario en el comentario modificado.

    \UCpaso Muestra el  \MSGref{MSG5}{Registro finalizado exitosamente.}

    \UCpaso[\UCactor] Cierra el mensaje presionando \IUbutton{Aceptar}.

    \UCpaso Se muestra la pantalla que invocó al SP2-CU12 con la modificación del comentario en la bitácora de comentarios. 

\end{UCtrayectoria}

%------------------------ CU TRAYECTORIA ALTERNARIVA A -------------------------

\hypertarget{SP2-CU12-A}{}
\begin{UCtrayectoriaA}{A}{El usuario presionó \IUbutton{Cancelar}}

  \UCpaso Deja el comentario sin modificaciones.
\end{UCtrayectoriaA}

%------------------------ CU TRAYECTORIA ALTERNARIVA B -------------------------
\begin{UCtrayectoriaA}{B}{El sistema detecta que el campo “comentario” se encuentra vacío.} 
    \hypertarget{SP2-CU12-B}{}
    \UCpaso El sistema muestra el mensaje \MSGref{MSG11}{Error: debe agregar un texto al nuevo comentario.}.
    \UCpaso[\UCactor] Cierra el mensaje presionando \IUbutton{Aceptar}.
    \UCpaso Continúa en el paso 5 de la trayectoria principal del SP2-CU12.
\end{UCtrayectoriaA}
%------------------------ CU TRAYECTORIA ALTERNARIVA B -------------------------
\begin{UCtrayectoriaA}{C}{No se puede subir al sistema el nuevo comentario} 
\hypertarget{SP2-CU12-C}{}
  \UCpaso Muestra el \MSGref{MSG25}.
  \UCpaso[\UCactor] Cierra el mensaje presionando \IUbutton{Aceptar}.
\end{UCtrayectoriaA}