\begin{UseCase}{SP2-CU14}{ Gestión de subrayados en revisión de propuesta de Unidad de Aprendizaje }{El usuario administrar los subrayados en el texto de la sección de propuesta de Unidad de Aprendizaje que se está revisando.}
		\UCitem{Versión}{\color{Gray}2.0}
		\UCitem{Autor}{\color{Gray}Romero Ponce Mauricio Isaac}
		\UCitem{Supervisa}{\color{Gray}Parra Garcilazo Cinthya Dolores}
		\UCitem{Actor}{Analista}
		\UCitem{Propósito}{Asignar puntos importantes a revisar por medio de subrayados en la sección de la propuesta de Unidad de Aprendizaje  que se está revisando.}
		\UCitem{Entradas}{Las dos entradas para agregar un subrayado en una sección de la propuesta de Unidad de Aprendizaje son:
          \begin{itemize}
          	\item texto seleccionado en la sección de Unidad de Aprendizaje que se está revisando
           % \item fecha en que se genera el nuevo comentario.
            %\item Identificador unico del analista.
          \end{itemize}
        }
		\UCitem{Origen}{Mouse}
		\UCitem{Salidas}{
        	\begin{itemize}
        		\item \MSGref{MSG50}{Error: Primero se debe seleccionar el punto en donde se gestionará el subrayado.}
            \item \MSGref{MSG25}{Servicios no disponibles.}
        	\end{itemize}
        }
		\UCitem{Destino}{Pantalla.}
		\UCitem{Precondiciones}{ Se llamó al caso de uso SP2-CU5 o SP2-CU6 o SP2-CU7 o SP2-CU8 o  SP2-CU9 o SP2-CU10}
		\UCitem{Postcondiciones}{
            \begin{itemize}
                \item Se agregará o eliminará al sistema el subrayado.
                \item Se mostrará o quitará  el subrayado en amarillo.
             \end{itemize}  
        }
		\UCitem{Errores}{}
    \UCitem{Puntos de Extensión}{

        \begin{itemize}

            \item \UCref{SP2-CU5}: Autorizar Sección Programa Sintético de Unidad de Aprendizaje.
            \item \UCref{SP4-CU6}: Autorizar Sección Programa en Extenso de Unidad de Aprendizaje.
            \item \UCref{SP4-CU7}: Autorizar Sección Relación de Prácticas de Unidad de Aprendizaje.
            \item \UCref{SP4-CU8}: Autorizar Sección Bibliografía de Unidad de Aprendizaje.
            \item \UCref{SP4-CU9}: Autorizar Sección Sistema de Evaluación de Unidad de Aprendizaje.
            \item \UCref{SP4-CU10}: Autorizar Unidad Temática de Unidad de Aprendizaje

        \end{itemize}
        }
		\UCitem{Estado}{Revisión.}
		\UCitem{Observaciones}{}
\end{UseCase}

%--------------------------- CU TRAYECTORIA PRINCIPAL -------------------------
\begin{UCtrayectoria}{Principal}

    \UCpaso[\UCactor] Selecciona con el mouse el texto donde se gestionará un subrayado de texto. \hyperlink{SP2-CU14-A}{Trayectoria A}. \hyperlink{SP2-CU14-B}{Trayectoria B}.

\end{UCtrayectoria}

%------------------------ CU TRAYECTORIA ALTERNARIVA A -------------------------
\begin{UCtrayectoriaA}{A}{El usuario presionó \IUbutton{Subrayar}}
  \hypertarget{SP2-CU14-A}{}
    \UCpaso verifica que se haya hecho una selección de texto \hyperlink{SP2-CU14-A1}{Trayectoria A1}.
    \UCpaso Toma la sección del documento donde se desea agregar un subrayado.
    \UCpaso Agrega del sistema por medio del caracter especial !!¡¡ donde inicia el subrayado y en donde termina el subrayado.\hyperlink{SP2-CU14-C}{Trayectoria C}.
    \UCpaso muestra la parte seleccionada con un rasaltado amarillo en el texto
     \UCpaso Se muestra la pantalla que invocó al SP2-CU14 con el subrayado en el texto.

\end{UCtrayectoriaA}

%------------------------ CU TRAYECTORIA ALTERNARIVA A1 -------------------------
\begin{UCtrayectoriaA}{A1}{El usuario no seleccionó alguna parte del texto.}
  \hypertarget{SP2-CU14-A1}{}
  \UCpaso Detecta que no hay texto seleccionado para la gestión.

  \UCpaso El sistema muestra el \MSGref{MSG48}{Error: Primero se debe seleccionar el punto en donde se gestionará el subrayado}.

  \UCpaso[\UCactor] Cierra el mensaje presionando \IUbutton{Aceptar}.

\end{UCtrayectoriaA}

%------------------------ CU TRAYECTORIA ALTERNARIVA B -------------------------
\begin{UCtrayectoriaA}{B}{El usuario presionó \IUbutton{Quitar subrayado}}
\hypertarget{SP2-CU14-B}{}
    \UCpaso verifica que se haya hecho una selección de texto \hyperlink{SP2-CU14-A1}{Trayectoria A1}.
    \UCpaso Toma la sección del documento donde se desea eliminar un subrayado.
    \UCpaso Elimina el caracter especial !!¡¡ en el texto guardado en el sistema donde inicia el subrayado y en donde termina el subrayado. \hyperlink{SP2-CU14-C}{Trayectoria C}.
    \UCpaso Quita la parte rasaltada amarillo en el previo subrayado.
     \UCpaso Se muestra la pantalla que invocó al SP2-CU14 sin el subrayado en el texto eliminado.

\end{UCtrayectoriaA}

%------------------------ CU TRAYECTORIA ALTERNARIVA B -------------------------
\begin{UCtrayectoriaA}{C}{No se pudo agregar o eliminar el subrayado en el sistema}
\hypertarget{SP2-CU14-C}{}
  \UCpaso Muestra el mensaje \MSGref{MSG25}
  \UCpaso[\UCactor] Cierra el mensaje presionando \IUbutton{Aceptar}.
\end{UCtrayectoriaA}