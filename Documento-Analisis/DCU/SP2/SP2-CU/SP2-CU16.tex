\chapter{SP2-CU16 Eliminar comentarios en sección de propuesta de Unidad de Aprendizaje}
\begin{UseCase}{SP2-CU16}{ Eliminar comentarios en sección de propuesta de Unidad de Aprendizaje }{El usuario podrá eliminar los comentarios previamente realizados en la sección de propuesta de Unidad de Aprendizaje que se está revisando.}
		\UCitem{Versión}{\color{Gray}1.0}
		\UCitem{Autor}{\color{Gray}Romero Ponce Mauricio Isaac}
		\UCitem{Supervisa}{\color{Gray}Parra Garcilazo Cinthya Dolores}
		\UCitem{Actor}{Analista}
		\UCitem{Propósito}{Elminar los comentarios que tienen contenido con errores o confuso que pueda causar una confusión al momento de corregir.}
		\UCitem{Entradas}{}
		\UCitem{Origen}{Mouse.}
		\UCitem{Salidas}{
        	\begin{itemize}
        		\hypertarget{CU16-MSG1}{\item MSG1 La operación se ha realizado con éxito.}
            \hypertarget{CU16-MSG13}{\item MSG13 Error interno: Intentelo más tarde.}
            \hypertarget{CU16-MSG14}{\item MSG14 ¿Está seguro que desea eliminar el comentario?}
        	\end{itemize}
        }
		\UCitem{Destino}{Pantalla.}
		\UCitem{Precondiciones}{ Se llamó al caso de uso SP2-CU14}
		\UCitem{Postcondiciones}{
            \begin{itemize}
                \item Se eliminará del sistema el comentario.
                \item Desaparecerá de la vizualización en la bitácora el comentario eliminado.  
             \end{itemize}  
        }
		\UCitem{Errores}{}
		\UCitem{Estado}{Revisión.}
		\UCitem{Observaciones}{}
\end{UseCase}

%--------------------------- CU TRAYECTORIA PRINCIPAL -------------------------
\begin{UCtrayectoria}{Principal}


    \UCpaso[\UCactor] Presiona el botón \IUbutton{Eliminar comentario}. 
    
    \UCpaso Muestra el mensaje \MSGref{CU16-MSG14}{¿Está seguro que desea eliminar el comentario?}.

    \UCpaso[\UCactor] Cierra el mensaje presionando \IUbutton{Aceptar}. \hyperref[SP2-CU16-A]{Trayectoria A}.
    
    \UCpaso Elimina el comentario del sistema. \hyperref[SP2-CU16-B]{Trayectoria B}.
    
    \UCpaso Muestra el \MSGref{CU16-MSG1}{La operación se ha realizado con éxito.}.

    \UCpaso[\UCactor] Cierra el mensaje presionando \IUbutton{Aceptar}.


\end{UCtrayectoria}

%------------------------ CU TRAYECTORIA ALTERNARIVA A -------------------------

\label{SP2-CU16-A}
\begin{UCtrayectoriaA}{A}{El usuario presionó \IUbutton{Cancelar}}
  \UCpaso No realiza ninguna modificación en el sistema o en la bitácora.
\end{UCtrayectoriaA}

%------------------------ CU TRAYECTORIA ALTERNARIVA A -------------------------
\label{SP2-CU16-B}
\begin{UCtrayectoriaA}{B}{No se pudo eliminar el comentario en el sistema}
  \UCpaso Muestra el mensaje \MSGref{CU16-MSG13}{Error interno: Intentelo más tarde}.
  \UCpaso[\UCactor] Cierra el mensaje presionando \IUbutton{Aceptar}.
\end{UCtrayectoriaA}


\chapter{Pantallas}
 \begin{figure}
  \centering
    \includegraphics[width=0.7\textwidth]{DCU/SP2/Pantallas/Nuevo_comentario}
  \caption{SP2-IU-Nuevo comentario}
  \label{SP2-IU-Nuevo_comentario}
\end{figure}
