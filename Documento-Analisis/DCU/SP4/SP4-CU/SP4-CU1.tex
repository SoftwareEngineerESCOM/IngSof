\begin{UseCase}{SP4-CU1}{Registrar Programa Académico}{El usuario Jefe de Innovación Educativa ingresa los datos de un Programa Académico.}
		\UCitem{Versión}{\color{Gray}1.0}
		\UCitem{Autor}{\color{Gray}Plata García Josué Eliasaf}
		\UCitem{Supervisa}{\color{Gray}Evelyn Reyes}
		\UCitem{Actor}{\hyperlink{Usuario}{Jefe de Innovación Educativa}}
		\UCitem{Propósito}{Registrar el nombre del Programa Académico en el sistema.}
		\UCitem{Entradas}{Las entradas para el registro de Programa Académico:
          \begin{itemize}
          	\item Nombre
          \end{itemize}
        }
		\UCitem{Origen}{Teclado.}
		\UCitem{Salidas}{
        	\begin{itemize}

                \item \MSGref{MSG5}{Registro finalizado exitosamente.}
                \item \MSGref{MSG25}{Servicios no disponibles por el momento.}
                \item \MSGref{MSG29}{¿Está seguro que desea cancelar? Se perderán todos los avances sin guardar.}
               	\item \MSGref{MSG44}{Este campo es requerido}
                \item \MSGref{MSG35}{Escribe información válida.}
                \item \MSGref{MSG25}{Servicios no disponibles por el momento.}
                \item \MSGref{MSG43}{El nombre del Programa Académico ya está registrado porfavor cámbielo.}

        	\end{itemize}
        }
		\UCitem{Destino}{Pantalla}
		\UCitem{Precondiciones}{ \begin{itemize}
                \item El nombre del Programa Académico a registrar, no debe existir previamente en el sistema.
            \end{itemize}}
		\UCitem{Postcondiciones}{El Programa Académico quedó registrado en el sistema, permitiendo consultarlo y generar tareas de registro del mismo.}
		\UCitem{Errores}{
            Ninguno
        }
		\UCitem{Estado}{Revisión.}
		\UCitem{Observaciones}{Ninguna}
\end{UseCase}
%--------------------------- CU TRAYECTORIA PRINCIPAL -------------------------
\begin{UCtrayectoria}{Principal}
    \UCpaso[\UCactor] Presiona el botón \IUbutton{(+)Registrar Programa Académico}de la interfaz de usuario \IUref{GPA-J}{Gestionar Programas Académicos}Existen las trayectorias alternativas [Trayectoria F] al presionar el boton Registrar.

    \UCpaso Muestra la interfaz de usuario \IUref{RPA-J}{Registrar Programa Académico}.

    \UCpaso[\UCactor] Ingresa el nombre del Programa Académico.

    \UCpaso[\UCactor] Termina la operación presionando el botón \IUbutton{Registrar}. Existen las trayectorias alternativas [Trayectoria A] y [Trayectoria B] al presionar el boton Cancelar.

    \UCpaso Verifica que se cumpla la \BRref{BR13}{Todos los campos son obligatorios}, \BRref{BR19}{El nombre del Programa Académico es único.}, \BRref{BR31}{Información de Programa Académico.} y la \BRref{BR32}{Longitud máxima de nombre de Programa Académico.}. [Trayectoria C][Trayectoria D][Trayectoria E][Trayectoria H]

    \UCpaso Persiste los datos ingresados.Accionado por el boton \IUbutton{Registrar} [Trayectoria G] activada al presionar el boton.

    \UCpaso Muestra el mensaje \MSGref{MSG5}{Registro finalizado exitosamente}.

    \UCpaso[\UCactor] Cierra el mensaje presionando el botón \IUbutton{Aceptar}.
    \UCpaso Muestra la interfaz de usuario \IUref{GPA-J}{Gestionar Programas Académicos}.
\end{UCtrayectoria}

%------------------------ CU TRAYECTORIA ALTERNARIVA A -------------------------
\begin{UCtrayectoriaA}{A}{El actor presiona el botón \IUbutton{Cancelar}.}
    \UCpaso Muestra el mensaje \MSGref{MSG29}{¿Está seguro que desea cancelar? Se perderán todos los avances sin guardar.}.
    \UCpaso[\UCactor] Confirma la operación presionando el botón \IUbutton{Si}.
    \UCpaso Muestra la interfaz de usuario \IUref{GPA-J}{Gestionar Programas Académicos}
\end{UCtrayectoriaA}
%------------------------ CU TRAYECTORIA ALTERNARIVA B -------------------------
\begin{UCtrayectoriaA}{B}{El actor presiona accidentalmente el botón \IUbutton{Cancelar}.}
    \UCpaso Muestra el mensaje \MSGref{MSG29}{¿Está seguro que desea cancelar? Se perderán todos los avances sin guardar.}.
    \UCpaso[\UCactor] Presiona el botón \IUbutton{No}.
    \UCpaso Cierra el mensaje.
    \UCpaso Continúa en el paso 3 de la trayectoria principal del \UCref{SP4-CU1}.
\end{UCtrayectoriaA}
%------------------------ CU TRAYECTORIA ALTERNARIVA C -------------------------
\begin{UCtrayectoriaA}{C}{El sistema detecta uno o más campos sin contestar.}
    \UCpaso Muestra el mensaje \MSGref{MSG44}{Este campo es requerido} debajo de cada campo obligatorio sin contestar.
    \UCpaso Continúa en el paso 3 de la trayectoria principal del \UCref{SP4-CU1}.
\end{UCtrayectoriaA}
%------------------------ CU TRAYECTORIA ALTERNARIVA D -------------------------
\begin{UCtrayectoriaA}{D}{El Sistema detecta que el Nombre de Programa Académico no cumple con la expresión regular}
    \UCpaso Muestra el mensaje \MSGref{MSG35}{Escribe información válida} debajo del campo.
    \UCpaso Continúa en el paso 3 de la trayectoria principal del \UCref{SP4-CU1}.
\end{UCtrayectoriaA}
    %------------------------ CU TRAYECTORIA ALTERNARIVA E -------------------------

\begin{UCtrayectoriaA}{E}{El nombre del Programa Académico ya está registrado.}
    \UCpaso Muestra el mensaje \MSGref{MSG43}{El nombre del Programa Académico ya está registrado porfavor cámbielo.}.
    \UCpaso[\UCactor] Cierra el mensaje presionando el botón \IUbutton{Aceptar}.
\UCpaso Muestra la interfaz de usuario Menú para Jefe de Innovación Educativa.
\end{UCtrayectoriaA}
    %------------------------ CU TRAYECTORIA ALTERNARIVA F -------------------------

\begin{UCtrayectoriaA}{F}{Problemas con la Conexión}
    \UCpaso Muestra el mensaje \MSGref{MSG25}{ Servicios no disponibles por el momento.}.
    \UCpaso[\UCactor] Cierra el mensaje presionando el botón \IUbutton{Aceptar}.
\UCpaso Muestra la interfaz de usuario Menú para Jefe de Innovación Educativa.
\end{UCtrayectoriaA}

%------------------------ CU TRAYECTORIA ALTERNARIVA G -------------------------
\begin{UCtrayectoriaA}{G}{Ocurre un error al momento de persistir los datos.}
    \UCpaso Muestra el mensaje \MSGref{MSG25}{Servicios no disponibles por el momento.}
    \UCpaso[\UCactor] Cierra el mensaje presionando el botón \IUbutton{Aceptar}.
    \UCpaso Muestra la interfaz de usuario \IUref{GPA-J}{Gestionar Programas Académicos}.
\end{UCtrayectoriaA}
%------------------------ CU TRAYECTORIA ALTERNARIVA H -------------------------
\begin{UCtrayectoriaA}{H}{El Sistema detecta que el Nombre de Programa Académico excede la longitud máxima}
    \UCpaso No permite al usuario superar la longitud establecida en la \BRref{Longitud máxima de nombre de Programa Académico}.
    \UCpaso Continúa en el paso 3 de la trayectoria principal del \UCref{SP4-CU1}.
\end{UCtrayectoriaA}