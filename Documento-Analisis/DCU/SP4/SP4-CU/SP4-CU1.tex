\begin{UseCase}{SP4-CU1}{Registrar Programa Académico}{El usuario Jefe de Innovación Educativa ingresa los datos de un Programa Académico.}
        \UCitem{Versión}{\color{Gray}1.0}
        \UCitem{Autor}{\color{Gray}Plata García Josué Eliasaf}
        \UCitem{Supervisa}{\color{Gray}Evelyn Reyes}
        \UCitem{Actor}{\hyperlink{Usuario}{Jefe de Innovación Educativa,Jefe de División de Innovación Académica y Jefe de Departamento de Desarrollo e Innovación Curricular}}
        \UCitem{Propósito}{Registrar el nombre del Programa Académico en el sistema.}
        \UCitem{Entradas}{Las entradas para el registro de Programa Académico:
          \begin{itemize}
            \item Nombre
          \end{itemize}
        }
        \UCitem{Origen}{Teclado.}
        \UCitem{Salidas}{
            \begin{itemize}
                \item \MSGref{MSG5}{Registro finalizado exitosamente.}
                \item \MSGref{MSG25}{Servicios no disponibles.}
                \item \MSGref{MSG29}{¿Está seguro que desea cancelar? Se perderán todos los avances sin guardar.}
                \item \MSGref{MSG44}{Este campo es requerido}
                \item \MSGref{MSG35}{Escribe información válida.}
                \item \MSGref{MSG43}{El nombre del Programa Académico ya está registrado porfavor cámbielo.}
            \end{itemize}
        }
        \UCitem{Destino}{Pantalla}
        \UCitem{Precondiciones}{ \begin{itemize}
                \item El nombre del Programa Académico a registrar, no debe existir previamente en el sistema.
            \end{itemize}}
        \UCitem{Postcondiciones}{El Programa Académico quedó registrado en el sistema, permitiendo consultarlo y generar tareas de registro del mismo.}
        \UCitem{Errores}{
            Ninguno
        }
        \UCitem{Estado}{Revisión.}
        \UCitem{Observaciones}{Ninguna}
\end{UseCase}
%--------------------------- CU TRAYECTORIA PRINCIPAL -------------------------
\begin{UCtrayectoria}{Principal}
    \UCpaso[\UCactor] Presiona el botón \IUbutton{(+)} que es Registrar Programa Académico de la interfaz de usuario \IUref{GPA-J}{Gestionar Programas Académicos}.Existen las trayectorias alternativas [Trayectoria F] al presionar el botón \IUbutton{(+)}.
    \UCpaso Muestra la interfaz de usuario \IUref{RPA-J}{Registrar Programa Académico}.
    \UCpaso[\UCactor] Ingresa el nombre del Programa Académico, el sistema verifica conforme al modelo de datos, la \BRref{BR38}{Verificación de formularios al momento} y la \BRref{BR39}{Todos los campos marcados con (*) son obligatorios}.[Trayectoria A][Trayectoria B]
    \UCpaso[\UCactor] Termina la operación presionando el botón \IUbutton{Registrar}. Existen las trayectorias alternativas [Trayectoria C] y [Trayectoria C.1] al presionar el botón Cancelar.
    \UCpaso Verifica que se cumpla la \BRref{BR13}{Todos los campos son obligatorios} y \BRref{BR19}{El nombre del Programa Académico es único.}. [Trayectoria A][Trayectoria D]
    \UCpaso Persiste los datos ingresados.Accionando el botón \IUbutton{Registrar} se activa la  [Trayectoria E].
    \UCpaso Muestra el mensaje \MSGref{MSG5}{Registro finalizado exitosamente}.
    \UCpaso[\UCactor] Cierra el mensaje presionando el botón \IUbutton{Aceptar}.
    \UCpaso Muestra la interfaz de usuario \IUref{GPA-J}{Gestionar Programas Académicos}.
\end{UCtrayectoria}

%------------------------ CU TRAYECTORIA ALTERNARIVA A -------------------------

\begin{UCtrayectoriaA}{A}{El sistema detecta uno o más campos sin contestar.}
    \UCpaso Muestra el mensaje \MSGref{MSG44}{Este campo es requerido} debajo de cada campo obligatorio sin contestar.
    \UCpaso Continúa en el paso 3 de la trayectoria principal del \UCref{SP4-CU1}.
\end{UCtrayectoriaA}

%------------------------ CU TRAYECTORIA ALTERNARIVA B -------------------------

\begin{UCtrayectoriaA}{B}{El Sistema detecta que el Nombre de Programa Académico no cumple con el diccionario de datos}
    \UCpaso Muestra el mensaje \MSGref{MSG35}{Escribe información válida} debajo del campo.
    \UCpaso Continúa en el paso 3 de la trayectoria principal del \UCref{SP4-CU1}.
\end{UCtrayectoriaA}
%------------------------ CU TRAYECTORIA ALTERNARIVA C -------------------------
\begin{UCtrayectoriaA}{C}{El actor presiona el botón \IUbutton{Cancelar}.}
    \UCpaso Muestra el mensaje \MSGref{MSG29}{¿Está seguro que desea cancelar? Se perderán todos los avances sin guardar.}.
    \UCpaso[\UCactor] Confirma la operación presionando el botón \IUbutton{Si}.
    \UCpaso Muestra la interfaz de usuario \IUref{GPA-J}{Gestionar Programas Académicos}
\end{UCtrayectoriaA}

%------------------------ CU TRAYECTORIA ALTERNARIVA C.1 -------------------------
\begin{UCtrayectoriaA}{C.1}{El actor presiona accidentalmente el botón \IUbutton{Cancelar}.}
    \UCpaso Muestra el mensaje \MSGref{MSG29}{¿Está seguro que desea cancelar? Se perderán todos los avances sin guardar.}.
    \UCpaso[\UCactor] Presiona el botón \IUbutton{No}.
    \UCpaso Cierra el mensaje.
    \UCpaso Continúa en el paso 4 de la trayectoria principal del \UCref{SP4-CU1}.
\end{UCtrayectoriaA}


    %------------------------ CU TRAYECTORIA ALTERNARIVA D -------------------------
\begin{UCtrayectoriaA}{D}{El nombre del Programa Académico ya está registrado.}
    \UCpaso Muestra el mensaje \MSGref{MSG43}{El nombre del Programa Académico ya está registrado.Por favor cámbielo.}.
    \UCpaso[\UCactor] Cierra el mensaje presionando el botón \IUbutton{Aceptar}.
    \UCpaso Continúa en el paso 3 de la trayectoria principal del \UCref{SP4-CU1}.
\end{UCtrayectoriaA}

%------------------------ CU TRAYECTORIA ALTERNARIVA E -------------------------
\begin{UCtrayectoriaA}{E}{Ocurre un error al momento de persistir los datos.}
    \UCpaso Muestra el mensaje \MSGref{MSG25}{Servicios no disponibles.}
    \UCpaso[\UCactor] Cierra el mensaje presionando el botón \IUbutton{Aceptar}.
\end{UCtrayectoriaA}
%------------------------ CU TRAYECTORIA ALTERNARIVA D -------------------------
\begin{UCtrayectoriaA}{F}{Problemas con la Conexión}
    \UCpaso Muestra el mensaje \MSGref{MSG25}{ Servicios no disponibles.}.
    \UCpaso[\UCactor] Cierra el mensaje presionando el botón \IUbutton{Aceptar}.
\end{UCtrayectoriaA}