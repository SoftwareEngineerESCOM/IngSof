% EDITAR PLAN DE ESTUDIOS.
\begin{UseCase}{SP4-CU10}{Editar Plan de Estudio}{El usuario Jefe de Innovación modifica los datos del Plan de Estudios registrado en el sistema.}
	\UCitem{Versión}{\color{Gray}3.0}
	\UCitem{Autor}{\color{Gray}Cervantes Moreno Christian Andres}
	\UCitem{Supervisa}{\color{Gray} Evelyn Reyes}
	\UCitem{Actor}{\hyperlink{Usuario}{Jefe de Innovación}}
	\UCitem{Propósito}{Editar el año, modalidad, créditos totales TEPIC, créditos totales SATCA, Total Horas/Teoría, Total Horas/Práctica, del plan de estudio registrado en el sistema.}
	\UCitem{Entradas}{Las entradas para la modificación del Plan de Estudios son:
		\begin{itemize}
			\item Año.
			\item Modalidad.
			\item Créditos Totales TEPIC.
			\item Créditos Totales SATCA.
			\item Total horas / Teoría.
			\item Total horas / Práctica.
		\end{itemize}
	}
	\UCitem{Origen}{Teclado, Mouse}
	\UCitem{Salidas}{
		\begin{itemize}
			\item \MSGref{MSG25}{Servicios no disponibles.}
            \item \MSGref{MSG29}{¿Está seguro que desea cancelar? Se perderán todos los avances sin guardar.}
            \item \MSGref{MSG31}{Los cambios se guardaron exitosamente.}
            \item \MSGref{MSG35}{Escriba información válida.}
            \item \MSGref{MSG44}{Este campo es requerido.}
            \item \MSGref{MSG54}{Créditos TEPIC fuera de rango}.
		\end{itemize}
	}
	\UCitem{Destino}{Pantalla.}
	\UCitem{Precondiciones}{ Debe de existir al menos un Plan de Estudios registrado en el sistema.}
	\UCitem{Postcondiciones}{El plan de Estudios queda modificado en el sistema.}
	\UCitem{Errores}{
		\begin{itemize}
			\item E1. El catálogo de modalidad no se cargó correctamente.
			\item E2. Hubo un problema al conectarse con la base de datos.
		\end{itemize}
	}
	\UCitem{Estado}{Revisión.}
	\UCitem{Observaciones}{}
\end{UseCase}
%--------------------------- CU TRAYECTORIA PRINCIPAL -------------------------
\begin{UCtrayectoria}{Principal}
	%Usuario
	\UCpaso[\UCactor] Presiona el botón \BtnLapiz de la Interfaz de usuario \IUref{GPE-J}{Gestionar Planes de Estudios} considerando la \BRref{BR15}{Planes de Estudios en re diseño por Porgrama Académico} .
	\UCpaso Carga el catálogo de modalidad descrito en la \BRref{BR14}{Catálogos del Sistema}.[Trayectoria A]
	\UCpaso Muestra la interfaz de usuario \IUref{EPE-J}{Editar Plan de Estudios}.
	\UCpaso[\UCactor] Modifica los campos deseados.[Trayectoria F] [Trayectoria C] [Trayectoria D]
	\UCpaso[\UCactor] Termina la operación presionando el botón \IUbutton{Finalizar}. [Trayectoria B] [Trayectoria B.1]
	\UCpaso Verifica que se cumpla con el diccionario de datos. [Trayectoria C] [Trayectoria D]
	\UCpaso Persiste los datos ingresados. [Trayectoria E]
	\UCpaso El sistema muestra el mensaje \MSGref{MSG31}{Los cambios se guardaron exitosamente}.
	\UCpaso[\UCactor] Cierra el mensaje presionando el botón \IUbutton{Aceptar}.
	\UCpaso Muestra la interfaz de usuario \IUref{GPE-J}{Gestionar Planes de Estudios}.
\end{UCtrayectoria}
%------------------------ CU TRAYECTORIA ALTERNARIVA A -------------------------
\begin{UCtrayectoriaA}{A}{ Los catálogos de la \BRref{BR14}{Catálogos del Sistema} necesarios no se pudieron cargar.}
    \UCpaso Muestra el mensaje \MSGref{MSG25}{Servicios no disponibles. }
    \UCpaso[\UCactor] Cierra el mensaje presionando el botón \IUbutton{Ok}.
    \UCpaso Muestra la interfaz de usuario \IUref{EPE-J}{Editar Plan de Estudios}.
\end{UCtrayectoriaA}

%------------------------ CU TRAYECTORIA ALTERNARIVA B -------------------------
\begin{UCtrayectoriaA}{B}{El actor presiona el botón \IUbutton{Cancelar}.}
	\UCpaso Muestra el mensaje \MSGref{MSG29}{¿Está seguro que desea cancelar? Se perderán todos los avances sin guardar}.
	\UCpaso[\UCactor] Confirma la operación presionando el botón \IUbutton{Si}.
	\UCpaso Muestra la interfaz de usuario \IUref{GPE-J}{Gestionar Planes de Estudio}
\end{UCtrayectoriaA}

%------------------------ CU TRAYECTORIA ALTERNARIVA B.1 -------------------------
\begin{UCtrayectoriaA}{B.1}{El actor presiona accidentalmente el botón \IUbutton{Cancelar}.}
	\UCpaso[\UCactor] Presiona el botón \IUbutton{Cancelar}
	\UCpaso Muestra el mensaje \MSGref{MSG29}{¿Está seguro que desea cancelar? Se perderán todos los avances sin guardar}.
	\UCpaso[\UCactor] Presiona el botón \IUbutton{No}.
	\UCpaso Cierra el mensaje.
	\UCpaso Continúa en el paso 4 de la trayectoria principal del \UCref{SP4-CU10}.
\end{UCtrayectoriaA}

%------------------------ CU TRAYECTORIA ALTERNARIVA C -------------------------
\begin{UCtrayectoriaA}{C}{El sistema detecta caracteres no numéricos.}
	\UCpaso Muestra el mensaje \MSGref{MSG35}{Escriba información válida} debajo del campo que incumplió con el diccionario de datos.
	\UCpaso Continúa en el paso 4 de la trayectoria principal del \UCref{SP4-CU10}.
\end{UCtrayectoriaA}

%------------------------ CU TRAYECTORIA ALTERNARIVA D -------------------------
\begin{UCtrayectoriaA}{D}{El sistema detecta uno o más campos sin contestar.}
	\UCpaso Muestra el mensaje \MSGref{MSG44}{Este campo es requerido} debajo de cada campo obligatorio sin contestar.
	\UCpaso Continúa en el paso 4 de la trayectoria principal del \UCref{SP4-CU10}.
\end{UCtrayectoriaA}

%------------------------ CU TRAYECTORIA ALTERNARIVA E -------------------------
\begin{UCtrayectoriaA}{E}{Ocurre un error al momento de persistir los datos.}
	\UCpaso Muestra el mensaje \MSGref{MSG25}{Servicios no disponibles}.
	\UCpaso[\UCactor] Cierra el mensaje presionando el botón \IUbutton{Aceptar}.
	\UCpaso Muestra la interfaz de usuario \IUref{EPE-J}{Editar Plan de Estudios}.
\end{UCtrayectoriaA}
%------------------------ CU TRAYECTORIA ALTERNARIVA F -------------------------
\begin{UCtrayectoriaA}{F}{El sistema detecta incumplimiento de la \BRref{BR20}{Los créditos totales TEPIC deben de estar entre 350 y 450 créditos} .}
	\UCpaso Muestra el mensaje \MSGref{MSG54}{Créditos TEPIC fuera de rango}.
	\UCpaso[\UCactor] Cierra el mensaje presionando el botón \IUbutton{Aceptar}.
	\UCpaso Muestra la interfaz de usuario \IUref{EPE-J}{Editar Plan de Estudios}.
\end{UCtrayectoriaA}



