% EDITAR PLAN DE ESTUDIOS.
\begin{UseCase}{SP4-CU10}{Editar Plan de Estudios}{El usuario Docente modifica los datos del Plan de Estudios registrado en el sistema.}
		\UCitem{Versión}{\color{Gray}1.0}
		\UCitem{Autor}{\color{Gray}Cervantes Moreno Christian Andres}
		\UCitem{Supervisa}{\color{Gray} Evelyn Reyes}
		\UCitem{Actor}{\hyperlink{Usuario}{Docente}}
		\UCitem{Propósito}{Editar el año, modalidad, créditos totales TEPIC, créditos totales SATCA, Total Horas/Teoría, Total Horas/Práctica, del plan de estudio registrado en el sistema.}
		\UCitem{Entradas}{Las entradas para la modificación de la Unidad de Aprendizaje serán:
          \begin{itemize}
          	\item Año (Entero).
          	\item Modalidad.
          	\item Créditos Totales TEPIC (Tipo double).
            \item Créditos Totales SATCA (Tipo double).
            \item Total horas / Teoría (Tipo entero).
            \item Total horas / Práctica (Tipo entero).
          \end{itemize}
        }
		\UCitem{Origen}{Teclado, Mouse}
		\UCitem{Salidas}{
        	\begin{itemize}
        		\item MSG1. Todos los campos son obligatorios
                \item MSG2. Plan de Estudios modificado exitosamente.
                \item MSG3. ¿Seguro que desea cancelar?.
                \item MSG4. Los catálogos necesarios no están disponibles por el momento, favor de intentarlo más tarde.
                \item MSG5. Entrada no válida.

        	\end{itemize}
        }
		\UCitem{Destino}{Pantalla.}
		\UCitem{Precondiciones}{ Debe de existir al menos un Plan de Estudios registrado en el sistema.}
		\UCitem{Postcondiciones}{El plan de Estudios queda modificado en el sistema.}
		\UCitem{Errores}{
			  \begin{itemize}
				\item E1. Entrada inválida de un caracter no entero.
				\item E2. Entrada inválida de un caracter no double.
			\end{itemize}
		}
		\UCitem{Estado}{Revisión.}
		\UCitem{Observaciones}{}
\end{UseCase}

%--------------------------- CU TRAYECTORIA PRINCIPAL -------------------------
\begin{UCtrayectoria}{Principal}

    %Usuario
    \UCpaso[\UCactor] Presiona el botón Editar de la Interfaz de usuario \IUref{IU2.1-D}{Consultar Mapa Curricular}.

	%Sistema
    \UCpaso El sistema carga el catálogo del Plan de Estudios.[Trayectoria A]


    \UCpaso Muestra la interfaz de usuario \IUref{IU2.1.1-D}{Registrar Plan de Estudios}.
    \UCpaso[\UCactor] Elige los campos que desea modificar.[Trayectoria B].
    \UCpaso[\UCactor] Termina la operación presionando el botón \IUbutton{Guardar}. [Trayectoria B] [Trayectoria C]
    \UCpaso Verifica que todos los campos marcados como obligatorios hayan sido completamente contestados. [Trayectoria D]

    \UCpaso Guarda la información de la Unidad de Aprendizaje en la base de datos.

    \UCpaso El sistema muestra el mensaje \MSGref{MSG2}{Plan de Estudios modificado exitosamente.}.

    \UCpaso[\UCactor] Cierra el mensaje presionando el botón \IUbutton{Aceptar}.

    \UCpaso Muestra la interfaz de usuario \IUref{IU2.1-D}{Consultar Mapa Curricular}.
\end{UCtrayectoria}

%------------------------ CU TRAYECTORIA ALTERNARIVA A -------------------------

\begin{UCtrayectoriaA}{A}{El sistema no carga el catálogo del Plan de Estudios}
	%\UCpaso[\UCactor] Presiona el botón \IUbutton{$\bigoplus$} que se encuentra a un lado del campo ``Nombre'' del formulario \IUref{IU2}{Registro de bibliografía}  [Trayectoria A.1]
	\UCpaso[\UCactor] Presiona el botón \IUbutton{registrar}
	\UCpaso Muestra el mensaje \MSGref{MSG4}{.Los catálogos necesarios no están disponibles por el momento, favor de intentarlo más tarde. }
	\UCpaso[\UCactor] Confirma la operación presionando el botón \IUbutton{Ok}.
	 \UCpaso Muestra la interfaz de usuario \IUref{IU2.1-D}{Consultar Mapa Curricular}

\end{UCtrayectoriaA}

%------------------------ CU TRAYECTORIA ALTERNARIVA B.1 -------------------------

\begin{UCtrayectoriaA}{B.1}{El usuario desea modificar el año del Plan de Estudios.}
	\UCpaso[\UCactor] Borra el año del Plan de Estudios.
	\UCpaso[\UCactor] Ingresa el nuevo año del Plan de Estudios.
	\UCpaso Continúa en el paso 5 de la trayectoria principal del \UCref{SP4-CU10}.
\end{UCtrayectoriaA}

%------------------------ CU TRAYECTORIA ALTERNARIVA B.2 -------------------------

\begin{UCtrayectoriaA}{B.2}{El usuario desea modificar los créditos TEPIC del Plan de Estuidos.}
	\UCpaso[\UCactor] Borra los créditos TEPIC del Plan de Estudios.
	\UCpaso[\UCactor] Ingresa los nuevos créditos TEPIC del Plan de Estudios.
	\UCpaso Continúa en el paso 5 de la trayectoria principal del \UCref{SP4-CU10}.
\end{UCtrayectoriaA}


%------------------------ CU TRAYECTORIA ALTERNARIVA B.3 -------------------------

\begin{UCtrayectoriaA}{B.3}{El usuario desea modificar los créditos TEPIC de la Unidad de Aprendizaje.}
	\UCpaso[\UCactor] Borra los créditos SATCA del Plan de Estudios.
	\UCpaso[\UCactor] Ingresa los nuevos créditos SATCA del Plan de Estudios.
	\UCpaso Continúa en el paso 5 de la trayectoria principal del \UCref{SP4-CU10}.
\end{UCtrayectoriaA}


%------------------------ CU TRAYECTORIA ALTERNARIVA B.4 -------------------------

\begin{UCtrayectoriaA}{B.4}{: El usuario desea modificar las Horas/Teoría del Plan de Estudios.}
	\UCpaso[\UCactor] Borra las Horas/Teoría del Plan de Estudios.
	\UCpaso[\UCactor] Ingresa las nuevas Horas/Teoría del Plan de Estudios.
	\UCpaso Continúa en el paso 5 de la trayectoria principal del \UCref{SP4-CU10}.
\end{UCtrayectoriaA}



%------------------------ CU TRAYECTORIA ALTERNARIVA B.5 -------------------------

\begin{UCtrayectoriaA}{B.5}{El usuario desea modificar las Horas/Prácticas del Plan de Estudios.}
	\UCpaso[\UCactor] Borra las Horas/Prácticas del Plan de Estudios.
	\UCpaso[\UCactor] Ingresa las nuevas Horas/Prácticas del Plan de Estudios.
	\UCpaso Continúa en el paso 5 de la trayectoria principal del \UCref{SP4-CU10}.
\end{UCtrayectoriaA}



%------------------------ CU TRAYECTORIA ALTERNARIVA B.6 -------------------------

\begin{UCtrayectoriaA}{B.6}{El usuario desea modificar la modalidad del Plan de Estudios.}
	\UCpaso[\UCactor] Selecciona la modalidad a la que pertenece el Plan de Estudios.
	\UCpaso Continúa en el paso 5 de la trayectoria principal del \UCref{SP4-CU10}.
\end{UCtrayectoriaA}



%------------------------ CU TRAYECTORIA ALTERNARIVA C -------------------------

\begin{UCtrayectoriaA}{C}{El actor quiere cancelar el registro de Unidad de Aprendizaje.}
	\UCpaso[\UCactor] Presiona el botón \IUbutton{Cancelar}.
	\UCpaso Muestra el mensaje \MSGref{MSG3}{¿Seguro que desea cancelar el registro?}.
	\UCpaso[\UCactor] Confirma la operación presionando el botón \IUbutton{Si}.
	\UCpaso Muestra la interfaz de usuario \IUref{IU2.1-D}{Consultar Mapa Curricular}
\end{UCtrayectoriaA}



%------------------------ CU TRAYECTORIA ALTERNARIVA D -------------------------

\begin{UCtrayectoriaA}{D}{El actor presiona accidentalmente el botón Cancelar}
	\UCpaso[\UCactor] Presiona el botón \IUbutton{Cancelar}
	\UCpaso Muestra el mensaje \MSGref{MSG3}{¿Seguro que desea cancelar el registro?}.
	\UCpaso[\UCactor] Presiona el botón \IUbutton{No}.
	\UCpaso Cierra el mensaje.
	\UCpaso Continúa en el paso 9 de la trayectoria principal del \UCref{SP4-CU3}.
\end{UCtrayectoriaA}



%------------------------ CU TRAYECTORIA ALTERNARIVA E -------------------------

\begin{UCtrayectoriaA}{E}{Uno o más campos  no fueron contestados.}
	\UCpaso Detecta uno o más campos sin contestar.
	\UCpaso Muestra el mensaje \MSGref{MSG1}{Todos los campos son obligatorios}.
	\UCpaso[\UCactor] Cierra el mensaje presionando el botón \IUbutton{Aceptar}.
	\UCpaso Continúa en el paso 2 de la trayectoria principal del \UCref{SP4-CU2}.
\end{UCtrayectoriaA}