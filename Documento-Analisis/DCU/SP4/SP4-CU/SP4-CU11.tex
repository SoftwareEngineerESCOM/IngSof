\begin{UseCase}{SP4-CU11}{Editar Recurso Humano}{El usuario Jefe de Innovación Educativa modifica la información general del Recurso Humano.}
        \UCitem{Versión}{\color{Gray}1.0}
        \UCitem{Autor}{\color{Gray}Plata García Josué Eliasaf}
        \UCitem{Supervisa}{\color{Gray}}
        \UCitem{Actor}{\hyperlink{Usuario}{Jefe de Innovación Educativa}}
        \UCitem{Propósito}{Cambiar los datos de un Recurso Humano.}
        \UCitem{Entradas}{Las entradas para Editar Recurso Humano:
            \begin{itemize}
                \item Nombre.
                \item Primer Apellido.
                \item Segundo Apellido.
                \item Cargo.
                \item Título.
                \item Lugar de Trabajo
            \end{itemize}
        }
        \UCitem{Origen}{Teclado.}
        \UCitem{Salidas}{
            \begin{itemize}
                \item \MSGref{MSG25}{Servicios no disponibles por el momento.}
                \item \MSGref{MSG29}{¿Está seguro que desea cancelar? Se perderán todos los avances sin guardar.}
                \item \MSGref{MSG31}{Los cambios se guardaron exitosamente.}
                \item \MSGref{MSG44}{Este campo es requerido.}
                \item \MSGref{MSG33}{El Recurso Humano con la Matrícula de Empleado [Número de Matrícula de Empleado] ya existe}
			    \item \MSGref{MSG35}{Escribe información válida.}
        	\end{itemize}
        }
        \UCitem{Destino}{Pantalla.}
        \UCitem{Precondiciones}{}
        \UCitem{Postcondiciones}{El Recurso Humano quedó actualizado en el sistema, permitiendo consultarlo y agregarlo al documento pdf.}
        \UCitem{Errores}{Ninguno}
        \UCitem{Estado}{Revisión.}
        \UCitem{Observaciones}{Ninguna}
\end{UseCase}
%--------------------------- CU TRAYECTORIA PRINCIPAL -------------------------
\begin{UCtrayectoria}{Principal}
    \UCpaso[\UCactor] Presiona el botón \IUbutton{Lapiz} de la interfaz de usuario \IUref{GRH-J}{Gestionar Recurso Humano}.
    \UCpaso Muestra la interfaz de usuario \IUref{ERH-J}{Editar Recurso Humano}.
    \UCpaso Carga el catálogo de Cargo descrito en la \BRref{BR14}{Catálogos existentes}. [Trayectoria A]
    \UCpaso Carga el catálogo de Título descrito en la \BRref{BR14}{Catálogos existentes}. [Trayectoria A]
    \UCpaso Carga el catálogo de Lugar de Trabajo descrito en la \BRref{BR14}{Catálogos existentes}. [Trayectoria A]
    \UCpaso Carga los datos del Recurso Humano a editar. [Trayectoria B]

    \UCpaso[\UCactor] Modifica los campos deseados.
    \UCpaso[\UCactor] Termina la operación presionando el botón \IUbutton{Guardar}.  Existen las trayectorias alternativas [Trayectoria C] y [Trayectoria C.1] al presionar el boton Cancelar.

    \UCpaso Verifica que todos los campos marcados como obligatorios hayan sido contestados(Exceptuando los dos apellidos los cuales son opcionales) conforme a la \BRref{BR13}{Todos los datos solicitados son obligatorios},\BRref{BR36}{Estructura de Información Laboral.} y \BRref{BR35}{Estructura de Recursos Humanos.}[Trayectoria D][Trayectoria E].

    \UCpaso Persiste los datos ingresados. [Trayectoria F]

    \UCpaso El sistema muestra el mensaje \MSGref{MSG31}{ Los cambios se guardaron exitosamente.}.
    \UCpaso[\UCactor] Cierra el mensaje presionando el botón \IUbutton{Aceptar}.
    \UCpaso Muestra la interfaz de usuario \IUref{GRH-J}{Gestionar Recurso Humano}
\end{UCtrayectoria}
    %------------------------ CU TRAYECTORIA ALTERNARIVA A -------------------------
\begin{UCtrayectoriaA}{A}{No cargan los catálogos de Recurso Humano.}
    \UCpaso Muestra el mensaje \MSGref{MSG25}{Servicios no disponibles por el momento.}.
    \UCpaso[\UCactor] Cierra el mensaje presionando el botón \IUbutton{Aceptar}.
    \UCpaso Muestra la interfaz de usuario \IUref{GRH-J}{Gestionar Recursos Humanos}.
\end{UCtrayectoriaA}
    %------------------------ CU TRAYECTORIA ALTERNARIVA B -------------------------
\begin{UCtrayectoriaA}{B}{No se pudieron obtener los datos del Recurso Humano.}
    \UCpaso Muestra el mensaje \MSGref{MSG25}{Servicios no disponibles por el momento.}.
    \UCpaso[\UCactor] Cierra el mensaje presionando el botón \IUbutton{Aceptar}.
    \UCpaso Muestra la interfaz de usuario \IUref{GRH-J}{Gestionar Recursos Humanos}.
\end{UCtrayectoriaA}
%------------------------ CU TRAYECTORIA ALTERNARIVA C -------------------------
\begin{UCtrayectoriaA}{C}{El actor presiona el botón \IUbutton{Cancelar}.}
    \UCpaso Muestra el mensaje \MSGref{MSG29}{¿Está seguro que desea cancelar? Se perderán todos los avances sin guardar.}.
    \UCpaso[\UCactor] Cierra el mensaje presionando el botón \IUbutton{Si}.
    \UCpaso Muestra la interfaz de usuario \IUref{GRH-J}{Gestionar Recurso Humano}
\end{UCtrayectoriaA}
%------------------------ CU TRAYECTORIA ALTERNARIVA C.1 -------------------------
\begin{UCtrayectoriaA}{C.1}{El actor presiona accidentalmente el botón \IUbutton{Cancelar}.}
    \UCpaso Muestra el mensaje \MSGref{MSG29}{¿Está seguro que desea cancelar? Se perderán todos los avances sin guardar.}.
    \UCpaso[\UCactor] Cierra el mensaje presionando el botón \IUbutton{No}.
    \UCpaso Continúa en el paso 4 de la trayectoria principal del \UCref{SP4-CU11}.
\end{UCtrayectoriaA}
%------------------------ CU TRAYECTORIA ALTERNARIVA D -------------------------
\begin{UCtrayectoriaA}{D}{El sistema detecta uno o más campos sin contestar.}
    \UCpaso Muestra el mensaje \MSGref{MSG44}{Este campo es requerido} debajo de los campos obligatorios sin contestar.
    \UCpaso Continúa en el paso 4 de la trayectoria principal del \UCref{SP4-CU11}.
\end{UCtrayectoriaA}
%------------------------ CU TRAYECTORIA ALTERNARIVA E -------------------------
\begin{UCtrayectoriaA}{E}{El Sistema detecta que los campos un cumplen con el diccionario de datos.}
    \UCpaso Muestra el mensaje \MSGref{MSG35}{Escribe información válida.} debajo del campo que incumplió con la \BRref{BR35}{Estructura de Recursos Humanos}.
    \UCpaso Continúa en el paso 4 de la trayectoria principal del \UCref{SP4-CU11}.
\end{UCtrayectoriaA}
%------------------------ CU TRAYECTORIA ALTERNARIVA F -------------------------
\begin{UCtrayectoriaA}{F}{Ocurre un error al momento de persistir los datos.}
    \UCpaso Muestra el mensaje \MSGref{MSG25}{Servicios no disponibles por el momento.}
    \UCpaso[\UCactor] Cierra el mensaje presionando el botón \IUbutton{Aceptar}.
    \UCpaso Muestra la interfaz de usuario \IUref{GRH-J}{Gestionar Recursos Humanos}.
\end{UCtrayectoriaA}
