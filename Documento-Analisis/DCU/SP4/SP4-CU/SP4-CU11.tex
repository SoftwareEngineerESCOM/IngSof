\begin{UseCase}{SP4-CU11}{Editar Recurso Humano}{El usuario Jefe de Innovación Educativa modifica la información general del Recurso Humano.}
        \UCitem{Versión}{\color{Gray}1.0}
        \UCitem{Autor}{\color{Gray}Plata García Josué Eliasaf}
        \UCitem{Supervisa}{\color{Gray}}
        \UCitem{Actor}{\hyperlink{Usuario}{Jefe de Innovación Educativa}}
        \UCitem{Propósito}{Cambiar los datos de un Recurso Humano.}
        \UCitem{Entradas}{Las entradas para Editar Recurso Humano:
            \begin{itemize}
                \item Nombre.
                \item Primer Apellido.
                \item Segundo Apellido.
                \item Cargo.
                \item Título.
                \item Lugar de Trabajo
            \end{itemize}
        }
        \UCitem{Origen}{Teclado.}
        \UCitem{Salidas}{
            \begin{itemize}
                \item \MSGref{MSG25}{Servicios no disponibles.}
                \item \MSGref{MSG29}{¿Está seguro que desea cancelar? Se perderán todos los avances sin guardar.}
                \item \MSGref{MSG31}{Los cambios se guardaron exitosamente.}
                \item \MSGref{MSG44}{Este campo es requerido.}
                \item \MSGref{MSG33}{El Recurso Humano con la Matrícula de Empleado [Número de Matrícula de Empleado] ya existe}
			    \item \MSGref{MSG35}{Escribe información válida.}
        	\end{itemize}
        }
        \UCitem{Destino}{Pantalla.}
        \UCitem{Precondiciones}{
            \begin{itemize}
                \item El catálogo de Cargo debe de estar cargado en el sistema.
                \item El catálogo de Titulo debe de estar cargado en el sistema.
                \item El catálogo de Lugar de Trabajo debe de estar cargado en el sistema.
            \end{itemize}
        }
        \UCitem{Postcondiciones}{El Recurso Humano quedó actualizado en el sistema, permitiendo consultarlo y agregarlo al documento pdf.}
        \UCitem{Errores}{Ninguno}
        \UCitem{Estado}{Revisión.}
        \UCitem{Observaciones}{Ninguna}
\end{UseCase}
%--------------------------- CU TRAYECTORIA PRINCIPAL -------------------------
\begin{UCtrayectoria}{Principal}
    \UCpaso[\UCactor] Presiona el botón \IUbutton{Lapiz} de la interfaz de usuario \IUref{GRH-J}{Gestionar Recurso Humano}.
    \UCpaso Muestra la interfaz de usuario \IUref{ERH-J}{Editar Recurso Humano}.
    \UCpaso Carga el catálogo de Cargo descrito en la \BRref{BR14}{Catálogos existentes}. [Trayectoria A]
    \UCpaso Carga el catálogo de Título descrito en la \BRref{BR14}{Catálogos existentes}. [Trayectoria A]
    \UCpaso Carga el catálogo de Lugar de Trabajo descrito en la \BRref{BR14}{Catálogos existentes}. [Trayectoria A]
    \UCpaso Carga los datos del Recurso Humano a editar. [Trayectoria B]
    \UCpaso[\UCactor] Modifica los campos deseados, el sistema verifica conforme al modelo de datos, la \BRref{BR38}{Verificación de formularios al momento} y la \BRref{BR39}{Todos los campos marcados con (*) son obligatorios}.[Trayectoria C][Trayectoria D]
    \UCpaso[\UCactor] Termina la operación presionando el botón \IUbutton{Finalizar}.  Existen las trayectorias alternativas [Trayectoria E] y [Trayectoria E.1] al presionar el boton Cancelar.

    \UCpaso Verifica que se cumpla la \BRref{BR39}{Todos los campos marcados con (*) son obligatorios}.[Trayectoria D]

    \UCpaso Persiste los datos ingresados. [Trayectoria F]

    \UCpaso El sistema muestra el mensaje \MSGref{MSG31}{ Los cambios se guardaron exitosamente.}.
    \UCpaso[\UCactor] Cierra el mensaje presionando el botón \IUbutton{Aceptar}.
    \UCpaso Muestra la interfaz de usuario \IUref{GRH-J}{Gestionar Recurso Humano}
\end{UCtrayectoria}
    %------------------------ CU TRAYECTORIA ALTERNARIVA A -------------------------
\begin{UCtrayectoriaA}{A}{No cargan los catálogos de Recurso Humano.}
    \UCpaso Muestra el mensaje \MSGref{MSG25}{Servicios no disponibles.}.
    \UCpaso[\UCactor] Cierra el mensaje presionando el botón \IUbutton{Aceptar}.
    \UCpaso Muestra la interfaz de usuario \IUref{GRH-J}{Gestionar Recursos Humanos}.
\end{UCtrayectoriaA}
    %------------------------ CU TRAYECTORIA ALTERNARIVA B -------------------------
\begin{UCtrayectoriaA}{B}{No se pudieron obtener los datos del Recurso Humano.}
    \UCpaso Muestra el mensaje \MSGref{MSG25}{Servicios no disponibles.}.
    \UCpaso[\UCactor] Cierra el mensaje presionando el botón \IUbutton{Aceptar}.
    \UCpaso Muestra la interfaz de usuario \IUref{GRH-J}{Gestionar Recursos Humanos}.
\end{UCtrayectoriaA}
%------------------------ CU TRAYECTORIA ALTERNARIVA E -------------------------
\begin{UCtrayectoriaA}{C}{El Sistema detecta campos que incumplen con el diccionario de datos.}
    \UCpaso Muestra el mensaje \MSGref{MSG35}{Escribe información válida.} debajo del campo que incumplió.
    \UCpaso Continúa en el paso 4 de la trayectoria principal del \UCref{SP4-CU11}.
\end{UCtrayectoriaA}
%------------------------ CU TRAYECTORIA ALTERNARIVA D -------------------------
\begin{UCtrayectoriaA}{D}{El sistema detecta uno o más campos sin contestar.}
    \UCpaso Muestra el mensaje \MSGref{MSG44}{Este campo es requerido} debajo de los campos obligatorios sin contestar.
    \UCpaso Continúa en el paso 4 de la trayectoria principal del \UCref{SP4-CU11}.
\end{UCtrayectoriaA}
%------------------------ CU TRAYECTORIA ALTERNARIVA E -------------------------
\begin{UCtrayectoriaA}{E}{El actor presiona el botón \IUbutton{Cancelar}.}
    \UCpaso Muestra el mensaje \MSGref{MSG29}{¿Está seguro que desea cancelar? Se perderán todos los avances sin guardar.}.
    \UCpaso[\UCactor] Cierra el mensaje presionando el botón \IUbutton{Si}.
    \UCpaso Muestra la interfaz de usuario \IUref{GRH-J}{Gestionar Recurso Humano}
\end{UCtrayectoriaA}
%------------------------ CU TRAYECTORIA ALTERNARIVA E.1 -------------------------
\begin{UCtrayectoriaA}{E.1}{El actor presiona accidentalmente el botón \IUbutton{Cancelar}.}
    \UCpaso Muestra el mensaje \MSGref{MSG29}{¿Está seguro que desea cancelar? Se perderán todos los avances sin guardar.}.
    \UCpaso[\UCactor] Cierra el mensaje presionando el botón \IUbutton{No}.
    \UCpaso Continúa en el paso 4 de la trayectoria principal del \UCref{SP4-CU11}.
\end{UCtrayectoriaA}
%------------------------ CU TRAYECTORIA ALTERNARIVA F -------------------------
\begin{UCtrayectoriaA}{F}{Ocurre un error al momento de persistir los datos.}
    \UCpaso Muestra el mensaje \MSGref{MSG25}{Servicios no disponibles.}
    \UCpaso[\UCactor] Cierra el mensaje presionando el botón \IUbutton{Aceptar}.
    \UCpaso Muestra la interfaz de usuario \IUref{GRH-J}{Gestionar Recursos Humanos}.
\end{UCtrayectoriaA}
