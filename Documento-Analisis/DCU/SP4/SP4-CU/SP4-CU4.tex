\begin{UseCase}{SP4-CU4}{Registrar Recurso Humano}{El usuario Jefe de Innovación Educativa ingresa los datos de los Recursos Humanos de su Unidad Académica.}
        \UCitem{Versión}{\color{Gray}1.1}
        \UCitem{Autor}{\color{Gray}Rivas Rojas Arturo}
        \UCitem{Supervisa}{\color{Gray}}
        \UCitem{Actor}{\hyperlink{JDIE}{Jefe de Innovación Educativa}}
        \UCitem{Propósito}{Ingresar al sistema la información necesaria de los Recursos Humanos.}
        \UCitem{Entradas}{Las entradas para el registro de un Recurso Humano serán:
          |\begin{itemize}
                \item Nombre.
                \item Primer Apellido.
                \item Segundo Apellido.
                \item Cargo.
                \item Título.
                \item Lugar de Trabajo
            \end{itemize}
        }
        \UCitem{Origen}{Teclado.}
        \UCitem{Salidas}{
            \begin{itemize}
                \item \MSGref{MSG5}{Registro finalizado exitosamente.}
                \item \MSGref{MSG25}{Servicios no disponibles.}
                \item \MSGref{MSG29}{¿Está seguro que desea cancelar? Se perderán todos los avances sin guardar.}
                \item \MSGref{MSG44}{Este campo es requerido.}
            \end{itemize}
        }
        \UCitem{Destino}{Pantalla.}
        \UCitem{Precondiciones}{
            \begin{itemize}
                \item El catálogo de Cargo debe de estar cargado en el sistema.
                \item El catálogo de Titulo debe de estar cargado en el sistema.
                \item El catálogo de Lugar de Trabajo debe de estar cargado en el sistema.
            \end{itemize}
        }
        \UCitem{Postcondiciones}{El Recurso Humano quedará registrado en el sistema, permitiendo consultarlo y relacionarlo a una Unidad de Aprendizaje.}
        \UCitem{Errores}{
            \begin{itemize}
                \item E1. No existen o no pudieron cargarse los catálogos necesarios.
            \end{itemize}
        }
        \UCitem{Estado}{Revisión.}
        \UCitem{Observaciones}{}
\end{UseCase}
%--------------------------- CU TRAYECTORIA PRINCIPAL -------------------------
\begin{UCtrayectoria}{Principal}
    \UCpaso[\UCactor] Presiona el botón \IUbutton{(+)} de la interfaz de usuario \IUref{GRH-J}{Gestionar Recursos Humanos}
    \UCpaso Carga el catálogo de Cargo descrito en la \BRref{BR14}{Catálogos del Sistema}. [Trayectoria A]
    \UCpaso Carga el catálogo de Título descrito en la \BRref{BR14}{Catálogos del Sistema}. [Trayectoria A]
    \UCpaso Carga el catálogo de Lugar de Trabajo descrito en la \BRref{BR14}{Catálogos del Sistema}. [Trayectoria A]
    \UCpaso Muestra la interfaz de usuario \IUref{RRH-J}{Registrar Recurso Humano}.
    \UCpaso[\UCactor] Ingresa el Nombre, el sistema verifica conforme al modelo de datos y la \BRref{BR38}{Verificación de formularios al momento} y la \BRref{BR39}{Todos los campos marcados con (*) son obligatorios}.[Trayectoria B][Trayectoria C]
    \UCpaso[\UCactor] Ingresa el Primer Apellido, el sistema verifica conforme al modelo de datos y la \BRref{BR38}{Verificación de formularios al momento}.[Trayectoria B]
    \UCpaso[\UCactor] Ingresa el Segundo Apellido, el sistema verifica conforme al modelo de datos y la \BRref{BR38}{Verificación de formularios al momento}.[Trayectoria B]
    \UCpaso[\UCactor] Selecciona el Título, el sistema verifica que se cumpla la \BRref{BR39}{Todos los campos marcados con (*) son obligatorios}.[Trayectoria C]
    \UCpaso[\UCactor] Selecciona uno o múltiples Cargos, el sistema verifica que se cumpla la \BRref{BR39}{Todos los campos marcados con (*) son obligatorios}.[Trayectoria C]
    \UCpaso[\UCactor] Selecciona el Lugar de Trabajo.
    \UCpaso[\UCactor] Termina la operación presionando el botón \IUbutton{Registrar}. [Trayectoria D] [Trayectoria D.1]
    \UCpaso Verifica que se cumpla la \BRref{BR39}{Todos los campos marcados con (*) son obligatorios}.[Trayectoria C]
    \UCpaso Persiste los datos ingresados. [Trayectoria E]
    \UCpaso Muestra el mensaje \MSGref{MSG5}{Registro finalizado exitosamente}.
    \UCpaso[\UCactor] Cierra el mensaje presionando el botón \IUbutton{Aceptar}.
    \UCpaso Muestra la interfaz de usuario \IUref{GRH-J}{Gestionar Recursos Humanos}.
\end{UCtrayectoria}
%------------------------ CU TRAYECTORIA ALTERNARIVA A -------------------------
\begin{UCtrayectoriaA}{A}{Los catálogos de la \BRref{BR14}{Catálogos del Sistema} necesarios no se pudieron cargar.}
    \UCpaso Muestra el mensaje \MSGref{MSG25}{Servicios no disponibles}.
    \UCpaso[\UCactor] Cierra el mensaje presionando el botón \IUbutton{Aceptar}.
\end{UCtrayectoriaA}
%------------------------ CU TRAYECTORIA ALTERNARIVA B -------------------------
\begin{UCtrayectoriaA}{B}{El Sistema detecta campos que incumplen con el diccionario de datos.}
    \UCpaso Muestra el mensaje \MSGref{MSG35}{ Escribe información válida} debajo del campo ique incumplió.
    \UCpaso Continúa en el paso 6 de la trayectoria principal del \UCref{SP4-CU11}.
\end{UCtrayectoriaA}
%------------------------ CU TRAYECTORIA ALTERNARIVA C -------------------------
\begin{UCtrayectoriaA}{C}{El Sistema detecta uno o más campos obligatorios sin contestar.}
    \UCpaso Muestra el mensaje \MSGref{MSG44}{Este campo es requerido}  debajo de cada campo obligatorio sin contestar.
    \UCpaso Continúa en el paso 6 de la trayectoria principal del \UCref{SP4-CU4}.
\end{UCtrayectoriaA}
%------------------------ CU TRAYECTORIA ALTERNARIVA D -------------------------
\begin{UCtrayectoriaA}{D}{El actor presiona el botón \IUbutton{Cancelar}}
    \UCpaso Muestra el mensaje \MSGref{MSG29}{¿Está seguro que desea cancelar? Se perderán todos los avances sin guardar}.
    \UCpaso[\UCactor] Cierra el mensaje presionando el botón \IUbutton{Si}.
    \UCpaso Muestra la interfaz de usuario \IUref{GRH-J}{Gestionar Recursos Humanos}.
\end{UCtrayectoriaA}
%------------------------ CU TRAYECTORIA ALTERNARIVA D.1 -------------------------
\begin{UCtrayectoriaA}{D.1}{El actor presiona accidentalmente el botón \IUbutton{Cancelar}}
    \UCpaso Muestra el mensaje \MSGref{MSG29}{¿Está seguro que desea cancelar el registro?}.
    \UCpaso[\UCactor] Cierra el mensaje presionando el botón \IUbutton{No}.
    \UCpaso Continúa en el paso 12 de la trayectoria principal del \UCref{SP4-CU4}.
\end{UCtrayectoriaA}
%------------------------ CU TRAYECTORIA ALTERNARIVA E -------------------------
\begin{UCtrayectoriaA}{E}{Ocurre un error al momento de persistir los datos.}
    \UCpaso Muestra el mensaje \MSGref{MSG25}{Servicios no disponibles.}
    \UCpaso[\UCactor] Cierra el mensaje presionando el botón \IUbutton{Aceptar}.
    \UCpaso Continúa en el paso 12 de la trayectoria principal del \UCref{SP4-CU4}.
\end{UCtrayectoriaA}