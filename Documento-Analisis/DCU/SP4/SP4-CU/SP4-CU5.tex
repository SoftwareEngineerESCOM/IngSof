\begin{UseCase}{SP4-CU5}{Consultar Programas Académicos}{El usuario visualiza la información de los Programas Académicos registrados.}
        \UCitem{Versión}{\color{Gray}1.0}
        \UCitem{Autor}{\color{Gray}Plata García Josué Eliasaf}
        \UCitem{Supervisa}{\color{Gray}Evelyn Reyes}
        \UCitem{Actor}{\hyperlink{Usuario}{Jefe de Innovación Educativa,Jefe de División de Innovación Académica y Jefe de Departamento de Desarrollo e Innovación Curricular}}
        \UCitem{Propósito}{El Usuario puede visualizar una lista  de  los Programas Académicos almacenados en el Sistema.}
        \UCitem{Entradas}{Las entradas para la consulta del Programa Académico:
          \begin{itemize}
            \item Ninguna
          \end{itemize}
        }
        \UCitem{Origen}{Ninguno}
        \UCitem{Salidas}{
        \begin{itemize}
                \item \MSGref{MSG25}{Servicios no disponibles.}
        \end{itemize}
        }
        \UCitem{Destino}{Pantalla.}
        \UCitem{Precondiciones}{Ninguna}
        \UCitem{Postcondiciones}{Ninguna}
        \UCitem{Errores}{Ningún}
        \UCitem{Puntos de Exteción}{
        \begin{itemize}
            \item \UCref{SP4-CU13}: Consultar Planes de Estudio.
            \item \UCref{SP4-CU1}: Registrar Programa Académico.
            \item \UCref{SP4-CU8}: Editar Programa Académico.
        \end{itemize}
    }
        \UCitem{Estado}{Revisión.}
        \UCitem{Observaciones}{}
\end{UseCase}
%--------------------------- CU TRAYECTORIA PRINCIPAL -------------------------
\begin{UCtrayectoria}{Principal}
    \UCpaso[\UCactor] Presiona el botón \IUbutton{Gestionar Programas Académicos} de la pantalla  \IUref{B-J}{Pantalla Bienvenida al Jefe de Innovación Educativa} o la pantalla \IUref{IU4}{Pantalla de bienvenida para el JDIA ó el JDIC.}. Existe la trayectoria alternativa [Trayectoria D] al presionar el boton  \IUbutton{Gestionar Programa Académico}.
    \UCpaso Muestra la interfaz de usuario \IUref{GPA-J}{Gestionar Programa Académico}.
    \UCpaso Carga el catálogo de Lugar de Trabajo descrito en la \BRref{BR14}{Catálogos existentes} [Trayectoria A].
    \UCpaso[\UCactor] Selecciona la Unidad Académica que desee como filtro. [Trayectoria B] lanzada al no seleccionar, [Trayectoria C] al verificar la \BRref{BR40}{El Jefe de Innovación Educativa solo hace operaciones dentro del Sistema que sean parte de su Unidad Académica.}
    \UCpaso[\UCactor] Aplica el filtro presionando el botón \IUbutton{Buscar}.
    \UCpaso Muestra tabla con los datos de los Programas Académicos que cumplan con el filtro.
\end{UCtrayectoria}

%------------------------ CU TRAYECTORIA ALTERNARIVA A -------------------------
\begin{UCtrayectoriaA}{A}{No cargan los catálogos de Lugar de Trabajo.}
    \UCpaso Muestra el mensaje \MSGref{MSG25}{Servicios no disponibles.}.
    \UCpaso[\UCactor] Cierra el mensaje presionando el botón \IUbutton{Aceptar}.
     \UCpaso Muestra la interfaz de usuario Menú para Jefe de Innovación Educativa \IUref{B-J}{Pantalla Bienvenida al Jefe de Innovación Educativa} o \IUref{IU4}{Pantalla de bienvenida para el JDIA ó el JDIC.} para Jefe de División de Innovación Académica y Jefe de Departamento de Desarrollo e Innovación Curricular.
\end{UCtrayectoriaA}
%------------------------ CU TRAYECTORIA ALTERNARIVA B -------------------------

\begin{UCtrayectoriaA}{B}{El actor no selecciona ningúna Unidad Académica}
    \UCpaso Muestra una tabla con los datos de todos los Programas Académicos.
    \UCpaso Continúa en el paso 6 de la trayectoria principal del \UCref{SP4-CU5}.
\end{UCtrayectoriaA}
%------------------------ CU TRAYECTORIA ALTERNARIVA C-------------------------

\begin{UCtrayectoriaA}{C}{El actor es el Jefe de Innovación Educativa.}
    \UCpaso Continúa en el paso 6 de la trayectoria principal del \UCref{SP4-CU5} con filtro predeterminado de su Unidad Académica.
\end{UCtrayectoriaA}
%------------------------ CU TRAYECTORIA ALTERNARIVA D -------------------------
\begin{UCtrayectoriaA}{D}{Problemas con la Conexión}
    \UCpaso Muestra el mensaje \MSGref{MSG25}{ Servicios no disponibles.}.
    \UCpaso[\UCactor] Cierra el mensaje presionando el botón \IUbutton{Aceptar}.
\end{UCtrayectoriaA}