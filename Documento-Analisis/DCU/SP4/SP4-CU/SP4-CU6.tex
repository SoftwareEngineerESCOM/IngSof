% Consultar Mapa Curricular.
\begin{UseCase}{SP4-CU6}{Consultar Plan de Estudio}{El usuario Docente visualiza el resumen del Plan de Estudios (Mapa Curricular).}
		\UCitem{Versión}{\color{Gray}3.0}
		\UCitem{Autor}{\color{Gray}Cervantes Moreno Christian Andres}
		\UCitem{Supervisa}{\color{Gray} Evelyn Reyes}
		\UCitem{Actor}{\hyperlink{Usuario}{Docente}}
		\UCitem{Propósito}{Visualizar el resumen del Plan de Estudios.}
		\UCitem{Entradas}{Ninguna}
		\UCitem{Origen}{Mouse}
		\UCitem{Salidas}{
        	\begin{itemize}
        		\item \MSGref{MSG38}{¿Está seguro que desea eliminar la Unidad de Aprendizaje [Nombre de la Unidad de Aprendizaje]?}.
        		\item \MSGref{MSG39}{¿Está seguro que desea eliminar el Semestre [Número del Semestre]?}.
        		\item \MSGref{MSG52}{El semestre no se puede eliminar debido a que contiene unidades de aprendizaje}.
        		\item \MSGref{MSG53}{¿Está seguro que desea rediseñar el plan de estudios? (Se hara una copia de este)}.
        	\end{itemize}
        }
		\UCitem{Destino}{Pantalla.}
		\UCitem{Precondiciones}{ Debe de existir al menos un Plan de Estudios registrado en el sistema.}
		\UCitem{Postcondiciones}{Se puede acceder a Registrar y Editar Unidades de Aprendizaje}
		\UCitem{Errores}{
			  \begin{itemize}
				\item Ningún
			\end{itemize}
		}
	 \UCitem{Puntos de Exteción}{
		\begin{itemize}
			\item \UCref{SP4-CU3}: Registrar Unidad de Aprendizaje.
			\item \UCref{SP4-CU9}: Editar Unidad de Aprendizaje.
			\item \UCref{SP-CU}: Asignar Unidad de Aprendizaje.
		\end{itemize}
	}
		\UCitem{Estado}{Revisión.}
		\UCitem{Observaciones}{}
\end{UseCase}
%--------------------------- CU TRAYECTORIA PRINCIPAL -------------------------
\begin{UCtrayectoria}{Principal}
    \UCpaso[\UCactor] Presiona sobre un Plan de Estudios en la Interfaz de usuario  \IUref{GPE-J}{Gestionar Planes de Estudio}.
    \UCpaso Muestra la interfaz de usuario \IUref{CMC-J}{Consultar Mapa Curricular}.
    \UCpaso[\UCactor] Selecciona el semestre de la Unidades de Aprendizaje que desea consultar.[Trayectoria A][Trayectoria B][Trayectoria B.1][Trayectoria F][Trayectoria F.1]
    \UCpaso Despliega las Unidades de Aprendizaje asociadas al Semestre.[Trayectoria C][Trayectoria C.1]
  
\end{UCtrayectoria}
%------------------------ CU TRAYECTORIA ALTERNARIVA A -------------------------
\begin{UCtrayectoriaA}{A}{El actor presiona el botón \IUbutton{+}.}
	\UCpaso Agrega el siguiente Semestre en la numeración.
    \UCpaso Continúa la trayectoria principal del \UCref{SP4-CU6} desde el paso 3.
\end{UCtrayectoriaA}
%------------------------ CU TRAYECTORIA ALTERNARIVA B -------------------------
\begin{UCtrayectoriaA}{B}{El usuario presiona el botón \IUbutton{X} de un Semestre}
	\UCpaso  Muestra el mensaje \MSGref{MSG39}{¿Está seguro que desea eliminar el Semestre [Número del Semestre]?}.
	\UCpaso[\UCactor] Cierra el mensaje presionando el botón \IUbutton{Si}.[Trayectoria E]
	\UCpaso Elimina el Semestre.
    \UCpaso Continúa en el paso 3 de la trayectoria principal del \UCref{SP4-CU6}.
\end{UCtrayectoriaA}
%------------------------ CU TRAYECTORIA ALTERNARIVA B.1 -------------------------
\begin{UCtrayectoriaA}{B.1}{El usuario presiona accidentalmente el botón \IUbutton{X} de un Semestre}
	\UCpaso  Muestra el mensaje \MSGref{MSG39}{¿Está seguro que desea eliminar el Semestre [Número del Semestre]?}.
	\UCpaso[\UCactor] Cierra el mensaje presionando el botón \IUbutton{No}.
    \UCpaso Continúa en el paso 3 de la trayectoria principal del \UCref{SP4-CU6}.
\end{UCtrayectoriaA}
%------------------------ CU TRAYECTORIA ALTERNARIVA C -------------------------
\begin{UCtrayectoriaA}{C}{El usuario presiona el botón \IUbutton{X} de una Unidad de Aprendizaje.}
	\UCpaso  Muestra el mensaje \MSGref{MSG38}{¿Está seguro que desea eliminar la Unidad de Aprendizaje [Nombre de la UNidad de Aprendizaje]?}.
	\UCpaso[\UCactor] Cierra el mensaje presionando el botón \IUbutton{Si}.
	\UCpaso Elimina la Unidad de Aprendizaje.
    \UCpaso Continúa en el paso 4 de la trayectoria principal del \UCref{SP4-CU6}.
\end{UCtrayectoriaA}
%------------------------ CU TRAYECTORIA ALTERNARIVA C.1 -------------------------
\begin{UCtrayectoriaA}{C.1}{El actor presiona accidentalmente el botón \IUbutton{X} de una Unidad de Aprendizaje.}
	\UCpaso Muestra el mensaje \MSGref{MSG38}{¿Está seguro que desea eliminar la Unidad de Aprendizaje [Nombre de la Unidad de Aprendizaje]?}.
	\UCpaso[\UCactor] Cierra el mensaje presionando el botón \IUbutton{No}.
    \UCpaso Continúa en el paso 4 de la trayectoria principal del \UCref{SP4-CU6}.
\end{UCtrayectoriaA}


%------------------------ CU TRAYECTORIA ALTERNARIVA  -------------------------
\begin{UCtrayectoriaA}{E}{El usuario desea eliminar un semestre con unidades de Aprendizaje}
	\UCpaso Verifica que el semestre contenga unidades de aprendizaje.
	\UCpaso Muestra el mensaje \MSGref{MSG52}{El semestre no se puede eliminar debido a que contiene unidades de aprendizaje.}
	\UCpaso[\UCactor] Cierra el mensaje presionando el botón \IUbutton{Aceptar}.
	\UCpaso Continúa en el paso 4 de la trayectoria principal del \UCref{SP4-CU6}.
\end{UCtrayectoriaA}

%------------------------ CU TRAYECTORIA ALTERNARIVA F -------------------------
\begin{UCtrayectoriaA}{F}{El usuario presiona accidentalmente el botón \BtnLapiz de Rediseñar}
	\UCpaso  Muestra el mensaje \MSGref{MSG53}{¿Está seguro que desea rediseñar el plan de estudios? (Se hara una copia de este)}.
	\UCpaso[\UCactor] Cierra el mensaje presionando el botón \IUbutton{Si}.[Trayectoria G]
	\UCpaso Muestra la interfaz de usuario \IUref{GPE-J}{Gestionar Planes de Estudio} de la copia de Plan de Estudio generada.
\end{UCtrayectoriaA}
%------------------------ CU TRAYECTORIA ALTERNARIVA F.1 -------------------------
\begin{UCtrayectoriaA}{F.1}{El usuario presiona el botón \BtnLapiz de Rediseñar}
	\UCpaso  Muestra el mensaje \MSGref{MSG53}{¿Está seguro que desea rediseñar el plan de estudios? (Se hara una copia de este)}.
	\UCpaso[\UCactor] Cierra el mensaje presionando el botón \IUbutton{No}.
	\UCpaso Continúa en el paso 3 de la trayectoria principal del \UCref{SP4-CU6}.
\end{UCtrayectoriaA}

%------------------------ CU TRAYECTORIA ALTERNARIVA G -------------------------
\begin{UCtrayectoriaA}{G}{El sistema detecta incumplimiento con la \BRref{BR41}{Solo puede haber un plan de estudios en estado de creación, rediseño y aprobado.} }
	\UCpaso Muestra el mensaje \MSGref{MSG57}{Ya se encuentra un Plan de Estudio en proceso}.
	\UCpaso[\UCactor] Cierra el mensaje presionando el botón \IUbutton{Aceptar}.
	\UCpaso Muestra la interfaz de usuario \IUref{CMC-J}{Consultar Mapa Curricular}.
\end{UCtrayectoriaA}

