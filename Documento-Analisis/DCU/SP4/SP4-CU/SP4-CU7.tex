\begin{UseCase}{SP4-CU7}{Consultar Recursos Humanos}{El usuario Jefe de Innovación Educativa visualiza los Recursos Humanos registrados de su Unidad Académica.}
        \UCitem{Versión}{\color{Gray}1.1}
        \UCitem{Autor}{\color{Gray}Rivas Rojas Arturo}
        \UCitem{Supervisa}{\color{Gray}}
        \UCitem{Actor}{\hyperlink{JDIE}{Jefe de Innovación Educativa}}
        \UCitem{Propósito}{Visualizar los Recursos Humanos registrados en el sistema.}
        \UCitem{Entradas}{Las entradas para la consulta de Recursos Humanos serán:
          \begin{itemize}
            \item Cargo.
          \end{itemize}
        }
        \UCitem{Origen}{Teclado.}
        \UCitem{Salidas}{
            \begin{itemize}
                \item \MSGref{MSG25}{Servicios no disponibles.}
                \item \MSGref{MSG34}{¿Está seguro que desea Eliminar el Recurso Humano?}
            \end{itemize}
        }
        \UCitem{Destino}{Pantalla.}
        \UCitem{Precondiciones}{
            \begin{itemize}
                \item El catálogo de Cargo debe de estar cargado en el sistema.
            \end{itemize}
        }
        \UCitem{Postcondiciones}{Los Recursos Humanos desplegados en pantalla podrán ser editados o eliminados.}
        \UCitem{Errores}{}
        \UCitem{Puntos de Extensión}{
            \begin{itemize}
                \item \UCref{SP4-CU4}: Registrar Recurso Humano.
                \item \UCref{SP4-CU10}: Editar Recurso Humano.
            \end{itemize}
        }
        \UCitem{Estado}{Revisión.}
        \UCitem{Observaciones}{}
\end{UseCase}
%--------------------------- CU TRAYECTORIA PRINCIPAL -------------------------
\begin{UCtrayectoria}{Principal}
    \UCpaso[\UCactor] Presiona la opción de Gestionar Recursos Humanos del menú en la interfaz de usuario \IUref{B-J}{Bienvenida al Jefe de Innovación Educativa}..
    \UCpaso Carga el catálogo de Cargo definido en la \BRref{BR14}{Catálogos existentes}. [Trayectoria A]
    \UCpaso Muestra la interfaz de usuario \IUref{GRH-J}{Gestionar Recursos Humanos}.
    \UCpaso[\UCactor] Selecciona el filtro.[Trayectoria B][Trayectoria B.1]
\end{UCtrayectoria}
%------------------------ CU TRAYECTORIA ALTERNARIVA A -------------------------
\begin{UCtrayectoriaA}{A}{Los catálogos de la \BRref{BR14}{Catálogos existentes} necesarios no se pudieron cargar.}
    \UCpaso Muestra el mensaje \MSGref{MSG25}{Servicios no disponibles}.
    \UCpaso[\UCactor] Cierra el mensaje presionando el botón \IUbutton{Aceptar}.
    \UCpaso Muestra la interfaz de usuario \IUref{B-J}{Bienvenida al Jefe de Innovación Educativa}.
\end{UCtrayectoriaA}
%------------------------ CU TRAYECTORIA ALTERNARIVA B -------------------------
\begin{UCtrayectoriaA}{B}{El actor no selecciona ningún filtro}
    \UCpaso Muestra una tabla con los datos de todos los Recursos Humanos.[Trayectoria C][Trayectoria C.1]
    \UCpaso Continúa en el paso 4 de la trayectoria principal del \UCref{SP4-CU7}.
\end{UCtrayectoriaA}
%------------------------ CU TRAYECTORIA ALTERNARIVA B.1 -------------------------
\begin{UCtrayectoriaA}{B.1}{El actor desea filtrar los Recursos Humanos}
    \UCpaso[\UCactor] Selecciona el Cargo que desea como filtro.
    \UCpaso Muestra una tabla con los datos de los Recursos Humanos con ese cargo.[Trayectoria C][Trayectoria C.1]
    \UCpaso Continúa en el paso 4 de la trayectoria principal del \UCref{SP4-CU7}.
\end{UCtrayectoriaA}
%------------------------ CU TRAYECTORIA ALTERNARIVA C -------------------------
\begin{UCtrayectoriaA}{C}{El actor presiona el botón \IUbutton{X} de un Recurso Humano}
    \UCpaso Muestra el mensaje \MSGref{MSG34}{¿Está seguro que desea Eliminar al Recurso Humano?}.
    \UCpaso[\UCactor] Cierra el mensaje presionando el botón \IUbutton{Si}.
    \UCpaso Continúa en el paso 4 de la trayectoria principal del \UCref{SP4-CU7}.
\end{UCtrayectoriaA}
%------------------------ CU TRAYECTORIA ALTERNARIVA C.1 -------------------------
\begin{UCtrayectoriaA}{C.1}{El actor presiona accidentalmente el botón \IUbutton{X} de un Recurso Humano}
    \UCpaso Muestra el mensaje \MSGref{MSG34}{¿Está seguro que desea Eliminar al Recurso Humano?}.
    \UCpaso[\UCactor] Cierra el mensaje presionando el botón \IUbutton{No}.
    \UCpaso Continúa en el paso 4 de la trayectoria principal del \UCref{SP4-CU7}.
\end{UCtrayectoriaA}