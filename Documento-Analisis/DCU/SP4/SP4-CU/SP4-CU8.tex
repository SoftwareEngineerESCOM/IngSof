\begin{UseCase}{SP4-CU8}{Editar Programa Académico}{El actor modifica los datos de los Programas Académicos.}
        \UCitem{Versión}{\color{Gray}1.0}
        \UCitem{Autor}{\color{Gray}Plata García Josué Eliasaf}
        \UCitem{Supervisa}{\color{Gray}Evelyn Reyes}
        \UCitem{Actor}{\hyperlink{Usuario}{Jefe de Innovación Educativa,Jefe de División de Innovación Académica y Jefe de Departamento de Desarrollo e Innovación Curricular}}
        \UCitem{Propósito}{Cambiar el nombre del Programa Académico en el sistema.}
        \UCitem{Entradas}{Las entradas para Editar Programa Académico:
          \begin{itemize}
            \item Nombre
          \end{itemize}
        }
        \UCitem{Origen}{Teclado.}
        \UCitem{Salidas}{
            \begin{itemize}
                \item \MSGref{MSG25}{Servicios no disponibles.}
                \item \MSGref{MSG29}{¿Está seguro que desea cancelar? Se perderán todos los avances sin guardar.}
                \item \MSGref{MSG31}{Los cambios se guardaron exitosamente.}
                \item \MSGref{MSG44}{Este campo es requerido}
                \item \MSGref{MSG35}{Escribe información válida.}
                \item \MSGref{MSG43}{El nombre del Programa Académico ya está registrado. Por favor cámbielo.}
            \end{itemize}
        }
        \UCitem{Destino}{Pantalla.}
        \UCitem{Precondiciones}{Ninguna.}
        \UCitem{Postcondiciones}{El Programa Académico quedó actualizado en el sistema, permitiendo consultarlo y generar tareas de registro del mismo o quedó sin modificarse.}
        \UCitem{Errores}{
          Ninguno
        }
        \UCitem{Estado}{Revisión.}
        \UCitem{Observaciones}{}
\end{UseCase}

%--------------------------- CU TRAYECTORIA PRINCIPAL -------------------------
\begin{UCtrayectoria}{Principal}
    \UCpaso[\UCactor] Presiona el botón \BtnLapiz de la interfaz de usuario \IUref{GPA-J}{Gestionar Programas Académicos}.
    \UCpaso Cargar el nombre del Programa Académico.[Trayectoria F]
    \UCpaso Muestra la interfaz de usuario \IUref{EPA-J}{Editar Programa Académico}.
    \UCpaso[\UCactor] Cambia el nombre del Programa Académico, el sistema verifica conforme al modelo de datos la \BRref{BR38}{Verificación de formularios al momento}[Trayectoria C].
    \UCpaso[\UCactor] Termina la operación presionando el botón \IUbutton{Finalizar}. Existen las trayectorias alternativas [Trayectoria A] y [Trayectoria A.1] al presionar el botón Cancelar.
    \UCpaso Verifica que se cumplan la \BRref{BR39}{Todos los campos marcados con (*) son obligatorios.} y la \BRref{BR19}{El nombre del Programa Académico es único.} [Trayectoria B][Trayectoria D].
    \UCpaso Persiste los datos ingresados.Accionando el botón \IUbutton{Finalizar} se activa la  [Trayectoria E].
    \UCpaso El sistema muestra el mensaje \MSGref{MSG31}{Los cambios se guardaron exitosamente.}
    \UCpaso[\UCactor] Cierra el mensaje presionando el botón \IUbutton{Aceptar}.
    \UCpaso Muestra la interfaz de usuario \IUref{GPA-J}{Gestionar Programas Académicos}.
\end{UCtrayectoria}

%------------------------ CU TRAYECTORIA ALTERNARIVA A -------------------------

\begin{UCtrayectoriaA}{A}{El actor presiona el botón \IUbutton{Cancelar}.}
    \UCpaso Muestra el mensaje \MSGref{MSG29}{¿Está seguro que desea cancelar? Se perderán todos los avances sin guardar.}.
    \UCpaso[\UCactor] Confirma la operación presionando el botón \IUbutton{Si}.
    \UCpaso Muestra la interfaz de usuario \IUref{GPA-J}{Gestionar Programas Académicos}
\end{UCtrayectoriaA}

%------------------------ CU TRAYECTORIA ALTERNARIVA A.1 -------------------------

\begin{UCtrayectoriaA}{A.1}{El actor presiona accidentalmente el botón \IUbutton{Cancelar}.}
    \UCpaso Muestra el mensaje \MSGref{MSG29}{¿Está seguro que desea cancelar? Se perderán todos los avances sin guardar.}.
    \UCpaso[\UCactor] Presiona el botón \IUbutton{No}.
    \UCpaso Cierra el mensaje.
    \UCpaso Continúa en el paso 4 de la trayectoria principal del \UCref{SP4-CU8}.
\end{UCtrayectoriaA}

%------------------------ CU TRAYECTORIA ALTERNARIVA B -------------------------

\begin{UCtrayectoriaA}{B}{El sistema detecta uno o más campos sin contestar.}
    \UCpaso Muestra el mensaje \MSGref{MSG44}{Este campo es requerido} debajo de cada campo obligatorio sin contestar.
    \UCpaso Continúa en el paso 4 de la trayectoria principal del \UCref{SP4-CU8}.

\end{UCtrayectoriaA}
%------------------------ CU TRAYECTORIA ALTERNARIVA C -------------------------

\begin{UCtrayectoriaA}{C}{El Sistema detecta que el Nombre de Programa Académico no cumple con el diccionario de datos.}
    \UCpaso Muestra el mensaje \MSGref{MSG35}{Escribe información válida} debajo del campo.
    \UCpaso Continúa en el paso 4 de la trayectoria principal del \UCref{SP4-CU8}.
\end{UCtrayectoriaA}
    %------------------------ CU TRAYECTORIA ALTERNARIVA D -------------------------

\begin{UCtrayectoriaA}{D}{El nombre del Programa Académico ya está registrado.}
    \UCpaso Muestra el mensaje \MSGref{MSG43}{El nombre del Programa Académico ya está registrado. Por favor cámbielo.}
    \UCpaso[\UCactor] Cierra el mensaje presionando el botón \IUbutton{Aceptar}.
    \UCpaso Continúa en el paso 4 de la trayectoria principal del \UCref{SP4-CU8}.
\end{UCtrayectoriaA}

%------------------------ CU TRAYECTORIA ALTERNARIVA F -------------------------

\begin{UCtrayectoriaA}{E}{Ocurre un error al momento de persistir los datos.}
    \UCpaso Muestra el mensaje \MSGref{MSG25}{Servicios no disponibles.}
    \UCpaso[\UCactor] Cierra el mensaje presionando el botón \IUbutton{Aceptar}.
\end{UCtrayectoriaA}

%------------------------ CU TRAYECTORIA ALTERNARIVA G -------------------------
\begin{UCtrayectoriaA}{F}{No se pudieron obtener los datos del Programa Académico.}
    \UCpaso Muestra el mensaje \MSGref{MSG25}{Servicios no disponibles.}.
    \UCpaso[\UCactor] Cierra el mensaje presionando el botón \IUbutton{Aceptar}.
    \UCpaso Muestra la interfaz de usuario \IUref{GPA-J}{Gestionar Programas Académicos}.
\end{UCtrayectoriaA}