% REGISTRAR PLAN DE ESTUDIOS.
\begin{UseCase}{SP4-CU9}{Editar Unidad de Aprendizaje}{El usuario Jefe de Innovación modifica los datos de las Unidades de Aprendizaje registradas en el sistema.}
	\UCitem{Versión}{\color{Gray}3.0}
	\UCitem{Autor}{\color{Gray}Cervantes Moreno Christian Andres}
	\UCitem{Supervisa}{\color{Gray} Evelyn Reyes}
	\UCitem{Actor}{\hyperlink{Usuario}{Jefe de Innovación}}
	\UCitem{Propósito}{Editar el nombre, créditos TEPIC, créditos SATCA, horas Teoría/Semana, horas Práctica/semana, Área de formación y semestre de una Unidad de Aprendizaje.}
	\UCitem{Entradas}{Las entradas para la modificación de la Unidad de Aprendizaje serán:
		\begin{itemize}
			\item Nombre.
			\item Créditos Totales TEPIC.
			\item Créditos Totales SATCA.
			\item Horas Teoría/Semana
			\item Horas Práctica/Semana
			\item Área de Formación.
			\item Semestre
		\end{itemize}
	}
	\UCitem{Origen}{Teclado, Mouse}
	\UCitem{Salidas}{
		\begin{itemize}
			\item \MSGref{MSG31}{Los cambios se guardaron exitosamente.}
			\item \MSGref{MSG47}{El nombre de la Unidad de Aprendizaje ya está registrado. Por favor cámbielo.}
			\item \MSGref{MSG29}{¿Está seguro que desea cancelar? Se perderán todos los avances sin guardar.}
			\item \MSGref{MSG35}{Escribe información válida.}
			\item \MSGref{MSG44}{Este campo es requerido.}
			\item \MSGref{MSG25}{Servicios no disponibles. }
		\end{itemize}
	}
	\UCitem{Destino}{Pantalla.}
	\UCitem{Precondiciones}{ Debe de existir al menos una Unidad de Aprendizaje registrada en el sistema.}
	\UCitem{Postcondiciones}{La Unidad de Aprendizaje queda modificada en el sistema.}
	\UCitem{Errores}{
		\begin{itemize}
			\item E1. Los catálogos: Área de formación y semestre no se cargaron correctamente.
			\item E2. Hubo un problema al conectarse con la base de datos.
		\end{itemize}
	}
	\UCitem{Estado}{Revisión.}
	\UCitem{Observaciones}{}
\end{UseCase}


%--------------------------- CU TRAYECTORIA PRINCIPAL -------------------------
\begin{UCtrayectoria}{Principal}

	 %Usuario
	\UCpaso[\UCactor] Elige el semestre que pertecene la Unidad de Aprendizaje a modificar, que se encuentra en la interfaz de usuario \IUref{CMC-J}{Consultar Mapa Curricular}.
	\UCpaso[\UCactor] Presiona el boton \BtnLapiz de la Unidad de Aprendizaje que desea modificar.
	%Sistema
	\UCpaso Carga los catálogos: Área de formación y Semestre descrito en la \BRref{BR14}{Catálogos del Sistema}.[Trayectoria A]
	\UCpaso Muestra la interfaz de usuario \IUref{RUA-J}{Registrar Unidad de Aprendizaje}.
	\UCpaso[\UCactor] Modifica los campos deseados.
	\UCpaso[\UCactor] Termina la operación presionando el botón \IUbutton{Finalizar}. [Trayectoria B] [Trayectoria B.1]
	\UCpaso Verifica que se cumpla con el modelos de datos. [Trayectoria C] [Trayectoria D] [Trayectoria F]
	\UCpaso Persiste los datos ingresados. [Trayectoria E]
	\UCpaso El sistema muestra el mensaje \MSGref{MSG31}{Los cambios se guardaron exitosamente.}
	\UCpaso[\UCactor] Cierra el mensaje presionando el botón \IUbutton{Aceptar}.
	\UCpaso Muestra la interfaz de usuario  \IUref{CMC-J}{Consultar Mapa Curricular}.
\end{UCtrayectoria}

%------------------------ CU TRAYECTORIA ALTERNARIVA A -------------------------
\begin{UCtrayectoriaA}{A}{ Los catálogos de la \BRref{BR14}{Catálogos del Sistema} necesarios no se pudieron cargar.}
	\UCpaso Muestra el mensaje \MSGref{MSG2}{Servicios no disponibles. }
	\UCpaso[\UCactor] Cierra el mensaje presionando el botón \IUbutton{Aceptar}.
	\UCpaso Muestra la interfaz de usuario \IUref{CMC-J}{Consultar Mapa Curricular}.
\end{UCtrayectoriaA}

%------------------------ CU TRAYECTORIA ALTERNARIVA B -------------------------
\begin{UCtrayectoriaA}{B}{El actor presiona el botón \IUbutton{Cancelar}.}
	\UCpaso Muestra el mensaje \MSGref{MSG29}{¿Está seguro que desea cancelar? Se perderán todos los avances sin guardar}.
	\UCpaso[\UCactor] Confirma la operación presionando el botón \IUbutton{Si}.
	\UCpaso Muestra la interfaz de usuario \IUref{CMC-J}{Consultar Mapa Curricular}.
\end{UCtrayectoriaA}

%------------------------ CU TRAYECTORIA ALTERNARIVA B.1 -------------------------
\begin{UCtrayectoriaA}{B.1}{El actor presiona accidentalmente el botón \IUbutton{Cancelar}.}
	\UCpaso[\UCactor] Presiona el botón \IUbutton{Cancelar}
	\UCpaso Muestra el mensaje \MSGref{MSG29}{¿Está seguro que desea cancelar? Se perderán todos los avances sin guardar}.
	\UCpaso[\UCactor] Presiona el botón \IUbutton{No}.
	\UCpaso Cierra el mensaje.
	\UCpaso Continúa en el paso 5 de la trayectoria principal del \UCref{SP4-CU9}.
\end{UCtrayectoriaA}

%------------------------ CU TRAYECTORIA ALTERNARIVA C -------------------------
\begin{UCtrayectoriaA}{C}{El sistema detecta uno o más campos sin contestar incumpliendo con el modelo de datos.}
	\UCpaso Muestra el mensaje \MSGref{MSG44}{Este campo es requerido}debajo de cada campo obligatorio sin contestar.
	\UCpaso Continúa en el paso 5 de la trayectoria principal del \UCref{SP4-CU9}.
\end{UCtrayectoriaA}

%------------------------ CU TRAYECTORIA ALTERNARIVA D -------------------------
\begin{UCtrayectoriaA}{D}{El sistema detecta incumplimiento con el diccionario de datos.}
	\UCpaso Muestra el mensaje \MSGref{MSG35}{Escribe información válida} debajo del campo que incumplio con el modelo de datos.
	\UCpaso Continúa en el paso 5 de la trayectoria principal del \UCref{SP4-CU9}.
\end{UCtrayectoriaA}

%------------------------ CU TRAYECTORIA ALTERNARIVA E -------------------------
\begin{UCtrayectoriaA}{E}{Ocurre un error al momento de persistir los datos.}
	\UCpaso Muestra el mensaje \MSGref{MSG25}{Servicios no disponibles}.
	\UCpaso[\UCactor] Cierra el mensaje presionando el botón \IUbutton{Aceptar}.
	\UCpaso Muestra la interfaz de usuario \IUref{CMC-J}{Consultar Mapa Curricular}.
\end{UCtrayectoriaA}


%------------------------ CU TRAYECTORIA ALTERNARIVA F -------------------------
\begin{UCtrayectoriaA}{F}{El nombre de la Unidad de Aprendizaje ya se encuentra registrado}
	\UCpaso Muestra el mensaje \MSGref{MSG47}{El nombre de la Unidad de Aprendizaje ya está registrado. Por favor cámbielo}.
	\UCpaso[\UCactor] Cierra el mensaje presionando el botón \IUbutton{Aceptar}.
	\UCpaso Continúa en el paso 5 de la trayectoria principal del \UCref{SP4-CU9}.
\end{UCtrayectoriaA}