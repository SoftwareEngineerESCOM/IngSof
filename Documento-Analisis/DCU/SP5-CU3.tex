
\begin{UseCase}{SP5-CU3}{Consultar usuarios.}{El JDIC, JDIA y JIE podrá visualizar la información de los empleados.}
	\UCitem{Versión}{\color{Gray}3.0}
	\UCitem{Autor}{\color{Gray}Hernández Ruiz Rafael}
	\UCitem{Supervisa}{\color{Gray}Abigail Nicolás Sayago}
	\UCitem{Actor}{JDIC, JDIA, JIE.}
	\UCitem{Propósito}{Ver los empleados que cada actor posee, así como realizar trabajos de gestión: editar y eliminar usuarios.}
	\UCitem{Entradas}{
		\begin{itemize}
			\item Selección del cargo de los empleados a buscar.
			\item Clic en botón x.
			\item Clic en botón Cancelar.
			\item Clic en botón Aceptar.
	\end{itemize}}
	\UCitem{Origen}{Mouse.}
	\UCitem{Salidas}{
		\begin{itemize}
			\item Lista de empleados de un cargo en específico con sus datos (matricula,  nombre, primer apellido, segundo apellido, titulo, cargo, lugar de trabajo). 
			
			\item \MSGref{MSG21}{No hay usuarios registrados con ese cargo.}
			\item \MSGref{MSG22}{¿Está seguro que desea eliminar al usuario?}
			\item \MSGref{MSG25}{Servicios no disponibles.}
		\end{itemize}
	}
	\UCitem{Destino}{Pantalla.}
	\UCitem{Precondiciones}{ 1.- Debe existir por lo menos un registro en el catálogo de cargos de la  \BRref{BR14}{Catálogos del Sistema}.}
	\UCitem{Postcondiciones}{1.- Habilita la llamada a los casos de uso  SP5-CU4, SP5-CU5.}
	\UCitem{Errores}{ \begin{itemize}
			\item El catálogo de cargos no se cargo correctamente.
			\item Hubo un problema al conectarse con el servidor.
			\item Hubo un problema al conectarse con la base de datos. \end{itemize}}
	\UCitem{Estado}{Gestión.}
	\UCitem{Observaciones}{}
	\UCitem{Puntos de extensión}{Casos de uso \UCref{SP5-CU5}, \UCref{SP5-CU4}.}
\end{UseCase}

\begin{UCtrayectoria}{Principal}
	
	\UCpaso[\UCactor] Presiona en la interfaz de usuario \IUref{IU1}{Menú} la opción de gestionar usuarios. 
	\UCpaso  El sistema verifica la existencia de registros del catalogo de la \BRref{BR14}{Catálogos del Sistema} para cargos  y carga los indicados para el actor según la \BRref{BR3}{Gestión de Usuarios} . [Trayectoria B] [Trayectoria F] 
	\UCpaso El sistema carga la interfaz de usuario  \IUref{IU2}{Gestionar usuarios}.
	\UCpaso El sistema despliega la información (matricula,  nombre, primer apellido, segundo apellido, titulo, cargo, lugar de trabajo) de todos los usuarios disponibles para el actor según la  \BRref{BR3}{Gestión de Usuarios} en la parte inferior de la pantalla \IUref{IU2}{Gestionar usuarios}.
	\UCpaso [\UCactor] Visualiza la información de los usuarios.  [Trayectoria C] [Trayectoria E] [Trayectoria G][Trayectoria D]
\end{UCtrayectoria}

\begin{UCtrayectoriaA}{B}{No existen registros en el catálogo de cargos.}
	\UCpaso     El sistema muestra el 	\MSGref{MSG25}{Servicios no disponibles.}
	\UCpaso[\UCactor] Cierra el mensaje presionando el botón \IUbutton{Aceptar}.
	\UCpaso  El sistema continua en la interfaz de usuario \IUref{IU1}{Menú}.
\end{UCtrayectoriaA}

\begin{UCtrayectoriaA}{C}{No existen  empleados con el cargo seleccionado.}
	\UCpaso     El sistema muestra el \MSGref{MSG21}{No hay usuarios registrados con ese cargo}.
	\UCpaso[\UCactor] Cierra el mensaje presionando el botón \IUbutton{Aceptar}.
	\UCpaso Continua en el paso 5 de la trayectoria principal del \UCref{SP5-CU3}.
\end{UCtrayectoriaA}


\begin{UCtrayectoriaA}{D}{El actor presiona el botón \IUbutton{X}.}
	\UCpaso El sistema muestra el mensaje \MSGref{MSG22}{¿Está seguro que desea eliminar al usuario?} solicitando confirmación. [Trayectoria D1]
	\UCpaso[\UCactor] El actor presiona el botón \IUbutton{No}.
	\UCpaso Continua en el paso 5 de la trayectoria principal del \UCref{SP5-CU3}.  	
\end{UCtrayectoriaA}

\begin{UCtrayectoriaA}{D1}{El actor presiona el botón \IUbutton{Sí}.}
	\UCpaso     El sistema elimina al empleado. [Trayectoria F] 
	\UCpaso Continua en el paso 5 de la trayectoria principal del \UCref{SP5-CU3}.  
\end{UCtrayectoriaA}


%------------------------ CU TRAYECTORIA ALTERNARIVA F -------------------------

\begin{UCtrayectoriaA}{E}{El actor selecciono sin filtro.}
	\UCpaso[\UCactor] Selecciona la opción sin filtro.
	\UCpaso El sistema despliega la información (matricula,  nombre, primer apellido, segundo apellido, titulo, cargo, lugar de trabajo) de todos los usuarios disponibles para el actor según la  \BRref{BR3}{Gestión de Usuarios} en la parte inferior de la pantalla \IUref{IU2}{Gestionar usuarios}.
	\UCpaso Continua en el paso 5 de la trayectoria principal del \UCref{SP5-CU3}. 
\end{UCtrayectoriaA}

\begin{UCtrayectoriaA}{F}{La base de datos no está disponible, o hubo un error de conexión.}
	\UCpaso El sistema muestra el mensaje \MSGref{MSG25}{Servicios no disponibles.}
	\UCpaso[\UCactor] Cierra el mensaje presionando el botón \IUbutton{Aceptar}.
	\UCpaso  El sistema continua en la interfaz de usuario \IUref{IU2}{Gestionar usuarios}.
\end{UCtrayectoriaA}

\begin{UCtrayectoriaA}{G}{El actor consulta a los usuarios por un cargo.}
	\UCpaso[\UCactor] Selecciona el cargo de los empleados a buscar.
	\UCpaso El sistema despliega la información (matricula,  nombre, primer apellido, segundo apellido, titulo, cargo, lugar de trabajo) de los usuarios con ese cargo.
	\UCpaso Continua en el paso 5 de la trayectoria principal del \UCref{SP5-CU3}. 
\end{UCtrayectoriaA}





