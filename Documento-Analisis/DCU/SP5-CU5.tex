\begin{UseCase}{SP5-CU5}{Editar usuario.}{Los jefes (\hyperlink{JDIC}{JDIC}, \hyperlink{JDIA}{JDIA}, \hyperlink{JUA}{JUA}) podran realizar modificaciones a la información de los empleados   a su cargo}
        \UCitem{Versión}{\color{Gray}2.0}
        \UCitem{Autor}{\color{Gray}Hernández Ruiz Rafael}
        \UCitem{Supervisa}{\color{Gray}Abigail Nicolás Sayago}
        \UCitem{Actor}{\hyperlink{JDIC}{JDIC}, \hyperlink{JDIA}{JDIA}, \hyperlink{JUA}{JUA}.}
        \UCitem{Propósito}{Poder realizar modificaciones en la información de los empleados por motivos de actualización o errores }
        \UCitem{Entradas}{
          \begin{itemize}
            \item Matricula del empleado.
            \item Clic en botón buscar.
        \end{itemize}}
        \UCitem{Origen}{Mouse.}
        \UCitem{Salidas}{
            \begin{itemize}
                \item Información del empleado (nombre, apellido paterno, apellido materno, titulo, cargo, lugar de trabajo y contraseña).
                \item \MSGref{MSG7}{Los catálogos necesarios no se han cargado, favor de intentarlo más tarde}
                \item \MSGref{MSG27}{Empleado modificado.}
           
            \end{itemize}
        }
        \UCitem{Destino}{Pantalla.}
        \UCitem{Precondiciones}{ \begin{itemize}
            \item Debe existir por lo menos un registro en el catálogo de cargos.
            \item Debe existir por lo menos un registro en el catalogo de lugares de trabajo.
        \end{itemize} }
        \UCitem{Postcondiciones}{}
        \UCitem{Errores}{ \begin{itemize}
        \item El catálogo de cargos no se cargo correctamente.
        \item Hubo un problema al conectarse con el servidor.
        \item Hubo un problema al conectarse con la base de datos. \end{itemize}}
        \UCitem{Estado}{Gestión.}
        \UCitem{Observaciones}{}

\end{UseCase}

\begin{UCtrayectoria}{Principal}
    
    \UCpaso[\UCactor] Presiona en la interfaz de usuario \IUref{IU2}{Gestionar empleados} la opción de editar un usuario. 
    \UCpaso  El sistema verifica la existencia de registros en los catálogos cargos  y  lugares de trabajo . [Trayectoria B] 
    \UCpaso El sistema carga la información del catalogo de empleado  según la \BRref{BR3}{Gestión de Usuarios} para el actor.
    \UCpaso El sistema carga la pantalla  \IUref{IU7}{Editar empleados}.
    \UCpaso[\UCactor] Edita la información de los empleados.[Trayectoria C] [Trayectoria D] 
    \UCpaso[\UCactor]  Presiona el botón de \IUbutton{Modificar}.
    \UCpaso El sistema realiza la modificación de la información del empleado.
    \UCpaso  El sistema muestra el \MSGref{MSG27}{Empleado modificado}.    
    \UCpaso El sistema regresa a la interfaz de usuario \IUref{IU2}{Gestionar empleados}
\end{UCtrayectoria}

\begin{UCtrayectoriaA}{B}{No existen registros en los catálogos.}
    \UCpaso     El sistema muestra el \MSGref{MSG7}{Los catálogos necesarios no se han cargado, favor de intentarlo más tarde}.
    \UCpaso[\UCactor] Cierra el mensaje presionando el botón \IUbutton{Aceptar}.
    \UCpaso El sistema regresa a la interfaz de usuario \IUref{IU2}{Gestionar empleados}.
\end{UCtrayectoriaA}

\begin{UCtrayectoriaA}{C}{Cambio en el cargo de un usuario.}
    \UCpaso     El sistema ajustara las opciones de zona de trabajo de acuerdo con el cargo y el actor según la \BRref{BR4}{Cada Usuario debe estar relacionado a una zona de trabajo.}
    \UCpaso     Continua en el paso 5 de la trayectoria principal del \UCref{SP5-CU5}.
\end{UCtrayectoriaA}

\begin{UCtrayectoriaA}{D}{El actor presiono el botón \IUbutton{Cancelar}.}
 \UCpaso El sistema regresa a la interfaz de usuario \IUref{IU2}{Gestionar empleados}
\end{UCtrayectoriaA}
