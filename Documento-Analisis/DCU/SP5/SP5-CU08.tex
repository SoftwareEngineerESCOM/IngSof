
\chapter{Especificación de Casos de Uso}

\begin{UseCase}{M-CU8}{Consultar Usuarios Jefe de División de Innovación Académica. }{El  Jefe de División de Innovación Académica podrá visualizar la información de los empleados.}
		\UCitem{Versión}{\color{Gray}1.0}
		\UCitem{Autor}{\color{Gray}Hernández Ruiz Rafael}
		\UCitem{Supervisa}{\color{Gray}Abigail Nicolás Sayago}
		\UCitem{Actor}{Jefe de División de Innovación Académica.}
		\UCitem{Propósito}{Que el Jefe de División de Innovación Académica vea sus empleados y pueda tomar otras acciones(Eliminar y editar usuarios).}
		\UCitem{Entradas}{
          \begin{itemize}
          	\item Selección del cargo de los empleados a buscar.
          	\item Clic en botón buscar.
		\end{itemize}}
		\UCitem{Origen}{Mouse.}
		\UCitem{Salidas}{
        	\begin{itemize}
        		\item Lista de empleados de un cargo en específico con sus datos(Cargo, nombre, matricula, titulo unidad académica). 
                \item \label{itm:MSGR1} MSGR1. No se han cargado los catálogos.
                \item \label{MSGR2} MSGR2. No hay usuarios registrados con ese cargo. 
        	\end{itemize}
        }
		\UCitem{Destino}{Pantalla.}
		\UCitem{Precondiciones}{ Debe existir por lo menos un registro en el catálogo de la BRR1}
		\UCitem{Postcondiciones}{Habilita la llamada a los casos de uso  M-CU10, M-CU11, M-CU13.}
		\UCitem{Errores}{ \begin{itemize}
		\item El catálogo de cargos no se cargo correctamente.
		\item Hubo un problema al conectarse con el servidor
		\item Hubo un problema al conectarse con la base de datos \end{itemize}}
		\UCitem{Estado}{Revisión.}
		\UCitem{Observaciones}{}
		\UCitem{Puntos de extensión}{casos de uso  \UCref{M-CU10} , \UCref{M-CU11}, \UCref{M-CU13} }
\end{UseCase}

\begin{UCtrayectoria}{Principal}
    
    \UCpaso[\UCactor] Presiona en el menú de navegación la opción de gestionar usuarios. 
    \UCpaso  El sistema verifica la existencia de registros del catalogo  BR2. [Trayectoria A2] 
    \UCpaso El sistema carga la pantalla sp5-pUJD.
    \UCpaso[\UCactor] Selecciona el cargo de los empleados a buscar. 
    \UCpaso[\UCactor]  Presiona el botón de \IUbutton{Buscar}. [Trayectoria A3].
    \UCpaso El sistema despliega la información  de los usuarios (Cargo, nombre, matricula, titulo unidad académica) en la parte inferior de la pantalla [Trayectoria Principal punto 2][Trayectoria A4][Trayectoria A5] .
\end{UCtrayectoria}

\begin{UCtrayectoriaA}{A1}{no existen registros en el catálogo de cargos.}
	\UCpaso 	El sistema muestra el \hyperref[itm:MSGR1]{MSGR1} .
\end{UCtrayectoriaA}

\begin{UCtrayectoriaA}{A2}{No existen  usuarios con el cargo seleccionado.}
	\UCpaso 	El sistema muestra el \hyperref[MSGR2]{MSGR2}.
\end{UCtrayectoriaA}

