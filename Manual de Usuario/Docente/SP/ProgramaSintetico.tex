\section{Registrar Programa Sintético}
Para registrar un Programa Sintético correspondiente a una Unidad de Aprendizaje, primero se da click en el \IUbutton{Ver Tareas} del menú lateral izquierdo.

\begin{figure}[!hbtp]
    \centering
    \includegraphics[width=0.4\linewidth]{images/SP6/VerTareas.png}
    \caption{Botón Ver Tareas} 
\end{figure}

Y la siguiente pantalla es desplegada:

\begin{figure}[H]
    \centering
    \hypertarget{RegLUA}{\includegraphics[width=0.7\linewidth]{images/SP6/PSListado.jpeg}}
    \caption{Pantalla Listado Unidades de Aprendizaje} 
\end{figure}

En la pantalla anterior se muestra un listado de las Unidades de Aprendizaje asignadas al Docente.

Para registrar un Programa Sintético correspondiente a una Unidad de Aprendizaje, el Docente da click sobre \IUbutton{Elaborar} y se muestra la pantalla:


\begin{figure}[H]
    \centering
    \hypertarget{RegPS}{\includegraphics[width=0.7\linewidth]{images/SP6/PSinicio.jpeg}}
    \includegraphics[width=0.7\linewidth]{images/SP6/PSinicio2.jpeg}
    \caption{Pantalla para Registrar Programa Sintético}
\end{figure}

Los campos desplegados en el formulario deberan ser llenados por el Docente.

Si el Docente desea:

\begin{itemize}
    \item Registrar Contenidos. El Docente debe dar click sobre el botón \IUbutton{Registrar Contenidos(*)}. Posteriormente consulte \hyperlink{RegC}{Registrar Contenido}
    \item Registrar Evaluación y Acreditación. El Docente debe dar click sobre el botón \IUbutton{Registrar Evaluación y Acreditación(*)}. Posteriormente consulte \hyperlink{RegEyA}{Registrar Evaluación y Acreditación}
\end{itemize}


Para concluir el registro. Revisar \hyperlink{GuardarFinalizar}{Guardar y/o Finalizar}
Si hay errores checar \hyperlink{Errores}{Posibles Errores}

\pagebreak
\hypertarget{RegC}{\subsection{Registrar Contenido}}

Para registrar un Contenido correspondiente a un Programa Sintético, se debe acceder por medio del botón \IUbutton{Registrar Contenidos(*)} de la pantalla \hyperlink{RegistrarPS}{Registrar Programa Sintético}. Posteriormente se muestra el siguiente formulario: 

\begin{figure}[H]
    \centering
    \hypertarget{RegC}{\includegraphics[width=0.5\linewidth]{images/SP6/11.jpeg}}
    \caption{Pantalla para Registrar Contenido}
\end{figure}

Para concluir el registro. Revisar \hyperlink{AceptarCancelar}{Aceptar y/o Cancelar}
Si hay errores checar

\hyperlink{Errores}{Posibles Errores}

\pagebreak
\hypertarget{RegEyA}{\subsection{Registrar Evaluación y Acreditación}}

Para registrar la Evaluación y Acreditación correspondiente a un Programa Sintético, se debe acceder por medio del botón \IUbutton{Evaluación y Acreditación(*)} de la pantalla \hyperlink{RegistrarPS}{Registrar Programa Sintético}. Posteriormente se muestra el siguiente formulario: 


\begin{figure}[H]
    \centering
    \hypertarget{RegEyA}{\includegraphics[width=0.5\linewidth]{images/SP6/8.jpeg}}
    \caption{Pantalla para Registrar Evaluación y Acreditación}
\end{figure}

Primeramente, el Docente debe seleccionar el tipo de Acreditación. En caso de no estar registrado debe dar click en el botón \IUbutton{Registrar Acreditación}(Consulte \hyperlink{RegA}{Registrar Acreditación}).

Posteriormente, se llena el campo de evaluación con el nombre y el porcentaje asignado. De ser necesario un nuevo tipo de evaluación, el Docente debe hacer click en el botón:

\begin{figure}[!h]
    \centering
    \includegraphics[width=0.3\linewidth]{images/SP6/BotonEval.jpeg}
    \caption{Botón Agregar Evaluación} 
\end{figure}

Y se desplegara un nuevo campo para el nombre de la evaluación y un nuevo campo para el procentaje de dicha evaluación.

Si el Docente desea eliminar la ultima evaluación agregada deberá dar click al botón: 


\begin{figure}[H]
    \centering
    \includegraphics[width=0.3\linewidth]{images/SP6/BotonEliEval.jpeg}
    \caption{Botón Eliminar Evaluación} 
\end{figure}

En caso de que los porcentajes de evaluación y acreditación no sumen 100\%, se muestra el mensaje:


\begin{figure}[H]
    \centering
    \hypertarget{XXX}{\includegraphics[width=0.5\linewidth]{images/SP6/ErrorPorcentaje.jpeg}}
    \caption{Pantalla para Registrar Acreditación}
\end{figure}

De esta manera, el Docente debe corregir los datos para que sumen un total de 100\%.

Para concluir el registro. Revisar \hyperlink{AceptarCancelar}{Aceptar y/o Cancelar}
Si hay errores checar \hyperlink{Errores}{Posibles Errores}

\pagebreak
\hypertarget{RegA}{\subsection{Registrar Acreditación}}

Para registrar un nuevo tipo Acreditación correspondiente a un Programa Sintético, se debe acceder por medio del botón \IUbutton{Registrar Acreditación} de la pantalla \hyperlink{RegEyA}{Registrar Evaluación y Acreditación}. Posteriormente se muestra el siguiente formulario: 


\begin{figure}[H]
    \centering
    \hypertarget{RegA}{\includegraphics[width=0.5\linewidth]{images/SP6/9.jpeg}}
    \caption{Pantalla para Registrar Acreditación}
\end{figure}



Para concluir el registro. Revisar \hyperlink{AceptarCancelar}{Aceptar y/o Cancelar}
Si hay errores checar \hyperlink{Errores}{Posibles Errores}