% -------------------------------------------------------------------------------------------------------
%-----------------------------CONSULTA DE PLANES DE ESTUDIO JEFE DE DEPARTAMENTO DE DESARROLLO E INOVACIÓN CURRICULAR---------------------------------
% -------------------------------------------------------------------------------------------------------

\section{Gestión de Plan de Estudios}
\subsection{Consulta de Planes de Estudios}
Cuando el Jefe de Departamento de Desarrollo e Innovación Curricular da clic a un programa académico en la sección de \textbf{Gestionar Programas Académicos} aparecerá la siguiente pantalla:

% Imagen menu

\begin{figure}[H]
	\centering
	\hypertarget{consultarPE}{\includegraphics[width=0.7\linewidth]{images/SP4-GPE/consultar}}
	\caption{Pantalla para Planes de Estudio}
	\label{consultarPE}
\end{figure}

En donde aparecerá, de forma predeterminada, un listado de todos los Planes de Estudios a su cargo registrados en el sistema al momento. Tendrá a su disposición tres funciones:

\subsubsection{Buscar Planes de Estudio según el programa académico}

Para ello, el Jefe de Departamento de Desarrollo e Innovación Curricular tendrá que seleccionar el Programa Académico que desea consultar en el siguiente componente:

\begin{figure}[H]
	\centering
	\hypertarget{academico}{\includegraphics[width=0.7\linewidth]{images/SP4-GPE/programa}}
	\caption{Selección de Programa Académico}
	\label{academico}
\end{figure}

\begin{figure}[H]
	\centering
	\hypertarget{academico2}{\includegraphics[width=0.7\linewidth]{images/SP4-GPE/programaD}}
	\caption{Despliegue de Programas Académicos}
	\label{academico2}
\end{figure}

A continuación el sistema mostrará todos los Planes de Estudios que tengan el Programa Acádemico seleccionado.
\begin{figure}[H]
	\centering
	\hypertarget{planes}{\includegraphics[width=0.7\linewidth]{images/SP4-GPE/consultar}}
	\caption{Planes de Estudios Encontrados}
	\label{planes}
\end{figure}
\newpage
\subsection{Editar Planes de Estudios}

Para ello, el Jefe de Departamento de Desarrollo e Innovación Curricular tendrá que dar clic en el boton \IUbutton{Lápiz} que esta al lado del Plan de Estudio que desea modificar. Al hacer esto, el sistema redireccionará al usuario a la pantalla de \hyperlink{editarPE}{\textit{Editar Plan de Estudio}}.

\begin{figure}[H]
	\centering
	\hypertarget{editar}{\includegraphics[width=0.7\linewidth]{images/SP4-GPE/editarC}}
	\caption{Botón Editar Plan de Estudio}
	\label{editar}
\end{figure}

\begin{figure}[H]
	\centering
	\hypertarget{editarPE}{\includegraphics[width=0.7\linewidth]{images/SP4-GPE/editarPE}}
	\caption{Pantalla para la edición de Planes de Estudios}
	\label{editarPE}
\end{figure}

En donde se cargarán los datos del Plan de Estudio seleccionado en la pantalla de \hyperlink{consultarPE}{\textit{Consultar Planes de Estudios}} y llenará el formulario.
\newpage
A continuación, el usuario puede modificar todos los campos del Plan de Estudio.
\begin{figure}[H]
	\centering
	\hypertarget{modif}{\includegraphics[width=0.7\linewidth]{images/SP4-GPE/editarPE1}}
	\caption{Datos del Plan de Estudio modificados}
	\label{modif}
\end{figure}

 Si el Jefe de Departamento de Desarrollo e Innovación Curricular da clic en el botón \IUbutton{Cancelar} sin haber concluido la edición del Plan de Estudio:

\begin{figure}[H]
	\centering
	\hypertarget{cancel2}{\includegraphics[width=0.7\linewidth]{images/SP4-GPE/cancelarPE}}
	\caption{Botón ''Cancelar''}
	\label{cancel2}
\end{figure}
\newpage

El sistema mostrará el siguiente mensaje:
\begin{figure}[H]
	\centering
	\hypertarget{ms1}{\includegraphics[width=0.7\linewidth]{images/SP4-GPE/m1}}
	\caption{Mensaje de cancelar}
	\label{ms1}
\end{figure}

Para confirmar, el usuario debe dar clic en el botón  \IUbutton{Si}, y el Plan de Estudio no será modificado.\\

Para cancelar, el usuario debe dar clic en botón  \IUbutton{No}, el mensaje se cerrará y continuaremos en el formulario. Aqui el Jefe de Departamento de Desarrollo e Innovación Curricular puede terminar la edición del Plan de Estudio.\\

Una vez modificados los datos, deberá de dar clic en el botón  \IUbutton{Finalizar}.
\begin{figure}[H]
	\centering
	\hypertarget{btnfin}{\includegraphics[width=0.7\linewidth]{images/SP4-GPE/editarPER}}
	\caption{Botón ''Finalizar''}
	\label{btnfin}
\end{figure}

Si no hubieron errores, el sistema muestra el siguiente mensaje:
%Imagen MSG27
\begin{figure}[H]
	\centering
	\hypertarget{ms2}{\includegraphics[width=0.7\linewidth]{images/SP4-GPE/m2}}
	\caption{Mensaje de modificar datos}
	\label{ms2}
\end{figure}


Al dar clic en en el botón  \IUbutton{Aceptar}, el sistema redireccionará al usuario a la pantalla de \hyperlink{consultarPE}{\textit{Consultar Planes de Estudios}}, en donde podrá ver las modificaciones del Plan de Estudios.\\
\newpage
\subsubsection{Posibles errores}

\begin{itemize}
	\item Problemas con la conexión o el sistema

	Si al momento de acceder a la pantalla de \hyperlink{editarPE}{\textit{Editar Plan de Estudio}} o al intentar modificar un Plan de Estudio, aparece alguno de los siguientes mensajes:

	% Imagen MSG7 Y MSG25
	\begin{figure}[H]
		\centering
		\hypertarget{ms3}{\includegraphics[width=0.7\linewidth]{images/SP4-GPE/m3}}
		\caption{Mensaje de ningun plan de estudios registrado}
		\label{ms3}
	\end{figure}
	\begin{figure}[H]
		\centering
		\hypertarget{error}{\includegraphics[width=0.7\linewidth]{images/SP4-GPE/error}}
		\caption{Servicios no disponibles}
		\label{error}
	\end{figure}


	Significa que hubo un error de conexión. Al dar clic en en botón  \IUbutton{Aeptar}, el sistema redireccionará al usuario a la pantalla de \hyperlink{consultarPE}{\textit{Consultar Planes de Estudios}}.
	\newpage

	\item Campos vacíos al momento de modificar el Plan de Estudio

	Si el Jefe de Departamento de Desarrollo e Innovación Curricular dejo en blanco algún campo del formulario, y posteriormente dio clic en el botón  \IUbutton{Registrar}, el sistema mostrará el siguiente mensaje:

	% Imagen MSG32
		\begin{figure}[H]
		\centering
		\hypertarget{ms4}{\includegraphics[width=0.7\linewidth]{images/SP4-GPE/m4}}
		\caption{Campo Obligatorio}
		\label{ms4}
	\end{figure}

	Al dar clic en botón  \IUbutton{Aceptar}, el mensaje se cerrará y regresaremos al formulario, en donde el usuario deberá llenar el o los campos que dejo vacío. Si se continúan dejando campos en blanco y dando clic en el botón  \IUbutton{Registrar}, aparecerá nuevamente el mensaje, hasta que todos los campos sean llenados.


	\item Los campos ingresados no son válidos

	Si al momento de dar clic en el \IUbutton{Registrar} aparece el siguiente mensaje:
	% Imagen MSG20
	\begin{figure}[H]
		\centering
		\hypertarget{ms5}{\includegraphics[width=0.7\linewidth]{images/SP4-GPE/m5}}
		\caption{Datos no válidos}
		\label{ms5}
	\end{figure}

	Significa que la composición de los datos ingresados en el formulario no es la correcta, verifíquelos e intente de nuevo.

\end{itemize}


\newpage
\subsection{Registrar Planes de Estudios}

Para ello, el Jefe de Departamento de Desarrollo e Innovación Curricular deberá dar clic en el botón \IUbutton{+} en la parte inferior de la pantalla.

\begin{figure}[H]
	\centering
	\hypertarget{add}{\includegraphics[width=0.7\linewidth]{images/SP4-GPE/mas}}
	\caption{Botón Agregar Plan de Estudio}
	\label{add}
\end{figure}

Al hacerlo, el sistema redireccionará al usuario a la pantalla de \hyperlink{registrarPE}{\textit{Registrar Plan de Estudio}}.


\textbf{NOTA:} En caso de que exista un error de conexión, aparecerá el siguiente mensaje:
%Imagen del MSG7
	\begin{figure}[H]
	\centering
	\hypertarget{error}{\includegraphics[width=0.7\linewidth]{images/SP4-GPE/error}}
	\caption{Servicios no disponibles}
	\label{error}
\end{figure}

Al dar clic en en botón \IUbutton{Aceptar}, el sistema redireccionará al usuario a la pantalla de \hyperlink{registrarPE}{\textit{Registrar Plan de Estudio}}.
\newpage
\subsubsection{Registro de Plan de Estudio}
Si el Jefe de Departamento de Desarrollo e Innovación Curricular en la pantalla de \hyperlink{consultarPE}{\textit{Consultar Planes de Estudios}} dio clic en el botón \IUbutton{+}, aparece la siguiente pantalla:

\begin{figure}[H]
	\centering
	\hypertarget{registrarPE}{\includegraphics[width=0.7\linewidth]{images/SP4-GPE/registrarPE}}
	\caption{Pantalla para registrar Planes de Estudio}
	\label{registrarPE}
\end{figure}
\newpage
En donde tendrá que ingresar los campos del nuevo Plan de Estudio o en el formulario. Un ejemplo del llenado sería el siguiente:

\begin{figure}[H]
	\centering
	\hypertarget{ejreg}{\includegraphics[width=0.7\linewidth]{images/SP4-GPE/registrarEjem}}
	\caption{Ejemplo de llenado para agregar un nuevo Plan de Estudios}
	\label{ejreg}
\end{figure}
%\newpage
Si el Jefe de Departamento de Desarrollo e Innovación Curricular da clic en el botón \IUbutton{Cancelar} sin haber concluido el registro del Plan de Estudio:

\begin{figure}[H]
	\centering
	\hypertarget{cancel2}{\includegraphics[width=0.7\linewidth]{images/SP4-GPE/cancelarPE}}
	\caption{Botón Cancelar}
	\label{cancel2}
\end{figure}
\newpage

El sistema mostrará el siguiente mensaje:
%Imagen MSG30
\begin{figure}[H]
	\centering
	\hypertarget{ms1}{\includegraphics[width=0.7\linewidth]{images/SP4-GPE/m1}}
	\caption{Mensaje de cancelar}
	\label{ms1}
\end{figure}

Para confirmar, el usuario debe dar clic en el botón  \IUbutton{Si}, y el Plan de Estudio no será registrado.\\

Para cancelar, el usuario debe dar clic en botón  \IUbutton{No}, el mensaje se cerrará y continuaremos en el formulario. Aqui el Docente puede terminar la edición del Plan de Estudio.

A continuación, una vez verificados los datos, deberá de dar clic en el botón \IUbutton{Registrar}.
\begin{figure}[H]
	\centering
	\hypertarget{btnreg}{\includegraphics[width=0.7\linewidth]{images/SP4-GPE/registrarB}}
	\caption{Botón Registrar}
	\label{btnreg}
\end{figure}

Si no hubieron errores, el sistema muestra el siguiente mensaje:
% Imagen MSG5
	\begin{figure}[H]
	\centering
	\hypertarget{exito}{\includegraphics[width=0.7\linewidth]{images/SP4-GPE/exito}}
	\caption{Registro exitoso}
	\label{exito}
\end{figure}

Al dar clic en en botón \IUbutton{Aceptar}, el sistema redireccionará al usuario a la pantalla de \hyperlink{consultarPE}{\textit{Consultar Planes de Estudios}}, en donde podrá ver el nuevo Plan de Estudios agregado.\\
\newpage
\subsubsection{Posibles errores}
\begin{itemize}

	\item Problemas con la conexión o el sistema

	Si al momento de acceder a la pantalla de \hyperlink{registrarPE}{\textit{Registrar Plan de Estudios}}, aparece alguno de los siguientes mensajes:
	%Imagen MSG7 Y MSG25
		\begin{figure}[H]
		\centering
		\hypertarget{error}{\includegraphics[width=0.7\linewidth]{images/SP4-GPE/error}}
		\caption{Servicios no disponibles}
		\label{error}
		\end{figure}


	Significa que existió un error de conexión o del sistema. Al dar clic en en botón \IUbutton{Aceptar}, el sistema redireccionará al usuario a la pantalla de \hyperlink{consultarPE}{\textit{Consultar Planes de Estudios}}. Debe esperar a que la página este disponible o intentar acceder nuevamente.
	\newpage
	\item Plan de Estudios en proceso

	Si al momento de registrar un nuevo Plan de Estudio, aparece alguno de los siguientes mensajes:
	%Imagen MSG7 Y MSG25
	\begin{figure}[H]
		\centering
		\hypertarget{error1}{\includegraphics[width=0.7\linewidth]{images/SP4-GPE/error1}}
		\caption{Plan de Estudio en proceso}
		\label{error1}
	\end{figure}


	Significa que ya hay un Plan de Estudio en proceoso. Al dar clic en en botón \IUbutton{Aceptar}, el sistema redireccionará al usuario a la pantalla de \hyperlink{registrarPE}{\textit{Registrar Planes de Estudios}}.

	\item Campos vacíos al momento de agregar un nuevo Plan de Estudio

	Si el Docente dejo en blanco algún campo del formulario, y posteriormente dio clic en el botón \IUbutton{Registrar}, el sistema mostrará el siguiente mensaje:
	%Imagen MSG32
		\begin{figure}[H]
		\centering
		\hypertarget{ms4}{\includegraphics[width=0.7\linewidth]{images/SP4-GPE/m4}}
		\caption{Campos obligatorios}
		\label{ms4}
	    \end{figure}

	Al dar clic en botón \IUbutton{Aceptar}, el mensaje se cerrará y regresaremos al formulario, en donde el usuario deberá llenar el o los campos que dejo vacío. Si se continúan dejando campos en blanco y dando clic en el botón \IUbutton{Registrar}, aparecerá nuevamente el mensaje, hasta que todos los campos sean llenados.\\

	\newpage

	\item Los campos ingresados no son válidos

	Si al momento de dar clic en el botón \IUbutton{Registrar} aparece el siguiente mensaje:
	%Imagen MSG20
	\begin{figure}[H]
		\centering
		\hypertarget{ms5}{\includegraphics[width=0.7\linewidth]{images/SP4-GPE/m5}}
		\caption{Datos no válidos}
		\label{ms5}
	\end{figure}

\end{itemize}
