% -------------------------------------------------------------------------------------------------------
%-----------------------------CONSULTA DE PLANES DE ESTUDIO---------------------------------
% -------------------------------------------------------------------------------------------------------

\chapter{Gestión de Plan de Estudios}
\section{Consulta de Planes de Estudios}
Cuando el Jefe de Innovación Educativa da clic a un programa académico en la sección de \textbf{Gestionar Programas Académicos} aparecerá la siguiente pantalla:

% Imagen menu

\begin{figure}[!hbtp]
	\centering
	\hypertarget{consultarPE}{\includegraphics[width=0.7\linewidth]{images/SP4-GPE/consultar}}
	\caption{Pantalla para Planes de Estudio}
	\label{consultarPE}
\end{figure}

En donde aparecerá, de forma predeterminada, un listado de todos los Planes de Estudios a su cargo registrados en el sistema al momento. Tendrá a su disposición tres funciones:

\subsection{Buscar Planes de Estudio según su cargo}

Para ello, el Jefe de Innovación Educativa tendrá que seleccionar el Programa Académico que desea consultar en el siguiente componente:

\begin{figure}[!hbtp]
	\centering
	\hypertarget{academico}{\includegraphics[width=0.7\linewidth]{images/SP4-GPE/programa}}
	\caption{Selección de Programa Académico}
	\label{academico}
\end{figure}

\begin{figure}[!hbtp]
	\centering
	\hypertarget{academico2}{\includegraphics[width=0.7\linewidth]{images/SP4-GPE/programaD}}
	\caption{Despliegue de Programas Académicos}
	\label{academico2}
\end{figure}

A continuación el sistema mostrará todos los Planes de Estudios que tengan el Programa Acádemico seleccionado.
\begin{figure}[!hbtp]
	\centering
	\hypertarget{planes}{\includegraphics[width=0.7\linewidth]{images/SP4-GPE/planes}}
	\caption{Planes de Estudios Encontrados}
	\label{planes}
\end{figure}
\newpage
\section{Editar Planes de Estudios}

Para ello, el Jefe de Innovación Educativa tendrá que dar clic en el boton \IUbutton{Lápiz} que esta al lado del Plan de Estudio que desea modificar. Al hacer esto, el sistema redireccionará al usuario a la pantalla de \hyperlink{editarPE}{\textit{Editar Plan de Estudio}}.

\begin{figure}[!hbtp]
	\centering
	\hypertarget{editar}{\includegraphics[width=0.7\linewidth]{images/SP4-GPE/editarC}}
	\caption{Botón Editar Plan de Estudio}
	\label{editar}
\end{figure}
\newpage
\begin{figure}[!hbtp]
	\centering
	\hypertarget{editarPE}{\includegraphics[width=0.7\linewidth]{images/SP4-GPE/editarPE}}
	\caption{Pantalla para la edición de Planes de Estudios}
	\label{editarPE}
\end{figure}

En donde se cargarán los datos del Plan de Estudio seleccionado en la pantalla de \hyperlink{consultarPE}{\textit{Consultar Planes de Estudios}} y llenará el formulario.
\newpage
A continuación, el usuario puede modificar todos los campos del Plan de Estudio.
\begin{figure}[!hbtp]
	\centering
	\hypertarget{modif}{\includegraphics[width=0.7\linewidth]{images/SP4-GPE/editarPE1}}
	\caption{Datos del Plan de Estudio modificados}
	\label{modif}
\end{figure}

 Si el Jefe de Innovación Educativa da clic en el botón \IUbutton{Cancelar} sin haber concluido la edición del Plan de Estudio:

\begin{figure}[!hbtp]
	\centering
	\hypertarget{cancel2}{\includegraphics[width=0.7\linewidth]{images/SP4-GPE/cancelarPE}}
	\caption{Botón ''Cancelar''}
	\label{cancel2}
\end{figure}

El sistema mostrará el siguiente mensaje:
%Imagen MSG30

Para confirmar, el usuario debe dar clic en el botón  \IUbutton{Si}, y el Plan de Estudio no será modificado.\\

Para cancelar, el usuario debe dar clic en botón  \IUbutton{No}, el mensaje se cerrará y continuaremos en el formulario. Aqui el Jefe de Innovación Educativa puede terminar la edición del Plan de Estudio.\\
\newpage
Una vez modificados los datos, deberá de dar clic en el botón  \IUbutton{Registrar}.
\begin{figure}[!hbtp]
	\centering
	\hypertarget{btnfin}{\includegraphics[width=0.7\linewidth]{images/SP4-GPE/editarPER}}
	\caption{Botón ''Registrar''}
	\label{btnfin}
\end{figure}

Si no hubieron errores, el sistema muestra el siguiente mensaje:
%Imagen MSG27

Al dar clic en en botón  \IUbutton{Aceptar}, el sistema redireccionará al usuario a la pantalla de \hyperlink{consultarPE}{\textit{Consultar Planes de Estudis}}, en donde podrá ver las modificaciones del Plan de Estudios.\\
\newpage
\subsection{Posibles errores}

\begin{itemize}
	\item Problemas con la conexión o el sistema

	Si al momento de acceder a la pantalla de \hyperlink{editarPE}{\textit{Editar Plan de Estudio}} o al intentar modificar un Plan de Estudio, aparece alguno de los siguientes mensajes:

	% Imagen MSG7 Y MSG25

	Significa que existió un error de conexión o del sistema. Al dar clic en en botón  \IUbutton{Aeptar}, el sistema redireccionará al usuario a la pantalla de \hyperlink{consultarPE}{\textit{Consultar Planes de Estudios}}. Debe esperar a que la página este disponible o intentar acceder nuevamente.

	\item Campos vacíos al momento de modificar el Plan de Estudio

	Si el Jefe de Innovación Educativa dejo en blanco algún campo del formulario, y posteriormente dio clic en el botón  \IUbutton{Registrar}, el sistema mostrará el siguiente mensaje:

	% Imagen MSG32

	Al dar clic en botón  \IUbutton{Aceptar}, el mensaje se cerrará y regresaremos al formulario, en donde el usuario deberá llenar el o los campos que dejo vacío. Si se continúan dejando campos en blanco y dando clic en el botón  \IUbutton{Registrar}, aparecerá nuevamente el mensaje, hasta que todos los campos sean llenados.

	\item Los campos ingresados no son válidos

	Si al momento de dar clic en el \IUbutton{Registrar} aparece el siguiente mensaje:
	% Imagen MSG20

	Significa que la composición de los datos ingresados en el formulario no es la correcta, verifíquelos e intente de nuevo.

\end{itemize}


\newpage
\section{Registrar Planes de Estudios}

Para ello, el Docente deberá dar clic en el botón \IUbutton{+} en la parte inferior de la pantalla.

\begin{figure}[!hbtp]
	\centering
	\hypertarget{add}{\includegraphics[width=0.7\linewidth]{images/SP4-GPE/mas}}
	\caption{Botón Agregar Plan de Estudio}
	\label{add}
\end{figure}

Al hacerlo, el sistema redireccionará al usuario a la pantalla de \hyperlink{registrarPE}{\textit{Registrar Plan de Estudio}}.


\textbf{NOTA:} En caso de que exista un error de conexión, aparecerá el siguiente mensaje:
%Imagen del MSG7

Al dar clic en en botón \IUbutton{Aceptar}, el sistema redireccionará al usuario a la pantalla de inicio. Debe esperar a que la página este disponible o intentar acceder nuevamente.
\newpage
\subsection{Registro de Plan de Estudio}
Si el Docente en la pantalla de \hyperlink{consultarPE}{\textit{Consultar Planes de Estudios}} dio clic en el botón \IUbutton{+}, aparece la siguiente pantalla:

\begin{figure}[!hbtp]
	\centering
	\hypertarget{registrarPE}{\includegraphics[width=0.7\linewidth]{images/SP4-GPE/registrarPE}}
	\caption{Pantalla para registrar Planes de Estudio}
	\label{registrarPE}
\end{figure}
\newpage
En donde tendrá que ingresar los campos del nuevo Plan de Estudio o en el formulario. Un ejemplo del llenado sería el siguiente:

\begin{figure}[!hbtp]
	\centering
	\hypertarget{ejreg}{\includegraphics[width=0.7\linewidth]{images/SP4-GPE/registrarEjem}}
	\caption{Ejemplo de llenado para agregar un nuevo Plan de Estudios}
	\label{ejreg}
\end{figure}
\newpage
Si el Docente  da clic en el botón \IUbutton{Cancelar} sin haber concluido el registro del Plan de Estudio:

\begin{figure}[!hbtp]
	\centering
	\hypertarget{cancel2}{\includegraphics[width=0.7\linewidth]{images/SP4-GPE/cancelarPE}}
	\caption{Botón Cancelar}
	\label{cancel2}
\end{figure}

El sistema mostrará el siguiente mensaje:
%Imagen MSG30

Para confirmar, el usuario debe dar clic en el botón  \IUbutton{Si}, y el Plan de Estudio no será registrado.\\

Para cancelar, el usuario debe dar clic en botón  \IUbutton{No}, el mensaje se cerrará y continuaremos en el formulario. Aqui el Docente puede terminar la edición del Plan de Estudio.

A continuación, una vez verificados los datos, deberá de dar clic en el botón \IUbutton{Registrar}.
\begin{figure}[!hbtp]
	\centering
	\hypertarget{btnreg}{\includegraphics[width=0.7\linewidth]{images/SP4-GPE/registrarB}}
	\caption{Botón Registrar}
	\label{btnreg}
\end{figure}

Si no hubieron errores, el sistema muestra el siguiente mensaje:
% Imagen MSG5

Al dar clic en en botón \IUbutton{Aceptar}, el sistema redireccionará al usuario a la pantalla de \hyperlink{consultarPE}{\textit{Consultar Planes de Estudios}}, en donde podrá ver el nuevo Plan de Estudios agregado.\\
\newpage
\subsection{Posibles errores}
\begin{itemize}
	
	\item Problemas con la conexión o el sistema
	
	Si al momento de acceder a la pantalla de \hyperlink{registrarPE}{\textit{Registrar Plan de Estudios}} o al intentar registrar un Plan de Estuido, aparece alguno de los siguientes mensajes:
	%Imagen MSG7 Y MSG25
	
	Significa que existió un error de conexión o del sistema. Al dar clic en en botón \IUbutton{Aceptar}, el sistema redireccionará al usuario a la pantalla de \hyperlink{consultarPE}{\textit{Consultar Planes de Estudios}}. Debe esperar a que la página este disponible o intentar acceder nuevamente.
	
	\item Campos vacíos al momento de agregar un nuevo Plan de Estudio
	
	Si el Docente dejo en blanco algún campo del formulario, y posteriormente dio clic en el botón \IUbutton{Registrar}, el sistema mostrará el siguiente mensaje:
	%Imagen MSG32
	
	Al dar clic en botón \IUbutton{Aceptar}, el mensaje se cerrará y regresaremos al formulario, en donde el usuario deberá llenar el o los campos que dejo vacío. Si se continúan dejando campos en blanco y dando clic en el botón \IUbutton{Registrar}, aparecerá nuevamente el mensaje, hasta que todos los campos sean llenados.\\
	
	
	
	\item Los campos ingresados no son válidos
	
	Si al momento de dar clic en el botón \IUbutton{Registrar} aparece el siguiente mensaje:
	%Imagen MSG20
	
	
\end{itemize}

