\chapter{Gestión de Unidades de Aprendizaje}
    \section{Consulta de Unidades de Aprendizaje}
Para consultar Unidades de Aprendizaje el Jefe de Innovación Educativa consulta un Plan de Estudios en la pantalla \hyperlink{consultarS}{\textit{Consultar Planes de Estudios}}:\\
\begin{figure}[!hbtp]
    \centering
    \hypertarget{consultarS}{\includegraphics[width=0.7\linewidth]{images/GUA/consultarS}}
    \caption{Pantalla Consultar Plan de Estudios}
    \label{consultarS}
\end{figure}
Una vez en esta pantalla el usuario selecciona un semestre y este se expande desplegando las Unidades de Aprendizaje que tiene asociadas en la siguiente pantalla \hyperlink{consultarUA}{\textit{Consultar Planes de Estudios}}:\\
\begin{figure}[!hbtp]
    \centering
    \hypertarget{consultarUA}{\includegraphics[width=0.7\linewidth]{images/GUA/consultarUA}}
    \caption{Pantalla Consultar Plan de Estudios Semestre expandido}
    \label{consultarUA}
\end{figure}
En esta ultima pantalla podemos realizar las tareas relacionadas a la gestión de Unidades de Aprendizaje, tendrá a su dispocisión las siguientes funciones relacionadas a los iconos en recuadros de colores para Registrar Unidad de Aprendizaje (verde), Editar Unidad de Aprendizaje (azul) y Eliminar Unidad de Aprendizaje (rojo).
\newpage
\section{Registrar Unidades de Aprendizaje}
Cuando el Jefe de Innovación Educativa presiona el icono de \IUbutton{+} en el recuadro verde en la pantalla \hyperlink{consultarUA}{\textit{Consultar Planes de Estudios}} lo lleva a la siguiente pantalla la siguiente pantalla \hyperlink{registrarUA}{\textit{Registrar Unidad de Aprendizaje}}:\\
\begin{figure}[!hbtp]
    \centering
    \hypertarget{registrarUA}{\includegraphics[width=0.7\linewidth]{images/GUA/registrarUA}}
    \caption{Pantalla Registrar Unidad de Aprendizaje}
    \label{registrarUA}
\end{figure}
\newpage
En donde tendrá que ingresar los campos de la nueva Unidad de Aprendizaje en el formulario para ser registrada.Durante el registro el sistema puede lanzar los siguientes mensajes:
\begin{figure}[!hbtp]
    \centering
    \hypertarget{invalidoR}{\includegraphics[width=0.7\linewidth]{images/GUA/invalido}}
    \caption{Campo invalido: aparece cuando el formato o el tipo de dato es incorrecto}
    \label{invalidoR}
\end{figure}
\begin{figure}[!hbtp]
    \centering
    \hypertarget{requeridoR}{\includegraphics[width=0.7\linewidth]{images/GUA/requerido}}
    \caption{Campo requerido: aparece cuando un campo obligatorio se dejo vacio}
    \label{requeridoR}
\end{figure}
\begin{figure}[!hbtp]
    \centering
    \hypertarget{exito}{\includegraphics[width=0.7\linewidth]{images/GUA/exito}}
    \caption{Regisro Exitoso: aparece cuando su registro finaliza sin errores}
    \label{exito}
\end{figure}
\begin{figure}[!hbtp]
    \centering
    \hypertarget{cancelarR}{\includegraphics[width=0.7\linewidth]{images/GUA/cancelar}}
    \caption{Cancelación: este mensaje nos permite finalizar la operacion sin guardar}
    \label{cancelarR}
\end{figure}
\newpage
\section{Editar Unidades de Aprendizaje}
Cuando el Jefe de Innovación Educativa presiona el icono de \IUbutton{lápiz} en el recuadro azul en la pantalla \hyperlink{consultarUA}{\textit{Consultar Planes de Estudios}} lo lleva a la siguiente pantalla la siguiente pantalla \hyperlink{editarUA}{\textit{Editar Unidad de Aprendizaje}}:\\
\begin{figure}[!hbtp]
    \centering
    \hypertarget{editarUA}{\includegraphics[width=0.7\linewidth]{images/GUA/editarUA}}
    \caption{Pantalla Editar Unidad de Aprendizaje}
    \label{editarUA}
\end{figure}
En donde se cargaran los datos de la Unidad de Aprendizaje seleccionada en la pantalla de \hyperlink{consultarUA}{\textit{Consultar Planes de Estudios}}  y llenará el formulario.\\
Durante la edición el sistema puede lanzar los siguientes mensajes:\\
\begin{figure}[!hbtp]
    \centering
    \hypertarget{invalidoE}{\includegraphics[width=0.7\linewidth]{images/GUA/invalido}}
    \caption{Campo invalido: aparece cuando el formato o el tipo de dato es incorrecto}
    \label{invalidoE}
\end{figure}
\begin{figure}[!hbtp]
    \centering
    \hypertarget{requeridoE}{\includegraphics[width=0.7\linewidth]{images/GUA/requerido}}
    \caption{Campo requerido: aparece cuando un campo obligatorio se dejo vacio}
    \label{requeridoE}
\end{figure}
\begin{figure}[!hbtp]
    \centering
    \hypertarget{modificacion}{\includegraphics[width=0.7\linewidth]{images/GUA/modificacion}}
    \caption{Modificación exitosa: mensaje puramente de notificación}
    \label{modificacion}
\end{figure}
\begin{figure}[!hbtp]
    \centering
    \hypertarget{cancelarE}{\includegraphics[width=0.7\linewidth]{images/GUA/cancelar}}
    \caption{Cancelación: este mensaje nos permite finalizar la operacion sin guardar}
    \label{cancelarE}
\end{figure}
\newpage
\section{Eliminar Unidades de Aprendizaje}
Cuando el Jefe de Innovación Educativa quiera eliminar una Unidad de Aprendizaje presiona el icono de \IUbutton{X} en el recuadro rojo en la pantalla \hyperlink{consultarUA}{\textit{Consultar Planes de Estudios}} donde aparece:\\
\begin{figure}[!hbtp]
    \centering
    \hypertarget{EliminarUA}{\includegraphics[width=0.7\linewidth]{images/GUA/EliminarUA}}
    \caption{Eliminar Unidad de Aprendizaje: solicita confirmación para eliminar permanentemente una Unidad de Aprendizaje}
    \label{EliminarUA}
\end{figure}