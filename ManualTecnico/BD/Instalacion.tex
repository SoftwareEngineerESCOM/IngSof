\chapter{Instalación}

\subsection{Propósito}
Este manual tiene como fin, dar una pequeña guía, de modo que, el personal encargado de dar mantenimiento o hacer actualizaciones pueda tener un entorno de trabajo completo y pueda poner en producción los cambios realizados.

\subsection{Instalación del entorno de trabajo}
Para llevar a cabo el proceso de instalación, el área de trabajo tiene que tener instalado:
\begin{itemize}
  \item JDK 1.8
  \item JRE 8
  \item Maven 3.6
\end{itemize}

Seguido de estos paquetes, se describen las herramientas y tecnologías que requiere para el entorno del trabajo del sistema:
\begin{itemize}
  \item Spring tools suite 3+
  \item Angular 6
  \item Postgres 11
  \item Ubuntu 18.04 LTS
\end{itemize}
De no tener java y maven instalado ejecute el archivo instaladorJ.sh, para que sea instalado el entorno completo.
Si ya tiene java y maven, ejecute el archivo instalador.sh, estos archivos se encargarán de ejecutar e instalar todo el entorno de trabajo.
Para postgres, se tiene un apartado en las hojas siguientes ya que es de vital importancia un entorno de desarrollo limpio para la base de datos.

Para ejecutar alguno de los scripts se debe de ejecutar el siguiente comando en una terminal a la cual se accede con la combinación de teclas: Ctrl+alt+t

\begin{figure}[H]
  \centering
  \includegraphics[width=0.1\linewidth]{images/tecnologias/instalarJ.png}
  \caption{Linea de comando que ejecuta instalarJ.sh para instalar el entorno de trabajo con java.}
\end{figure}

\begin{figure}[H]
  \centering
  \includegraphics[width=0.1\linewidth]{images/tecnologias/instalar.png}
  \caption{Linea de comando que ejecuta intalar.sh para instalar el entorno de trabajo.}
\end{figure}


También se necesita tener una cuenta de github y una cuenta en el servicio de heroku para hacer el levantamiento de producción. Para crear una cuenta diríjase a los links presentados:
\begin{itemize}
  \item \href{htt://codigofacilito.com/articulos/como-crear-una-cuenta-y-un-repo-en-github}{Github}
  \item \href{https://signup.heroku.com/dc}{Heroku}
\end{itemize}
