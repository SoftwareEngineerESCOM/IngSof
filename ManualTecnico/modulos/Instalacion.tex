\section{Instalación}

\begin{figure}[!h]
\raggedright
\includegraphics[width=0.1\linewidth]{images/tecnologias/spring.png}
\end{figure}

\begin{figure}[!h]
\raggedleft
\includegraphics[width=0.1\linewidth]{images/tecnologias/angular.png}
\end{figure}


Para poder instalar el sistema se requiere:

\begin{itemize}
	\item Spring tools suite 4+
  \item Angular 6+
  \item Postgres 11+
  \item Google Chrome versión 11+
  \item Ubuntu 18.04 LTS
\end{itemize}

\subsection{Spring tools suite}
Para instalar spring tools suite en linux se siguen los siguiente pasos:

\begin{enumerate}
  \item Instalar el JDK de java 1.8.1 que se encuentra en la carpeta APMSEjecutables/java.
  \item Ejecutar el archivo STS dentro de la ruta APMSEjecutables/spring con el comando desde la consola. \textit{./STS}
\end{enumerate}

\subsection{Angular}
Para instalar angular en linux se siguen los siguiente pasos:

\begin{enumerate}
  \item Instalar node js con los siguiente comandos:
  \textit{sudo apt-get install nodejs} 
  \newline\textit{sudo apt-get install npm}
  \item Abrir una consola y ejecutar el comando:
  \textit{npm install -g @angular/cl} 
\end{enumerate}

\subsection{Postgres}
Para instalar postgres en linux se siguen los siguiente pasos:

\begin{enumerate}
  \item Ejecutar el comando en una consola. 
  \textit{sudo apt-get install postgresql postgresql-contrib} 
  \item Para acceder al servidor de postgres se ejecuta:
  \textit{sudo -i -u postgres}seguido de:\textit{psql} 
  \item Crear la base de datos con el siguiente comando: 
  \textit{postgres: create database apms;} 
  \item Para salir del terminal se ejecuta:
  \textit{postgres: $\backslash$q''} 
\end{enumerate}

