% -------------- TABLA PARA REQUERIMIENTOS FUNCIONALES ---------------- % 
% Nomenclatura para la prioridad: 
%	A - Alta
%	M - Media
%	B - Baja

\begin{table}[htbp!]
	\begin{requerimientos}
		\FRitem{RF1}{Cálculo de horas}{El sistema calcula el número de horas de la unidad de aprendizaje con base en los créditos que ésta tiene.}{A}{Fórmula/Usuario} 
		
		\FRitem{RF2}{Formatos PDF}{El sistema exporta la ``Descripción General de la Unidad de Aprendizaje'' y el ``Programa de la Unidad de Aprendizaje'' en formato PDF.}{M}{Usuario}
		
		\FRitem{RF3}{Notificaciones de estado}{El sistema envía notificaciones al usuario del estado de la creación de la Unidad de Aprendizaje.}{A}{Usuario}
		
		\FRitem{RF4}{Guardar la información}{El sistema guarda los avances de la elaboración de la Descripción General de la Unidad de Aprendizaje.}{A}{Usuario}
		
		\FRitem{RF5}{Notificaciones de correcciones}{El sistema notifica al usuario cuando un documento tiene correcciones.}{A}{Usuario}
		
		\FRitem{RF6}{Verificación de horas}{El sistema verifica que las horas asignadas por semana coincidan con el total de horas por semestre.}{A}{Usuario}
		
		\FRitem{RF7}{Verificación de porcentajes}{El sistema verifica que las horas asignadas por semana coincidan con el total de horas por semestre.}{A}{Usuario}
		
		\FRitem{RF8}{Comentarios}{El sistema permite realizar comentarios acerca de la elaboración de la ``Descripción General de la Unidad de Aprendizaje'' y el ``Programa de la Unidad de Aprendizaje''. }{A}{Usuario}
		
		\FRitem{RU1}{Elaboración de la Descripción General de la Unidad de Aprendizaje}{El usuario ingresa la Descripción General de la Unidad de Aprendizaje.}{A}{Usuario}
		
		\FRitem{RU2}{Consulta de la Descripción General de la Unidad de Aprendizaje}{El usuario consulta la Descripción General de la Unidad de Aprendizaje}{M}{Usuario}
		
		\FRitem{RU3}{Actualización de la Descripción General de la Unidad de Aprendizaje}{El usuario actualiza la Descripción General de la Unidad de Aprendizaje}{A}{Usuario}
		
		\FRitem{RU4}{Elaboración del Programa de la Unidad de Aprendizaje}{El usuario ingresa el Programa de la Unidad de Aprendizaje}{A}{Usuario}
		
		\FRitem{RU5}{Consulta del Programa de la Unidad de Aprendizaje}{El usuario consulta el Programa de la Unidad de Aprendizaje}{M}{Usuario}
		
		\FRitem{RU6}{Actualización del Programa de la Unidad de Aprendizaje}{El usuario actualiza el Programa de la Unidad de Aprendizaje}{A}{Usuario}
		
	\end{requerimientos}
    \caption{Requerimientos funcionales del sistema para la creación de una Unidad de Aprendizaje}
    \label{tbl:RFUA}
\end{table}

