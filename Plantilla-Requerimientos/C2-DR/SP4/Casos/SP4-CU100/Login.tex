\begin{UseCase}[marker]{SP4-CU1}{01}
  {
      Descripción larga e informal del Use Case
  }
  \UCitem{Versión}{Versión del Use Case}
  \UCitem{Super UC}{Use Case del que hereda}
  \UCitem{Propósito o Justificación}{Razón del use case del Use Case}
  \UCitem{Objetivo}{Objetivo final del Use Case}
  \UCitem{Precondiciones}{Estado inicial necesario para ejecutar el Use Case}
  \UCitem{Entradas}{Datos de entrada: Nombre, tipo y descripción}
  \UCitem{Salidas}{Datos de salida o resultados esperados}
  \UCitem{Resumen}{Se identifica al usuario por medio de su Login y Password}
  \UCitem{PostcondiciDocumentos, personas o ones}{Estado final del sistema después del Use Case}
  \UCitem{Restricciones}{Requerimientos no funcionales que debe cumplir}
  \UCitem{Suposiciones}{Hechos que se consideran reales para la descripción de este Use Case}
  \UCitem{Área}{Área o grupo a la que pertenece el Use Case}
  \UCitem{Referencias}{Documentos, personas o especificaciones que proporcionaron información para la especificación de este UseCase}
  \UCitem{Autor}{Nombre del responsable de la especificación del Use Case o de las modificaciones de la versión}
  \UCitem{Fecha}{Fecha de el último cambio al Use Case}

\end{UseCase}

% \UCEDdef{condición}{región}{casos de uso a los que extiende}
% \begin{UCtrayectoria}{prefijo}{nombre}
% 	\UCpaso[\UCActor] El actor oprime el boton buscar
% 	\UCpaso[\UCsist] El sistema ...
% \end{UCtrayectoria}
% \begin{UCtrayectoriaA}{Id}{Condicion}
% 	\UCpaso[\UCActor] El actor oprime el boton buscar
% 	\UCpaso[\UCsist] El sistema ...
% \end{UCtrayectoriaA}