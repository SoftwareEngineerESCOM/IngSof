 \begin{UseCase}[marker]{SP4-CU5}{Consultar Programas Académicos}
	{
		El usuario Jefe de Innovación Educativa visualiza la informacion
de los Programas Académicos.
	}
	\UCitem{Version}{1}
	\UCitem{Super UC}{Ninguno}
	\UCitem[\Actor]{Principal}{Jefe de Innovación Educativa desea visualizar en pantalla los Programas Académicos almacenados en el Sistema.}
	\UCitem[\ActorS]{Secundarios}{Ninguno} 
	\UCitem{Propósito o Justificación}{El Usuario Jefe de Innovación Educativa visualice los Programas Académicos almacenados en el Sistema.}
	\UCitem{Objetivo}{Mostrar en pantalla una lista de los Programas Académicos.}
	\UCitem{Precondiciones}{Tener los catálogos llenos.}
	\UCitem{Entradas}{Datos de entrada: Ninguna}
	\UCitem{Salidas}{
					 MSG4. ¿Seguro que desea editar el Programa Académico?
					 MSG5. ¿Seguro que desea cancelar la consulta de Programas Académicos?
					 MSG6. ¿Seguro que desea registrar un Programa Académico?
					 
}
	\UCitem{Resumen}{El usuario Jefe de Innovación Educativa visualiza una lista de Programas Académicos ya registrados, en esta lista visualiza el nombre del Programa Académico y al lado del nombre un botón  que le permita ir a SP4-CU9 "Editar Programa Académico" del Programa Académico correspondiente.}
	\UCitem{Postcondiciones}{Ninguna}
	\UCitem{Restricciones}{Ninguna}
	\UCitem{Prioridad}{\Square Alta }
	\UCitem{Suposiciones}{El Usuario Jefe de Innovación Educativa visualiza los Programas Académicos en una lista.}
	\UCitem{Área}{Casos de Uso del Jefe de Innovación Educativa}
	\UCitem{Referencias}{Modelo de Datos,SP4-U5}
	\UCitem{Autor}{Josué Eliasaf Plata García}
	\UCitem{Fecha}{24/10/2018}
 \end{UseCase}

Trayectoria Principal
1.	El sistema Muestra la interfaz de usuario IU5 ConsultarProgramaAcadémico.
2.	El sistema muestrala lista de Programas Académicos registrados. [Trayectoria A][Trayectoria B][Trayectoria C][Trayectoria D][Trayectoria E][Trayectoria F]

Trayectoria Alternativa A.
Condición: El usuario presiona el botón cancelar.
1.	El usuario presiona el botón cancelar.
2.	El sistema muestra el mensaje MSG5. ¿Seguro que desea cancelar la consulta de Pogramas Académicos?
3.	El usuario confirma la operación presionando sí.
4.	El sistema muestra la interfaz de usuario IU 2 “Consultar Tareas”
-	- - - - Fin de la trayectoria

Trayectoria Alternativa B
Condición: El usuario por error presiona el botón cancelar.
1.	El usuario presiona el botón cancelar.
2.	El sistema muestra el mensaje MSG5. ¿Seguro que desea cancelar la consulta de Pogramas Académicos?
3.	El usuario cancela la operación presionando no.
4.	El sistema cierra el mensaje.
5.	Continua con el paso 1 de la trayectoria principal del CU5.

Trayectoria Alternativa C.
Condición: El usuario presiona el botón Registrar Programa Académico.
1.	El usuario presiona el botón Registrar.
2.	El sistema muestra el mensaje MSG6. ¿Seguro que desea registrar un Programa Académico?
3.	El usuario confirma la operación presionando sí.
4.	El sistema muestra la interfaz de usuario IU 2.4.1 “Registrar Programas Académicos”
-	- - - - Fin de la trayectoria

Trayectoria Alternativa D
Condición: El usuario por error presiona el botón  Registrar Programa Académico.
1.	El usuario presiona el botón cancelar.
2.	El sistema muestra el mensaje MSG6. ¿Seguro que desea registrar un Programa Académico?
3.	El usuario cancela la operación presionando no.
4.	El sistema cierra el mensaje.
5.	Continua con el paso 1 de la trayectoria principal del CU5.
-	- - - - Fin de la trayectoria

Trayectoria Alternativa E.
Condición: El usuario presiona el botón Editar.
1.	El usuario presiona el botón Editar.
2.	El sistema muestra el mensaje MSG4. ¿Seguro que desea editar el Programa Académico?
3.	El usuario confirma la operación presionando sí.
4.	El sistema muestra la interfaz de usuario IU 2.4.1 “Editar Programas Académicos”
-	- - - - Fin de la trayectoria

Trayectoria Alternativa F
Condición: El usuario por error presiona el botón Editar.
1.	El usuario presiona el botón cancelar.
2.	El sistema muestra el mensaje MSG4. ¿Seguro que desea editar el Programa Académico?
3.	El usuario cancela la operación presionando no.
4.	El sistema cierra el mensaje.
5.	Continua con el paso 1 de la trayectoria principal del CU5.
-	- - - - Fin de la trayectoria



