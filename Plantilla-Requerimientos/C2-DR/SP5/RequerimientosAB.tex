% -------------- TABLA PARA REQUERIMIENTOS FUNCIONALES ---------------- % 
% Nomenclatura para la prioridad: 
%	A - Alta
%	M - Media
%	B - Baja

\textbf{Nota:} Se usa :
	\begin{itemize}
		\item \textbf{JDIC:} Jefe de departamento de Innovación curricular
		\item \textbf{JDIA:} Jefe de división de innovación académica
		\item \textbf{DES:} Directora de Educación Superior
	\end{itemize}

\begin{table}[htbp!]
	\begin{requerimientos}
		
		\FRitem{VMP-01}{Registrar empleados}{El sistema permitirá al usuario JDIA registrar usuarios JDIC y al JDCI registrar analistas y jefes de innovación educativa}{A}{}
		\FRitem{VMP-02}{Consultar empleados}{El sistema permitirá al usuario JDIA y JDIC  consultar la información y trabajo activo de los usuarios analista y jefe de innovación educativa}{M}{}
		\FRitem{VMP-03}{Eliminar empleados}{El sistema permitirá al usuario JDIA eliminar usuarios JDIC y al JDCI eliminar analistas y jefes de innovación educativa}{B}{}
		\FRitem{VMP-04}{Consultar asignaciones del analista}{El Sistema le mostrará al usuario analista los tareas  que tiene asignadas }{A}{Origen}
		\FRitem{VMP-05}{Consultar Mapas curriculares}{El JDIA y JDIC consultarán la lista de los mapas curriculares por nombre de la unidad académica o todos los mapas curriculares registrados}{M}{}
		\FRitem{VMP-06}{Ver Mapa curricular}{El sistema mostrará al JDIA y JDIC la información de un mapa curricular previamente seleccionado}{}{}
		\FRitem{VMP-07}{Asignar analista}{El JDIC seleccionará (de la lista de usuarios analistas y él incluido ) el empleado (visualizando su nombre y el número de tareas que tiene asignadas) que revisará el mapa curricular previamente elegido}{M}{}
		\FRitem{VMP-08}{Atender mapa curricular por parte de la JDIC}{El sistema permitirá a la JDIC analizar, junto a los analistas, uno o más mapas curriculares recibidos por las Unidades Académicas}{A}{}
		\FRitem{VMP-09}{Comunicar observaciones del mapa curricular a los analistas}{El sistema permitirá a la JDIC contactar al analista encargado de un mapa curricular para comunicarle observaciones, correcciones, notas, etc respecto a éste}{A}{}
		\FRitem{VMP-10}{Realizar modificaciones al mapa curricular}{El sistema permitirá a la JDIC realizar correcciones y ajustes al mapa curricular junto a su analista asignado, y llegar a un acuerdo.}{A}{}
		\FRitem{VMP-11}{Delegar tareas a los analistas}{El sistema permitirá a la JDIC delegar la tarea de revisar un mapa curricular a los analistas}{A}{}
		\FRitem{VMP-12}{Verificar nombre de unidades de aprendizaje}{El sistema permitirá a la JDIC y a la JDIA aprobar o rechazar el nombre que se le asignó a las unidades de aprendizaje registradas}{A}{}
		\FRitem{VMP-13}{Aprobar Mapa curricular}{El mapa curricular enviado por la Unidad Académica debe estar completo para que pueda ser aprobado por la JDIC en el sistema}{A}{}
		\FRitem{VMP-14}{Enviar mapa curricular junto a unidades de aprendizaje}{El sistema permitirá a la JDIC notificar a la Unidad Académica que pueden subir su mapa curricular y trabajar sus semestres al mismo tiempo}{M}{}
	\end{requerimientos}
    \caption{Requerimientos funcionales del sistema.}
    \label{tbl:}
\end{table}
